\chapter{Optimale Zustandsrückführung}
\section{Zeitdiskrete Optimalsteuerung bei endlichen Zeithorizont}


\section{Das zeitdiskrete \NoCaseChange{\acl{LQR}}-Problem}

\section{Zeitkontinuierliche Optimalsteuerng bei endlichen Zeithorizont}
\label{sec:3_3_zeitkont_opt_endl}

\section{Das zeitkontinuierliche \NoCaseChange{\acl{LQR}}-Problem}

\section{Beispiel der räumlichen Führung (\ac{PN})}
\subsection{Modellbildung}
\begin{figure}[htb]
	\centering
	\input{tikz/dummy}
	\caption{Geometrie des Problems der räumlichen Führung}
	\label{fig:kap_3_bsp_rf_geometrie}
\end{figure}
Die Geometrie des Problems der räumlichen Führung  soll vereinfachend in der Ebene betrachtet werden (siehe \figref{fig:kap_3_bsp_rf_geometrie}). In $(r,\sigma)$-Koordinaten wird es
beschrieben durch
\begin{align}
	\begin{split}\label{eqn:kap_3_bsp_rf_rskoord}
		r\dot{\sigma} & = v_1\sin(\varphi - \sigma) - v_2\sin(\gamma - \sigma)\\
		\dot{r} & = v_1\cos(\varphi - \sigma) - v_2\cos(\gamma - \sigma)
	\end{split}
\end{align}
und in $(x,y)$-Koordinaten durch
\begin{align}
	\begin{aligned}\label{eqn:kap_3_bsp_rf_xykoord}
		\dot{x}_2 & = v_2 \cos \gamma  & \qquad \dot{x}_1 & = v_1 \cos \varphi\\
		\dot{y}_2 & = v_2 \sin \gamma  & \qquad \dot{y}_1 & = v_1 \sin \varphi.
	\end{aligned}
\end{align}
Gegeben sind der Gierwinkel $\varphi(t)$, die Geschwindigkeiten $v_1(t)$ und $v_2(t)$ sowie die Anfangswerte $x_1(0)$, $y_1(0)$, $x_2(0)$, $y_2(0)$. Es gelte $v_1(t) < v_2(t)$. Mit
\begin{align}\label{eqn:kap_3_bsp_rf_sigma}
\sigma & = \left\{ \begin{array}{rl}
	\arcsin\frac{y_1-y_2}{r}	& \text{für } x_1\geq x_2\\
	\pi-\arcsin\frac{y_1-y_2}{r} & \text{für } x_1 < x_2
\end{array} \right.
\end{align}
und
\begin{align}
	r & = \sqrt{ (x_1 - x_2)^2 + (y_1 - y_2)^2 } \label{eqn:kap_3_bsp_rf_regelgesetz}
\end{align}
werden die Anfangswerte $r(0)$ und $\sigma(0)$ für \eqnref{eqn:kap_3_bsp_rf_rskoord} berechnet (zum Wertebereich von $\sigma$ siehe \figref{fig:kap_3_bsp_rf_wertebereich}). Die Funktion
$\gamma(t)$ dient als Steuerfunktion. Im Falle der sogenannten Proportionalnavigation wird das Regelgesetz
\begin{align}
	\dot{\gamma} & = a \dot{\sigma}
\end{align}
verwendet, wobei für die Proportionalverstärkung $a\geq 1$ gilt.

Das Annäherungsverhalten kann mit \eqnref{eqn:kap_3_bsp_rf_rskoord} zusammen mit \eqnref{eqn:kap_3_bsp_rf_regelgesetz} simuliert werden (ergibt drei Zustandsvariablen). Soll die
Flugbahn in die Kartenebene gezeichnet werden, so wird man für die Simulation \eqnref{eqn:kap_3_bsp_rf_xykoord} zusammen mit \eqnref{eqn:kap_3_bsp_rf_regelgesetz} (ergibt fünf
Zustandsvariablen), wobei die erforderliche Messgröße $\dot{\sigma}$ entweder aus \eqnref{eqn:kap_3_bsp_rf_rskoord} (weitere zwei Zustandsvariablen) oder ohne Vergrößerung der Anzahl
der Zustandsvariablen wie folgt berechnet wird. Mit
\begin{align}
	\zeta & := \begin{bmatrix}
	y_2 - y_1\\
	x_1 - x_2
	\end{bmatrix},\quad \zeta_0:=\frac{\zeta}{\left\| \zeta \right\|_2},\quad \alpha_1:=\zeta_0^T\begin{bmatrix}
	\dot{x}_1 \\ \dot{y}_1
	\end{bmatrix},\quad \alpha_2:=\zeta_0^T\begin{bmatrix}
	\dot{x}_2\\ \dot{y}_2
	\end{bmatrix}
\end{align}
und wegen $\alpha_1 = v_1 \sin(\varphi - \sigma)$, $\alpha_2 = v_2 \sin(\gamma - \sigma)$ (siehe \figref{fig:kap_3_bsp_rf_geschwindigkeit}), $r = \left\|\zeta \right\|_2$ und
\eqnref{eqn:kap_3_bsp_rf_rskoord} erhält man
\begin{align}
	\dot{\sigma} & = \frac{\zeta_0^T\left( \begin{bmatrix}
	\dot{x}_1 \\ \dot{y}_1
	\end{bmatrix} - \begin{bmatrix} 
	\dot{x}_2 \\ \dot{y}_2
	\end{bmatrix}\right)}{\left\| \zeta \right\|_2}.	\label{eqn:kap_3_bsp_rf_messgroesse}
\end{align}
\begin{figure}[htb]
	\centering
	\input{tikz/dummy}
	\caption{$\sigma$-Wertebereich für $x_2=y_2=0$ und $(x_1,y_1)$ auf dem Kreis $\mathcal{K}$}
	\label{fig:kap_3_bsp_rf_wertebereich}
\end{figure}
\begin{remark}
Die Berechnung der Ableitung erfolgt in \eqnref{eqn:kap_3_bsp_rf_messgroesse} ohne die Verwendung von Winkelfunktionen. Alternativ kann $\dot{\sigma}$ auch über
$r\dot{\sigma}=v_1\sin(\varphi - \sigma)-v_2\sin(\gamma - \sigma)$ bestimmt werden. Das dafür nötige $\sigma$ wird mittels \eqnref{eqn:kap_3_bsp_rf_sigma} bestimmt. Das dabei nur die
Einschränkung der Sinus-Funktion auf $\left[-\frac{\pi}{2},\frac{3}{2}\pi \right]$ betrachtet wird und damit bei einer kontinuierlichen Bewegung des Körpers 2 auf einer Kreisbahn um den Körper 1 bei
$\sigma = -\frac{\pi}{2}$ ein Sprung der Höhe $2\pi$ auftritt (\figref{fig:kap_3_bsp_rf_wertebereich}), ist unproblematisch, da dieser Sprung durch die Periodizität der Sinunsfunktion
nicht in den Funktionswerten $\sin(\varphi - \sigma)$ und $\sin(\gamma - \sigma)$ erscheint.
\end{remark}
\begin{figure}[htb]
	\centering
	\input{tikz/dummy}
	\caption{Komponenten der Geschwindigkeitsvektoren}
	\label{fig:kap_3_bsp_rf_geschwindigkeit}
\end{figure}

\subsection{Parallele Annäherung als Spezialfall der \ac{PN}}