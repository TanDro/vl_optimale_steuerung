\section*{Übungen Kapitel 1}
\addcontentsline{toc}{section}{Übungen Kapitel 1} 
\label{sec:uebung_kapitel_1}

\subsection*{Definitheit und Extremstellen}
\addcontentsline{toc}{subsection}{Übung - Definitheit und Extremstellen}
\label{sec:uebung_kapitel_1_def_extrem} 
Bestimmen Sie von beiden Funktionen die Definitheit und die Extremstellen!
\begin{alignat*}{2}
f_1(x_1,x_2) & = \frac12\left(ax_1^2+bx_2^2\right) &\Rightarrow H f_1&=\begin{bmatrix} a & 0\\ 0 & b \end{bmatrix}\\
f_2(x_1,x_2) & = \frac12x_1^2+\frac16x_2^3 & \Rightarrow H f_2&=\begin{bmatrix} 1 & 0\\ 0 & x_2 \end{bmatrix}
\end{alignat*}

\subsubsection{Lösung}
 
\subsection*{Kuhn-Tucker-Bedingung}
\addcontentsline{toc}{subsection}{Kuhn-Tucker-Bedingung}
\label{sec:uebung_kapitel_1_ktb} 
Finden Sie alle Lösungen von $f(x)=-x_1-x_2\rightarrow\min\limits_{x\in\mathbb{R}^2}$ bei $g_1(x)=x_1+x_2^2\le 0$! Skizzieren Sie ihr Ergebnis.

\subsubsection{Lösung} 

\subsection*{Beispiele zur Quadratischen Optimierung}
\addcontentsline{toc}{subsection}{Beispiele zur Quadratischen Optimierung}
\label{sec:uebung_bsp_quad_opt}
Seien $Q=\begin{bmatrix}
1 & 1\\ 1 & 2
\end{bmatrix}$, $q=\begin{bmatrix}
2\\1
\end{bmatrix}$, $C = \begin{bmatrix}
1 & 1
\end{bmatrix}$ und $d = 1$. Lösen Sie die folgenden Aufgaben
\begin{enumerate}[label=\alph*)]
  \item $\min\limits_{x\in\mathbb{R}^n}\frac12 x^TQx + q^T x$
  \item $\min\limits_{x\in\mathbb{R}^n}\frac12 x^TQx +q^T x$ bei $Cx=d$
\end{enumerate}

\subsubsection{Lösung}

\subsection*{Projektion auf Untervektorraum}
\addcontentsline{toc}{subsection}{Projektion auf Untervektorraum}
\label{sec:uebung_proj_untervekraum} 
Seien $A=\begin{bmatrix}
							2\\1
							\end{bmatrix}$, $b=\begin{bmatrix}
							1\\2
							\end{bmatrix}$. Berechnen Sie 							
\begin{enumerate}[label=\alph*)]
	\item $Pb$. 
	\item $(I-P)b$.
	\item $(I-P)P$.
	\item $Pb+(I-P)b$.
	\item $(Pb)^T(I-P)b$.
	\item Was ist das kleinste $\alpha\in\mathbb{R}^1$, so dass $\|Pb\|_2\le\alpha\|b\|_2$ erfüllt ist? 
\end{enumerate}  

\subsubsection{Lösung} 

\subsection*{Projektion eines Vektors}
\addcontentsline{toc}{subsection}{Projektion eines Vektors}
\label{sec:uebung_proj_vektor} 
\begin{enumerate}[label=\alph*)]
  \item Für $C=\begin{bmatrix}1 & 1 \end{bmatrix}$, $d=1$ projiziere $v_1=\begin{bmatrix}1 \\ 1 \end{bmatrix}$, $v_2=\begin{bmatrix}1 \\ 0 \end{bmatrix}$ auf $M$ (Skizze)!
  \item Für $C=\begin{bmatrix}1 & 1\\ 0 & 1 \end{bmatrix}$, $d=\begin{bmatrix} 1\\1\end{bmatrix}$ projiziere $v_1=\begin{bmatrix}1 \\ 1 \end{bmatrix}$, $v_2=\begin{bmatrix}1 \\ 0
  \end{bmatrix}$ auf $M$ (Skizze)!
\end{enumerate}

\subsubsection{Lösung} 

\subsection*{Finden eines zulässigen Startwertes für den Aktiven Mengen Algorithmus}
\addcontentsline{toc}{subsection}{Finden eines zulässigen Startwertes für den Aktiven Mengen Algorithmus}
\label{sec:uebung_finden_startwert} 
Erweitern Sie den Algorithmus um die Auffindung eines zulässigen Startpunktes!
 
\subsubsection{Lösung}
 
\subsection*{Erweiterung des Aktiven Mengen Algorithmuss}
\addcontentsline{toc}{subsection}{Erweiterung des Aktiven Mengen Algorithmus}
\label{sec:uebung_erweiterung_algo} 
Erweitern Sie den Algorithmus um die Aufgabe der linearen Optimierung und die Verwendung von GNB als aktive UNB!

\subsubsection{Lösung} 

\subsection*{Aktiver Mengen Algorithmus}
\addcontentsline{toc}{subsection}{Aktiver Mengen Algorithmus}
\label{sec:uebung_ama} 
Wiederholen Sie das \exmpref{exmp:kap_1_ama} mit dem Aktiven Mengen Algorithmus mit dem Startwert $x^0:=\begin{bmatrix}
1\\ -1
\end{bmatrix}$.
\subsubsection{Lösung} 

