\section*{Übungen Kapitel 1}
\addcontentsline{toc}{section}{Übungen Kapitel 1} 
\label{sec:uebung_kapitel_1}

\subsection*{Übung - Definitheit und Extremstellen}
\addcontentsline{toc}{subsection}{Übung - Definitheit und Extremstellen}
\label{sec:uebung_kapitel_1_def_extrem} 
Bestimmen Sie von beiden Funktionen die Definitheit und die Extremstellen!
\begin{alignat*}{2}
f_1(x_1,x_2) & = \frac12\left(ax_1^2+bx_2^2\right) &\Rightarrow H f_1&=\begin{bmatrix} a & 0\\ 0 & b \end{bmatrix}\\
f_2(x_1,x_2) & = \frac12x_1^2+\frac16x_2^3 & \Rightarrow H f_2&=\begin{bmatrix} 1 & 0\\ 0 & x_2 \end{bmatrix}
\end{alignat*}

\subsubsection{Lösung}
 
\subsection*{Übung - Kuhn-Tucker-Bedingung}
\addcontentsline{toc}{subsection}{Übung - Kuhn-Tucker-Bedingung}
\label{sec:uebung_kapitel_1_ktb} 
Finden Sie alle Lösungen von $f(x)=-x_1-x_2\rightarrow\min\limits_{x\in\mathbb{R}^2}$ bei $g_1(x)=x_1+x_2^2\le 0$! Skizzieren Sie ihr Ergebnis.

\subsubsection{Lösung} 

\subsection*{Übung - Quadratische Optimierung}
\addcontentsline{toc}{subsection}{Übung - Quadratische Optimierung}
\label{sec:uebung_kapitel_1_quadopt} 
Zeigen Sie, dass für $\min\limits_{x\in\mathbb{R}}$, d.h. $Q=0$, $q=1$, keine Lösung exisiert.

\subsubsection{Lösung} 

\subsection*{Übung - ?}
\addcontentsline{toc}{subsection}{Übung - ?}
\label{sec:uebung_kapitel_1_?} 
Wiederholen Sie den Algorithmus aus dem Beispiel mit $x^0=\begin{bmatrix}1\\ -1 \end{bmatrix}$. 

\subsubsection{Lösung} 

\subsection*{Übung - Projektion}
\addcontentsline{toc}{subsection}{Übung - Projektion}
\label{sec:uebung_kapitel_1_projektion} 
\begin{enumerate}[label=\alph*)]
	\item Seien $A=\begin{bmatrix}
							2\\1
							\end{bmatrix}$, $b=\begin{bmatrix}
							1\\2
							\end{bmatrix}$. Berechnen Sie $Pb$, $(I-P)b$, $(I-P)P$, $Pb+(I-P)b$, $(Pb)^T(I-P)b$.
	\item Was ist das kleinste $\alpha\in\mathbb{R}$, so dass $\|Pb\|_2\le\alpha\|b\|_2$ erfüllt ist? 
  \item Für $C=\begin{bmatrix}1 & 1 \end{bmatrix}$, $d=1$ projiziere $v_1=\begin{bmatrix}1 \\ 1 \end{bmatrix}$, $v_2=\begin{bmatrix}1 \\ 0 \end{bmatrix}$ auf $M$ (Skizze)!
  \item Für $C=\begin{bmatrix}1 & 1\\ 0 & 1 \end{bmatrix}$, $d=\begin{bmatrix} 1\\1\end{bmatrix}$ projiziere $v_1=\begin{bmatrix}1 \\ 1 \end{bmatrix}$, $v_2=\begin{bmatrix}1 \\ 0
  \end{bmatrix}$ auf $M$ (Skizze)!
\end{enumerate}

\subsubsection{Lösung} 

\subsection*{Übung - Aktive-Mengen-Strategie}
\addcontentsline{toc}{subsection}{Übung - Aktive-Mengen-Strategie}
\label{sec:uebung_kapitel_1_ams} 
\begin{enumerate}[label=\alph*)]
  \item Erweitern Sie den Algorithmus um die Auffindung eines zulässigen Startpunktes!
  \item Erweitern Sie den Algorithmus um die Aufgabe der linearen Optimierung und die Verwendung von GNB als aktive UNB!
\end{enumerate}

\subsubsection{Lösung} 
