\section*{Übungen Kapitel 2}
\addcontentsline{toc}{section}{Übungen Kapitel 2} 
\label{sec:uebung_kapitel_2}

\subsection*{Vorgehensweise zur analytischen Bestimmung der Optimallösung}
\addcontentsline{toc}{subsection}{Vorgehensweise zur analytischen Bestimmung der Optimallösung}
\label{sec:uebung_anal_best_opt_lsg}
\begin{enumerate}
  \item Bestimmen Sie die optimale Steuerfunktion zu folgenden Problemen
  \begin{enumerate}
    \item $J = \int\limits_0^1 u(t)^2dt\rightarrow\min$ bei $\dot{x}=u-c$ mit gegebenen Konstanten $c>0$ sowie $x(0)=0$ und $x(1)=1$.
    \item $J = \int\limits_0^{t_b} u(t)^2dt\rightarrow\min$ bei $\dot{x}=u-c$ mit gegebenen Konstanten $c>0$ sowie $x(t_b)=0$ und $x(1)=1$, $t_b$ ist
    frei.
  \end{enumerate}
  Diskutieren Sie für die beiden Probleme jeweils die Abhängigkeit der Lösung, d.h. von $u$ und $J$, vom Parameter $c$. Vergleichen Sie die Ergebnisse
  der Lösung der beiden Probleme miteinander. 
  \item Bestimmen Sie eine optimale Steuerfunktion für die Aufgabenstellung
  \begin{align*}
  	J & = \int\limits_0^1 u_1(t)^2+u_2(t)^2 dt\rightarrow\min
  \end{align*} 
  bei
  \begin{align*}
  	\dot{x}_1 & = u_1,\ \dot{x}_2=u_2,\ \dot{x}_3=x_1,\ \dot{x}_4=x_2\\
  	x_3(1)-x_4(1) & = 0
  \end{align*}
  für die Anfangszustände
  \begin{enumerate}
    \item $x_1(0) = x_2(0)=x_3(0)=x_4(0)=0$,
    \item $x_1(0) = 2,\ x_2(0)=0,\ x_3(0)=1,\ x_4(0)=0$.
  \end{enumerate}
\end{enumerate}

\subsection*{Vorgehensweise zur numerischen Bestimmung der Optimallösung}
\addcontentsline{toc}{subsection}{Vorgehensweise zur numerischen Bestimmung der Optimallösung}
\label{sec:uebung_num_best_opt_lsg}
Implementieren Sie den Algorithmus und testen Sie ihn am \exmpref{exmp:kap_2_vor_optlsg_2}. Beachten Sie, dass eine
sinvolle Wahl von $\alpha$ für ein Kovergenzverhalten notwendig ist.
