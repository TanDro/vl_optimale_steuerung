\addchap[Symbolverzeichnis]{Symbolverzeichnis}
\begin{acronym}[LabVIEW] %<--in Klammern das laengste Wort
	\acro{Notation}{\textbf{Bedeutung}}
	\acro{dotx}[\ensuremath{\dot{x},\ \ddot{x}\ bzw.\ x^{(k)}}]{erste, zweite bzw. k-te Ableitung}
	\acro{tx}[\ensuremath{\tilde{x}}]{Schätzvektor}
	\acro{hx}[\ensuremath{\hat{x}}]{Schätzfehlervektor}
	\acro{hx}[\ensuremath{e}]{Folgefehlervektor}
	\acro{Symbol}{\textbf{Bedeutung}}
	\acro{RN}[\ensuremath{\mathbb{R},\mathbb{N}}]{Menge der reellen bzw. natürlichen Zahlen}
	\acro{t}[\ensuremath{t}]{Zeit}
	\acro{x}[\ensuremath{\mathbold{x}}]{Zustandsvektor}
	\acro{z}[\ensuremath{\mathbold{z}}]{transformierte Zustandsvektor}
	\acro{u}{Systemeingang}
	\acro{v}{virtueller Eingang}
	\acro{faus}[\ensuremath{\mathbold{y}_f}]{flacher Ausgang}
	\acro{yhx}[\ensuremath{y = h(x)}]{Systemausgang, bzw. Regelgröße}
	\acro{f}[\ensuremath{f}]{System-Vektorfelder}
	\acro{A}[\ensuremath{A}]{Systemmatrix (lin. System)}
	\acro{n}[\ensuremath{n}]{Systemordnung}
	\acro{r}[\ensuremath{r}]{relatriver Grad}
	\acro{s}[\ensuremath{s}]{Gleitfläche}
	\acro{tau}[\ensuremath{\vect{\tau}}]{Tangentialvektor}
	\acro{nu}[\ensuremath{\vect{\nu}}]{Normalenvektor}
\end{acronym}