\chapter{Endlichdimensionale Optimierungsprobleme}
\begin{figure}[htb]
	\centering
	\begin{tikzpicture}[auto, scale=0.65, every node/.style={scale=0.65}, node distance=2.0cm,>=latex']
	\node [input, node distance=2.5cm] (input) {};
	\node [block, node distance=2.5cm, right of=input] (prozess) {$P$};
	\node [output, node distance=2.5cm, right of=prozess] (output) {};

	\draw [->] (input) -- node {$u\in U$} (prozess);
	\draw [->] (prozess) -- node {$y\in Y$}  (output);
		
\end{tikzpicture}

	\caption{Steuerung}
	\label{fig:kap_1_steuerung}
\end{figure}
Eingänge $u$ können in ihren möglichen Werten auf eine Menge möglicher Werte $U$ beschränkt werden.
\begin{align*}
K(u,y) & \rightarrow \min\limits_u\rightsquigarrow u^{\star}\\
K & \ldots \text{Kostenfunktion}\\
u^{\star} & \ldots \text{Steuerfunktion}
\end{align*}
Optimierung bzgl. eines zeitlich beschränkten Zeitraumes:\\
Möglichkeit
\begin{itemize}
  \item Ausdehnung auf unendliche Zeit
  \item Verschiebung des Intervalls $[t_0,t_e]$ nach jedem Zeitpunkt
\end{itemize}
um einen Übergang zu einer kontinuierlichen Regelung zu ermöglichen.
\begin{figure}[htb]
	\centering
	\begin{tikzpicture}[auto, scale=0.65, every node/.style={scale=0.65}, node distance=2.0cm,>=latex']
	\draw[->] (0,0) -- (6,0) node[anchor=north] {$t$};
	\draw (0,0) node[anchor=north] {$t_a$}
		  (3,0) node[anchor=north] {$t_e$};
	\draw (0, 0.1) -- (0,-0.1);
	\draw (3, 0.1) -- (3,-0.1);
	
	\draw (12,0) node{endlicher Optimierungshorizont}
		  (12,-2) node{mitführen des Optimierungshorizontes};
		  
	\draw (0,-1) -- (3,-1);
	\draw (0, -0.9) -- (0,-1.1);
	\draw (3, -0.9) -- (3,-1.1);
	
	\draw (1,-2) -- (4,-2);
	\draw (1, -1.9) -- (1,-2.1);
	\draw (4, -1.9) -- (4,-2.1);

	\draw (2,-3) -- (5,-3);
	\draw (2, -2.9) -- (2,-3.1);
	\draw (5, -2.9) -- (5,-3.1);
\end{tikzpicture}

	\caption{Optimierungshorizont}
	\label{fig:kap_1_horizont}
\end{figure}
\section{Wiederholung aus der Analysis: Optimierung unter Nebenbedingungen}
\begin{defi}
Sei $F:D\subset\mathbb{R}^n\rightarrow\mathbb{R}$. Eine Stelle $x_0\in D$ heißt
\begin{itemize}
  \item globale Minimumstelle von $f$ auf $D$, wenn $f(x_0)\le f(x)\forall x\in D$
  \item lokale Minimumstelle von $f$ auf $D$, wenn Umgebung $U_{\epsilon}(x_0):=\left\{y\in\mathbb{R}^n|\ \|y-y_0\|<\epsilon\right\}$ von $x_0$ existiert, so dass $f(x_0)\le
  f(x)\forall x\in D\cap U_{\epsilon}(x_0)$
  \item isolierte lokale Minimumstelle von $f$ auf $D$, wenn Umgebung $U_{\epsilon}(x_0)$ existiert, so dass $f(x_0)<f(x)\forall x\in D\cap U_{\epsilon}(x_0), x\neq
  x_0$
  \item Analog Maximum mit $\ge$ statt $\le$ und $>$ statt $<$.
  \item Minimumstellen und Maximumstellen treten gemeinsam auf: Extremstellen
\end{itemize}
\end{defi}
\begin{exmp}\hspace{1cm}
\begin{minipage}[c][][c]{0.3\textwidth}
\centering
\begin{tikzpicture}[auto, scale=0.65, every node/.style={scale=0.65}, node distance=2.0cm,>=latex']
	\draw[->] (0,0) -- (6,0);
	\draw	(1,0) node[anchor=north] {$a$}
			(2,0) node[anchor=north] {$x_1$}
			(3,0) node[anchor=north] {$x_2$}
			(4,0) node[anchor=north] {$x_3$}
			(5,0) node[anchor=north] {$b$};

	\draw[->] (0,0) -- (0,4);
	
	\draw[dotted] (1,0) -- (1,4);
	\draw[dotted] (2,0) -- (2,4);
	\draw[dotted] (3,0) -- (3,4);
	\draw[dotted] (4,0) -- (4,4);
	\draw[dotted] (5,0) -- (5,4);
	
	\draw [black] plot [smooth] coordinates {(1,1) (2,3.5) (3,0.75) (4,3.45) (4.5,3.75) (5,4)};
\end{tikzpicture}

\end{minipage}
\hfill
\begin{minipage}[c][][c]{0.7\textwidth}
\begin{align*}
f &: [a,b]\subset\mathbb{R}\rightarrow \mathbb{R}\\
a &: \text{isolierte lokale Minimumstelle}\\
x_1 &: \text{isolierte lokale Maximumstelle}\\
x_2 &: \text{globale und isolierte lokale Minimumstelle}\\
x_3 &: \text{lokale Minimum- und Maximumstelle (nicht isoliert)}
\end{align*}
\end{minipage}
\end{exmp}
\begin{satz}\label{satz:1}
Sei $f:D\in\mathbb{R}^n\rightarrow\mathbb{R}, x_0\in\Inter D$\footnote{$\Inter\ldots $ interior (Mengenoperator)} und $\nabla f(x_0)$ vorhanden. Dann gilt: $x_0$ ist eine lokale
Extremstelle $\rightarrow \nabla f(x_0)=0$.
\end{satz}
\begin{defi}
 Eine Stelle $x_0$ mit $\nabla f(x_0)=0$ heißt kritische Stelle (stationärer Punkt) von $f$.
\end{defi}
\begin{satz}\label{satz:2}
Sei $D\subset\mathbb{R}^n$ offen, $f\in C^2(D,\mathbb{R}),x_0\in D$ und $\nabla f(x_0)=0$. Dann gilt 
\begin{align*}
H f(x_0)=\left\{\begin{tabular}{c}pos. definit\\neg. definit\\ indefinit \end{tabular} \right\}\Rightarrow f\text{ ist in }x_0\left\{\begin{tabular}{c}isol. lokale Minimumstelle\\
isol. lokale Maximumstelle\\
keine lokale Extremstelle \end{tabular} \right\}
\end{align*}
\end{satz}
\begin{uea}
Aufgabe: Bestimmen Sie von beiden Funktionen die Definitheit und die Extremstellen!
\begin{alignat*}{2}
f_1(x_1,x_2) & = \frac12\left(ax_1^2+bx_2^2\right) &\Rightarrow H f_1&=\begin{bmatrix} a & 0\\ 0 & b \end{bmatrix}\\
f_2(x_1,x_2) & = \frac12x_1^2+\frac16x_2^3 & \Rightarrow H f_2&=\begin{bmatrix} 1 & 0\\ 0 & x_2 \end{bmatrix}
\end{alignat*}
Lösung: 
\end{uea}
\begin{defi}
Sei $D\subset\mathbb{R}^n$ offen, $f:D\rightarrow\mathbb{R},h_1,\ldots,h_m:D\rightarrow\mathbb{R}$. $\overline{x}\in D$ heisst lokale oder globale Extremstelle von $f$ unter den
\ac{GNB} $h_1(x)=0,\ldots,h_m(x)=0$, wenn $\overline{x}$ lokale oder globale Extremstelle von $f$ auf
\begin{align*}
M &:=\left\{x\in D|\ h_1(x)=\ldots=h_m(x)=0\right\}
\end{align*}
ist.
\end{defi}
\begin{satz}\label{satz:3}
(Lagrange Multiplikatorregel) Seien $f,h_1,\ldots,h_m$ stetig differenzierbar, $\overline{x}\in M$ und $\nabla h_1(\overline{x}),\ldots,\nabla h_m(\overline{x})$ linear unabhängig, dann
ist $\overline{x}$ eine lokale Extremstelle von $f$ unter den \ac{GNB}. So gibt es Zahlen $\lambda_1,\ldots,\lambda_m\in\mathbb{R}$ (Lagrangesche Multiplikatoren), so dass
\begin{align*}
\nabla f(\overline{x}) & = \sum\limits_{i=1}^m\lambda_i\nabla h_i(\overline{x})
\end{align*}
gilt.\footnote{Hinweis: Dimension von $h$ ist $n$ für linear unabhängige $h_1,\ldots,h_m$. Für $h_1,\ldots,h_m$
muss gelten $m\le n$. Meist ist sogar $m<n$, da sich sonst alle $h_1,\ldots,h_m$ in genau einem Punkt schneiden würden. Eine Optimierung wäre dann sinnlos.}
\end{satz}
\textbf{Technik zum Auflisten von Extremstellen}:
\begin{enumerate}
  \item Definiere Langrange-Funktion $F(x,\lambda_1,\ldots,\lambda_m):=f(x)+\sum\limits_{i=1}^m\lambda_i\nabla h_i(x)$.
  \item Kanidaten für lokale Extremstelle unter den Lösungen von\\ $F'(x,\lambda_1,\ldots,\lambda_m)=0$ suchen:
	\begin{align*}
	\nabla F(x,\lambda_1,\ldots,\lambda_m)=\begin{bmatrix}
	\nabla f(x)+\sum\lambda\nabla h_i(x)\\
	h_1(x)\\
	\vdots\\
	h_m(x)
	\end{bmatrix} = 0\in\mathbb{R}^{n+m}.
	\end{align*}
	D.h. durch Einführung der Lagrange-Funktion kann man die Extremstellenbedingung für unbeschränkte Probleme verwenden. 
\end{enumerate}
Geometrische Interpretation im Fall $n=2$, $m=1$:
\begin{itemize}
  \item[] Der Tangentialraum $T_{\overline{x}}M=\left\{v\in\mathbb{R}^n|\ \nabla h_i^T(\overline{x})v=0\ (i=1,\ldots,m) \right\} $ in $\overline{x}$ an die Menge $M$ ist gleich dem
  Tangentialraum $T_{\overline{x}}N=\left\{ v\in\mathbb{R}^n|\ \nabla f(\overline{x})^Tv=0 \right\}$ in $\overline{x}$ an die Niveaumenge $N :=\left\{x\in D\subset\mathbb{R}^n|\
  f(x)=f(\overline{x})\right\}$
	\begin{figure}[htb]
	\centering
		\begin{tikzpicture}[auto, >=latex'] 
	
	\draw (-0.5,1) node[anchor=south] (dh1) {$\nabla h_1(\overline{x})$};
	\draw (0.5,-1) node[anchor=north] (df) {$\nabla f(\overline{x})$};
	\draw (1,0.5) node[anchor=west] (tang) {$\overline{x}+T_{\overline{x}}M=\overline{x}+T_{\overline{x}}N$};
	\draw (0,0) node[branch] (dx) {};
	\draw (0.1,0) node[anchor=west] (x) {$\overline{x}$};
	\draw (0.5,1) node[anchor=south] (M) {$M$};
	
	\draw [dashed] (-1,-0.5) -- (1,0.5);
	\draw [->] (0,0) -- (-0.5,1);
	\draw [->] (0,0) -- (0.5,-1);
	
	\draw[line width=0.6pt] (-1,0.25) .. controls (0,-0.25) .. (0.5,1);
	
	\draw[line width=0.2pt] (-1,-0.6) .. controls (0,0.1) .. (1,0.4);
	\draw[line width=0.2pt] (-1,-0.1) .. controls (0,0.6) .. (1,0.9);
	\draw[line width=0.2pt] (-1,-1) .. controls (0,-0.3) .. (1,0);
	
\end{tikzpicture}

		\caption{Auffinden von Extremstellen: Geometrische Interpretation im Fall $n=2$, $m=1$}
		\label{fig:kap_1_tangraum}
	\end{figure}
\end{itemize}

\begin{enumerate}[label=\arabic*)]
  \item $G\subset\mathbb{R}^n$: $\Inter G:=\left\{x\in\mathbb{R}^n|\ \exists\epsilon>0:U_{\epsilon}(x)\subset G \right\}$: Menge aller inneren Punkte von $G$. Wenn $G$ offen:
  $\Leftrightarrow G=\Inter G$
 	\begin{figure}[htb]
	\centering
		\begin{tikzpicture}[auto, >=latex'] 
	\draw plot[smooth cycle] coordinates{(0,0) (1,0) (2,1) (1,2)};
	
	\draw (1.5,1.5) node[anchor=north] {$G$};
	\draw (2,1) node[branch] (online) {};
	\draw (1.8,0.1) node[branch] (outline) {};
	\draw (1,1) node[branch] {};
	
	\draw (3,1) node[anchor=west] (txt) {nicht zu $\Inter G$};
	
	\draw [->] (txt) -- (online);
	\draw [->] (txt) -- (outline);
	
\end{tikzpicture}

		\caption{Visualisierung von $\Inter G$}
		\label{fig:kap_1_inter_g}
	\end{figure}
  \item $G$ offen. $C^k(G):=$ Menge aller $k$-mal stetig differenzierbaren Funktionen auf $G$. Auch: $C^k(G,\mathbb{R}^m)$ zur Angabe des Bildraums.
  \item $D\subset\mathbb{R}^n$ offen, $f\in C^2(D,\mathbb{R})\Rightarrow H f(x)=\begin{bmatrix}
  f_{11}(x) & \ldots & f_{1n}(x) \\
  \vdots 	&		 & \vdots\\
  f_{n1}(x) & \ldots & f_{nn}(x)
  \end{bmatrix}$ symmetrisch\footnote{Andere Schreibweise: $\nabla^2 f$, $\frac{\partial^2 f}{\partial x^2}$, $\nabla_{xx}f$}\\
  Sei $A$ reell und symmetrisch, d.h. $A=A^T$\\
  Dann $A$ possitiv definit, i.Z. $A>0:\Leftrightarrow x^TAx>0\forall x\neq 0\Leftrightarrow \lambda_i(A)>0\forall i$ und $A$ positiv demidefinit, i.Z. $A\ge 0:\Leftrightarrow x^TAx\ge
  0\forall x\Leftrightarrow \lambda_i(A)\ge 0\forall i$.\\
  Negativ definit bzw. semidefinit analog.\\
  $A$ indefinit: $\Leftrightarrow x^TAx$ kann sowohl positive als auch negative Werte annehmen. $\Leftrightarrow$ Es gibt positive und negative Eigenwerte von $A$.
  \item $M\subset \mathbb{R}^n$.\\
		$\bd M = \cl M\backslash\Inter M$: Rand von $M$\\
		$\cl M =\left\{x\in\mathbb{R}^n|\ \exists(x_n)_{n\in N}\subset M:x_n\rightarrow x \right\}$: Abschluss der Menge $M$ (auch: Menge der Berührungspunkte)\\
		Beispiel: \begin{tabular}[t]{lr}
			$M=[0,1)$ & $\cl M=[0,1]$\\
					  & $\Inter M = (0,1)$\\
					  & $\bd M =\{0,1\}$
		\end{tabular}
  \item \begin{tabular}[t]{rl}
  $\left\{v_1,\ldots,v_n \right\}$ linear unabhängig &:$\Leftrightarrow [\lambda_1 v_1+\ldots+\lambda_n v_n=0\Rightarrow \lambda_1=\ldots=\lambda_n=0]$\\
  $\left\{v \right\}$ linear unabhängig &:$\Leftrightarrow [\lambda v=0\Rightarrow \lambda = 0]\Leftrightarrow v\neq 0$
  \end{tabular}
\end{enumerate}
\begin{exmp}
Gesucht ist die Extremstelle von 
\begin{align*}
	f(x,y) & = 4x^2-3xy \text{ auf } K=\left\{(x,y)^T\in\mathbb{R}|\ x^2+y^2\le 1 \right\}.
\end{align*}
Nach Satz \ref{satz:1} findet man 
\begin{align*}
	\begin{bmatrix}
	0\\0
	\end{bmatrix} & = \nabla f = \begin{bmatrix}
	8x-3y\\-3x
	\end{bmatrix}\\
	\begin{bmatrix}
	8x-3y\\-3x
	\end{bmatrix}\Leftrightarrow \begin{bmatrix}
	x\\y
	\end{bmatrix} & = \begin{bmatrix}
	0\\0
	\end{bmatrix}
\end{align*}
kritische Stelle von $f$ im Inneren von $K$, jedoch
\begin{align*}
		H f(x,y) & = \begin{bmatrix}
		8 & -3\\ -3 & 0
		\end{bmatrix} \text{ mit } \det\left(H f(0,0)\right) = -9 < 0
\end{align*}
Determinante ist Produkt der Eigenwerte $\rightarrow \lambda_1>0$, $\lambda_2<0\rightarrow$ es negative Eigenwerte. $H f(0,0)$ ist indefinit, d.h. $(0;0)$ keine lokalen Extremstellen
existieren, nach Satz \ref{satz:2}.\\ 
Andererseits ist $f$ stetig auf der kompakten Menge $K$, nach Satz von Weierstrass besitzt $f$ auf dieser Menge Minimum und Maximum. Diese müssen
auf dem Rand liegen.
\begin{align*}
	\bd K  & = \left\{(x,y)^T|\ h(x,y):=x^2+y^2-1=0\right\}\\
	\Inter K &=\left\{(x,y)^T\in\mathbb{R}^2|\ x^2+y^2<1 \right\}\\
	\cl(\Inter K) & = K
\end{align*}
Da $\bd K$\footnote{$\bd K\ldots$ Rand von $K$} keine offene Menge ist, ist Satz \ref{satz:1} nicht anwendbar. Mit
\begin{align*}
M & = \bd K \text{ ist } \nabla h(x,y) = 2\begin{bmatrix}
x\\y
\end{bmatrix} \neq 0\ \forall (x,y)^T\in M 
\end{align*}
linear unabhängig. Nach Satz \ref{satz:3} existiert $\lambda\in\mathbb{R}$ mit\\
\begin{minipage}{\dimexpr.5\linewidth-1em\relax} 
  \begin{tabular}{ll}
  (1) & 	$8x-3y=\lambda\cdot 2x$ \\
  (2) & 	$-3x=\lambda\cdot 2y$ 
  \end{tabular} 
\end{minipage}% 
\begin{minipage}[0pt]{2em} 
  $\left.\mbox{\rule{0pt}{\baselineskip}}\right\}$ 
\end{minipage}% 
\begin{minipage}{\dimexpr.5\linewidth-1em\relax} 
$\Delta F(x,y) = H f(x,y)$
\end{minipage}
\begin{minipage}{\dimexpr.5\linewidth-1em\relax} 
  \begin{tabular}{ll}
  (3) & 	$x^2+y^2-1=0$ 
  \end{tabular} 
\end{minipage}% 
\begin{minipage}[0pt]{2em} 
  $\mbox{\rule{0pt}{0.35\baselineskip}}$ 
\end{minipage}% 
\begin{minipage}{\dimexpr.5\linewidth-1em\relax} 
Erfüllung der \ac{NB}en
\end{minipage}
Lösung:
\begin{alignat*}{8}
	x_{1/2} & = \pm\frac{1}{\sqrt{10}}, & \quad & y_{1/2} & = \pm\frac{3}{\sqrt{10}}, & \quad & \lambda_{1/2}=-\frac12, & \quad & F(x_{1/2},y_{1/2}) & = -\frac12\\
	x_{3/4} & = \mp\frac{3}{\sqrt{10}}, & \quad & y_{3/4} & = \pm\frac{1}{\sqrt{10}}, & \quad & \lambda_{3/4}=\frac92, & \quad & F(x_{3/4},y_{3/4}) & = \frac92
\end{alignat*}
Technik:
\begin{align*}
	F(x,y,\lambda) & = f(x,y) +\lambda h(x,y) \Rightarrow \begin{bmatrix}
	F_x\\ F_y \\ F_{\lambda}
	\end{bmatrix} = 0
\end{align*}
Abbildung \ref{fig:kap_1_beispiel_extremstelle} stellt die Niveaulinien der Ausgangsfunktion und der Berechnungen dar. 
\begin{figure}[htb]
	\centering
	\begin{tikzpicture}[auto, scale=2.0, >=latex']
	\draw[->] (-2,0) -- (2,0) node[right] {$x$};
	\draw[->] (0,-2) -- (0,2) node[above] {$y$};
	
   	\draw (-2,0) node[anchor=north] {-2};
   	\draw (-1,0) node[anchor=north] {1};
   	\draw (0,0) node[anchor=north] {0};
   	\draw (1,0) node[anchor=north] {1};
   	\draw (2,0) node[anchor=north] {2};

   	\draw (0,-2) node[anchor=east] {-2};
   	\draw (0,-1) node[anchor=east] {1};
   	\draw (0,1) node[anchor=east] {1};
   	\draw (0,2) node[anchor=east] {2};
   	
   	\foreach \x in {-2,-1.5,...,1.5}
     	\draw (\x,0.02) -- (\x,-0.02);
   	
   	\foreach \y in {-2,-1.5,...,1.5}
     	\draw (0.02,\y) -- (-0.02,\y);

	\draw[thick,color=black] plot[samples=200, domain=-1.5:1.5] (\x,{4/3*\x});

	\draw[thick,color=yellow] plot[samples=100, domain=-1.7:-0.22] (\x,{4/3*\x-1/(2*\x)});
	\draw[thick,color=yellow] plot[samples=100, domain=0.22:1.7] (\x,{4/3*\x-1/(2*\x)});
	
	\draw[thick,color=green] plot[samples=100, domain=-1.4:-0.09] (\x,{4/3*\x+1/(6*\x)});
	\draw[thick,color=green] plot[samples=100, domain=0.09:1.4] (\x,{4/3*\x+1/(6*\x)});
	
	\draw[thick,color=red] plot[samples=100, domain=-2:-0.55] (\x,{4/3*\x-9/(6*\x)});
	\draw[thick,color=red] plot[samples=100, domain=0.55:2] (\x,{4/3*\x-9/(6*\x)});
	
	\draw[thick,color=blue] (0,0) circle (1);
\end{tikzpicture}

	\caption{Niveaulinien der Funktion $f$: $f=-\frac12$ (Grün), $f=0$ (Schwarz), $f=1$ (Gelb), $f=\frac92$ (Rot), $x^2+y^2=1$ (Blau)}
	\label{fig:kap_1_beispiel_extremstelle}
\end{figure}
\end{exmp}

\begin{defi}
Sei $D\in\mathbb{R}^n$ offen, $f:D\rightarrow\mathbb{R},g_1,\ldots,g_m:D\rightarrow\mathbb{R},\overline{x}\in D$ heisst (lokale bzw. globale) Extremstelle von $f$ unter der \ac{UNB}
$g_1(x)\le 0,\ldots,g_m(x)\le 0$, wenn $\overline{x}$ (lokale oder globale) Extremstelle von $f$ auf $G:=\left\{x\in D|\ g_1(x)\le 0,\ldots,g_m(x)\le 0\right\}$.
\end{defi}
\begin{defi}
$I_0(x):=\left\{ i\in\left\{ 1,\ldots,m \right\}|\ g_i(x)=0 \right\}$: Indexmenge der für $x\in G$ aktiven Restriktionen (für $x\in\int G$ gilt $I_0(x)\in\emptyset$)
\end{defi}
$L(x,u)=f(x)+\sum\limits_{i=1}^{m}u_ig_i(x)$: Lagrange-Funktion zur Aufgabe $\min\limits_{x\in G}f(x)$
\begin{satz}\label{satz:4}
	(Kuhn-Tucker Bedinungen 1. Ordnung):\\
	Sei $f,g_1,\ldots,g_m$ stetig differenzierbar, $\overline{x}\in G$ und $\nabla g_i(x)$ mit $i\in I_0(\overline{x})$ linear unabhängig. Ist $\overline{x}$ lokale Minimumstelle von $f$
	unter \ac{UNB}, so gibt es $a_1,\ldots,a_m\in\mathbb{R}$ (Lagrange-Multiplikatoren) so, dass gilt
	\begin{align*}
	\nabla_x L(\overline{x},u) & = \nabla f(\overline{x})+\sum\limits_{i=1}^m u_i\nabla g_i(\overline{x})= 0\quad\text{mit } \begin{bmatrix}
	g_i(\overline{x})\le 0\\ u_i\ge 0\\ u_ig_i(\overline{x})=0
	\end{bmatrix},\ i=1,\ldots,m
	\end{align*}
\end{satz}
\textbf{Geometrische Interpretation der Kuhn-Tucker-Bedingungen}\\

\begin{figure}[!htb]
	\centering
	\begin{tikzpicture}[auto, >=latex'] 
	\draw (0,-0.25) .. controls (1,-0.5) .. (3,0.5);
	\draw (-0.5,-2.25) .. controls (1,-2.5) and (2.5,-2) .. (4,-0.5);
	\draw (2,0.25) .. controls (3,0) and (3.5,-0.25) .. (3.75,-1.25);
	
	\draw (0,-0.25) node[anchor=east] {$g_3=0$};
	\draw (-0.5,-2.25) node[anchor=east] {$g_1=0$};
	\draw (2,0.25) node[anchor=east] {$g_2=0$};
	\draw (-0.25,-1.25) node[anchor=east] {$G$};
	
	\draw (1,-1.3) circle (0.4);
	\draw (1,-1.3) circle (0.8);
	\draw (1,-1.3) circle (1.2);
	\draw (1,-1.3) circle (0.05);
	\draw (1,-1.3) node[anchor=west] {$\overline{x}$};

	\draw (0.8,-3) node[anchor=west] {$f= const.$};
	
\end{tikzpicture}

	\caption{Optimalstelle mit $I_0(\overline{x})=\emptyset$}
	\label{fig:kap_1_ktb_opt_0}
\end{figure}

\begin{figure}[!htb]
	\centering
	\begin{tikzpicture}[auto, >=latex'] 
	\draw (0,-0.25) .. controls (1,-0.5) .. (3,0.5);
	\draw (-0.5,-2.25) .. controls (1,-2.5) and (2.5,-2) .. (4,-0.5);
	\draw (2,0.25) .. controls (3,0) and (3.5,-0.25) .. (3.75,-1.25);
	
	\draw (0,-0.25) node[anchor=east] {$g_3=0$};
	\draw (-0.5,-2.25) node[anchor=east] {$g_1=0$};
	\draw (2,0.25) node[anchor=east] {$g_2=0$};
	\draw (-0.25,-1.25) node[anchor=east] {$G$};

	\draw[->] (0.9,-2.25) -- (1.1,-3.45);
	\draw[->] (0.9,-2.25) -- (1,-2.85);
	
	\draw (1.1,-3.45) node[anchor=north] {$\nabla g_1(\overline{x})$};
	\draw (1,-2.85) node[anchor=west] {$-\nabla f(\overline{x})$};

	\draw (1.1,-3.45) circle (0.4);
	\draw (1.1,-3.45) circle (0.8);
	\draw (1.1,-3.45) circle (1.2);
	\draw (0.9,-2.25) circle (0.05);
	\draw (0.9,-2.25) node[anchor=south] {$\overline{x}$};
	
	\draw (2.2,-3.55) node[anchor=west] {$f= const.$};
	
\end{tikzpicture}

	\caption{Optimalstelle mit $I_0(\overline{x})=\{1\}$}
	\label{fig:kap_1_ktb_opt_1}
\end{figure}

\begin{figure}[!htb]
	\centering
	\begin{tikzpicture}[auto, >=latex'] 
	\draw (0,-0.25) .. controls (1,-0.5) .. (3,0.5);
	\draw (-0.5,-2.25) .. controls (1,-2.5) and (2.5,-2) .. (4,-0.5);
	\draw (2,0.25) .. controls (3,0) and (3.5,-0.25) .. (3.75,-1.25);
	
	\draw (0,-0.25) node[anchor=east] {$g_3=0$};
	\draw (-0.5,-2.25) node[anchor=east] {$g_1=0$};
	\draw (2,0.25) node[anchor=east] {$g_2=0$};
	\draw (-0.25,-1.25) node[anchor=east] {$G$};

	\draw[->] (3.625,-0.85) -- (4.77,-1.2);
	\draw[->] (3.625,-0.85) -- (4.5, -1.7);
	\draw[->] (3.625,-0.85) -- (4.6, -0.6);
	
	\draw (4.77,-1.2) node[anchor=west] {$-\nabla f(\overline{x})$};
	\draw (4.5, -1.7) node[anchor=west] {$\nabla g_1(\overline{x})$};
	\draw (4.6, -0.6) node[anchor=west] {$\nabla g_2(\overline{x})$};

	\draw (4.77,-1.2) circle (0.4);
	\draw (4.77,-1.2) circle (0.8);
	\draw (4.77,-1.2) circle (1.2);
	\draw (3.625,-0.85) circle (0.05);
	\draw (3.625,-0.85) node[anchor=east] {$\overline{x}$};

	\draw (4.77,0) node[anchor=south] {$f= const.$};
	
\end{tikzpicture}

	\caption{Optimalstelle mit $I_0(\overline{x})=\{1,2\}$}
	\label{fig:kap_1_ktb_opt_1_2}
\end{figure}
\begin{enumerate}[label=\alph*)]
  \item zu \figureref{fig:kap_1_ktb_opt_0}: $u_ig_i(\overline{x})=0\Rightarrow u_1=u_2=u_3=0\Rightarrow \nabla f(\overline{x})=0$
  \item zu \figureref{fig:kap_1_ktb_opt_1}: $u_ig_i(\overline{x})=0\Rightarrow u_2=u_3=0\Rightarrow \nabla f(\overline{x})+u_1\nabla g_1(\overline{x})=0$
  \item zu \figureref{fig:kap_1_ktb_opt_1_2}: $u_ig_i(\overline{x})=0\Rightarrow u_3=0\Rightarrow \nabla f(\overline{x})+u_1\nabla g_1(\overline{x})+u_2\nabla g_2(\overline{x})=0$ mit
  $u_1$, $u_2\ge 0$\\
\end{enumerate}

Bemerkung:
\begin{itemize}
  \item[] $x$ ist Maximumstelle von $f$ genau dann, wenn $x$ Minimumstelle von $-f$ ist.  
\end{itemize} 
\begin{gegenexmp}\hspace{1cm}
\begin{minipage}{0.5\textwidth}
\begin{align*}
f(x) & = (x_1+1)^2+x^2_2\\
g_1(x) & = x_2-x^3_1 \le 0\\
g_2(x) & = -x_2 \le 0
\end{align*}
Minimumstelle: $(0,0)$
\end{minipage}
\begin{minipage}{0.5\textwidth}
\centering
\begin{tikzpicture}[auto, scale=1.5, >=latex']

	\fill[pattern=north east lines wide, pattern color=red] (0,0) rectangle (2, -0.2);

	\draw[->] (-2,0) -- (2,0) node[right] {$x_1$};;
	\draw[->] (0,-2) -- (0,2) node[above] {$x_2$};;
	
   	\draw (-2,0) node[anchor=north] {-2};
   	\draw (-1,0) node[anchor=north] {1};
   	\draw (0,0) node[anchor=north] {0};
   	\draw (1,0) node[anchor=north] {1};
   	\draw (2,0) node[anchor=north] {2};

   	\draw (0,-1) node[anchor=east] {1};
   	\draw (0,1) node[anchor=east] {1};
   	\draw (0,2) node[anchor=east] {2};
   	
   	\foreach \x in {-2,-1.5,...,1.5}
     	\draw (\x,0.02) -- (\x,-0.02);
   	
   	\foreach \y in {-2,-1.5,...,1.5}
     	\draw (0.02,\y) -- (-0.02,\y);

	\draw[thick,color=green] plot[samples=200, domain=-1.25:1.25] (\x,{\x*\x*\x});
	\draw[thick,color=green,pattern=north west lines wide, pattern color=green] plot[samples=200, domain=0:1.25] (\x,{\x*\x*\x});

	\draw[thick,color=blue] (-1,0) circle (0.5);
	\draw[thick,color=blue] (-1,0) circle (1);

\end{tikzpicture}

\end{minipage}
\end{gegenexmp}
Bemerkung:
\begin{itemize}
  \item[] Die Forderung nach der Unabhängigkeit der Vektoren in $\nabla h_1(\overline{x}),\ldots,\nabla h_m(\overline{x})$ in Satz \ref{satz:3} bzw. $\nabla g_i(\overline{x})$ mit
  $i\in I_0(\overline{x})$ in Satz \ref{satz:4} ist eine sogenannte \textit{Regularitätsbedingung} (\ac{LICQ}). Es gibt andere Regularitätsbedingungen, z.B. die Forderung das
  $h_1,\ldots,h_m$ bzw. $g_1,\ldots,g_m$ affine lineare Funktionen sind, d.h. eine Darstellung der Form 
  \begin{align*}
  	M = \left\{ x\in\mathbb{R}^n:\ Ax=b \right\} & \text{ bzw. } G=\left\{ x\in\mathbb{R}^n:\ Ax\le b \right\}
  \end{align*}
  mit der Matrix $A\in\mathbb{R}^{m\times n}$ und dem Vektor $b\in\mathbb{R}^m$ möglich ist.
\end{itemize}
Für die Aufgabe $\min f(x)$ bei $g_i(x)\le 0$ ($i=1,\ldots,m$), $h_j(x)=0$ ($j=1,\ldots,p$) mit stetig differenzierbaren $f$, $g_i$, $h_j$ lautet die \ac{KTB}:
\begin{align}
	\nabla_x L(x,u,\lambda) & = \nabla f(x)+\sum\limits_{i=1}^m u_ig_i(x)+\sum\limits_{j=1}^p\lambda_j h_j(x) \label{eq:kap_1_ktp}\\
	& h(x) = 0,\ u\ge 0,\ g(x)\le 0,\ u_ig_i(x)=0\ (i=1,\ldots,m) \notag
\end{align}
Ein Punkt $L(x,u,\lambda)$ der Gleichung \eqnref{eq:kap_1_ktp} erfüllt, heisst \ac{KTP}.

\begin{exmp}
Gesucht ist die optimale Lösung der Aufgabe $f(x)=-x^2_1+x_2^2\rightarrow\min$ bei $(x_1+1)(x_2+2)\le 4$ mit $x_1\ge 0$, $x_2\ge 0$.\\
Mit 
\begin{align*}
	g_1(x) & = (x_1+1)(x_2+2)-4\le 0,\\
	g_2(x) & = -x_1\le 0,\\
	g_3(x) & = -x_2\le 0 
\end{align*}
erhält man
\begin{align*}
	L(x,u) & = f(x)+\sum\limits_{i=1}^3u_ig_i(x)
\end{align*}
und
\begin{align*}
\nabla_x L(x,u) & = \begin{bmatrix}
-2x_1+u_1(x_2+2)-u_2\\
2x_2+u_1(x_1+1)-u_3
\end{bmatrix}.
\end{align*}
Die $5$ nichtlinearen Gleichungen $\nabla_x L(x,u)=0$, $u_ig_i(x)=0$ mit $i=1,2,3$ haben die reelen Lösungen:
\begin{itemize}
	\item $u_1=0$, $u_2=0$, $u_3=0$, $x_1=0$, $x_2=0$
	\item $u_1=-4$, $u_2=16$, $u_3=0$, $x_1=1$, $x_2=2$ erfällt wegen Nichterfüllen der Ungleichung 
	\item $u_1=1$, $u_2=0$, $u_3=2$, $x_1=1$, $x_2=0$
\end{itemize}
Wegen $f(0,0)=0$, $f(1,0)=-1$ ist $x=\begin{pmatrix} 1 & 0 \end{pmatrix}^T$ die gesuchte Minimumstelle.
\begin{figure}[!htb]
	\centering
	\begin{tikzpicture}[auto, scale=2.0, >=latex']

	\fill[pattern=north east lines wide, pattern color=red] (0,0) rectangle (2, -0.2);
	\fill[pattern=north east lines wide, pattern color=orange] (0,0) rectangle (-0.2, 2.5);

	\draw[->] (-2,0) -- (2,0) node[right] {$x_1$};;
	\draw[->] (0,-1) -- (0,2.5) node[above] {$x_2$};;
	
   	\draw (-2,0) node[anchor=north] {-2};
   	\draw (-1,0) node[anchor=north] {1};
   	\draw (0,0) node[anchor=north] {0};
   	\draw (1,0) node[anchor=north] {1};
   	\draw (2,0) node[anchor=north] {2};

   	\draw (0,-1) node[anchor=east] {1};
   	\draw (0,1) node[anchor=east] {1};
   	\draw (0,2) node[anchor=east] {2};
   	
   	\foreach \x in {-2,-1.5,...,1.5}
     	\draw (\x,0.02) -- (\x,-0.02);
   	
   	\foreach \y in {-1,-0.5,...,2}
     	\draw (0.02,\y) -- (-0.02,\y);

	\draw[thick,color=black] plot[samples=200, domain=-1:2] (\x,{\x});

	\draw[thick,color=yellow] plot[samples=200, domain=1:2] (\x,{sqrt(\x*\x-1)});
	\draw[thick,color=yellow] plot[samples=200, domain=1:1.4] (\x,{-sqrt(\x*\x-1)});
	
	\draw[thick,color=green] plot[samples=200, domain=-1.75:1.75] (\x,{sqrt(\x*\x+1)});
	
	\draw[thick,color=blue,pattern=north east lines wide, pattern color=blue] plot[samples=200, domain=0:1] (\x,{4/(\x+1)-2});


\end{tikzpicture}

	\caption{Beispiel mit \ac{KTB}: $f=-1$ (Gelb), $f=0$ (Schwarz), $f=1$ (Grün), $g_1$ (Blau), $g_2$ (Rot), $g_3$ (Orange)}
	\label{fig:kap_1_ktp}
\end{figure}
\end{exmp}

\textit{Konvexe Optimierungsprobleme}:\\
$M\subset\mathbb{R}^n$ konvex $\Leftrightarrow \left[x_1,x_2\in M\Rightarrow \lambda x_1+(1-\lambda)x_2\in M\forall\lambda\in[0,1]\right]$
\begin{figure}[!htb]
	\centering
	\subfloat[konvex]{\begin{tikzpicture}[auto, >=latex'] 
	\draw plot[smooth cycle] coordinates{(0,0) (1,0) (2.5,1) (1,2.5) (0,1)};
	
	\draw (0.8,2) node[anchor=north] {$M$};
	\draw (0.5,0.5) node[anchor=north] {$x_1$};
	\draw (1.5,1.5) node[anchor=south] {$x_2$};
	\draw (0.5,0.5) node[branch] (x1) {};
	\draw (1.5,1.5) node[branch] (x2) {};
	
	\draw [->] (x2) -- node {$\lambda$} (x1);
	
\end{tikzpicture}
}\qquad
	\subfloat[nicht konvex]{\begin{tikzpicture}[auto, >=latex'] 
	\draw plot[smooth cycle] coordinates{(0,0) (1,0) (2,1) (3,0) (4,0) (4,1) (3,2) (1,2) (0,1)};
	
	\draw (2,2) node[anchor=north] {$M$};
	\draw (3.5,0.8) node[anchor=south] {$x_1$};
	\draw (0.5,0.2) node[anchor=south] {$x_2$};
	\draw (3.5,0.8) node[branch] (x1) {};
	\draw (0.5,0.2) node[branch] (x2) {};
	
	\draw [dashed] (x2) -- (x1);
	
\end{tikzpicture}
}
	\caption{Konvexität}
	\label{fig:kap_1_konvex}
\end{figure}

Sei $M\subset\mathbb{R}^n$ konvex: $f:M\rightarrow\mathbb{R}$ konvex auf $M$, wenn 
\begin{align*}
	\left[ x_1,x_2\in M \Rightarrow M(\lambda x_1+(1-\lambda)x_2)\right. & \left.\le \lambda f(x_1)+(1-\lambda)f(x_2)\forall\lambda\in[0,1]\right]
\end{align*}
\begin{figure}[!htb]
	\centering
	\subfloat[$f$ konvex]{\begin{tikzpicture}[auto, >=latex'] 

	\draw[->] (0,0) -- (5,0) node[right] {$x$};
	\draw[->] (0,-0.02) -- (0,4) node[above] {$f(x)$};
	
	\draw[dotted] (4.1,0) -- node[anchor=west] {$f(x_1+(1-\lambda)x_2)$} (4.1,4); 
	\draw (0, 0.5) node[anchor=east] {$f(x_2)$};
	\draw (4.1, 4) node[anchor=west] {$f(x_1)\forall\lambda\in(0,1)$};
	
	\draw[black] plot[samples=200, domain=0:4] (\x,{0.5*exp(0.5*\x)});
		
	\draw (4.1, 0.02) -- (4.1, -0.02);
	\draw (0, -0.02) node[anchor=north] {$0$};
	\draw (4.1, -0.02) node[anchor=north] {$1$};

\end{tikzpicture}
}\qquad
	\subfloat[$f$ nicht konvex]{\begin{tikzpicture}[auto, >=latex'] 

	\draw[->] (0,0) -- (4,0) node[right] {$x$};
	\draw[->] (0,0) -- (0,4) node[above] {$f(x)$};
	
	\draw[black] plot[samples=200, domain=0:4] (\x,{2.0+sin(6*\x r-pi)});
			
\end{tikzpicture}
}
	\caption{Konvexe Funktionen}
	\label{fig:kap_1_konvex_fkt}
\end{figure}

\textit{streng konvex auf} $M$:\\
Sei $M\subset\mathbb{R}^n$ offen, konvex, $f\in C^2(M,\mathbb{R})$:
\begin{itemize}
  \item $f$ konvex auf $M\Leftrightarrow\forall x\in M:H f(x)\ge 0$
  \item $f$ streng konvex auf $M\Leftrightarrow\forall x\in M: H f(x)>0$
\end{itemize}
Falls $M$ konvex und $f$ konvex auf $M$, so heisst $\min\limits_{x\in M}f(x)$ konvexe Optimierungsaufgabe und es gilt:
\begin{itemize}
  \item Falls $x^{\ast}\in M$ eine lokale Minimumstelle ist, so ist $x^{\ast}$ auch globale Minimumstelle auf $M$.
  \item Ist $f$ streng konvex, so gibt es höchstens ein globales Minimum von $f$ über $M$.
\end{itemize}
\begin{exmp}
Die streng konvexe Funktion $f(x)=e^x$ soll minimiert werden $\min\limits_{x\in\mathbb{R}^1}e^x$. Es soll gezeigt werden, dass höchstens ein globales Minimum von $f$ über $M$ existiert.
Dabei ist zu beachten:
\begin{align*}
\left.
 \begin{tabular}{rl}
 $f:\mathbb{R}^n$ & $\rightarrow \mathbb{R}^1$\\
 $g_i:\mathbb{R}^n$ & $\rightarrow \mathbb{R}^1$\\
 $h_j:\mathbb{R}^n$ & $\rightarrow \mathbb{R}^1$
 \end{tabular}\right\} & \text{stetig differenzierbar mit }i=1,\ldots,n\text{ und } j=1,\ldots,p
\end{align*}
Damit lässt sich die lokale Minimumstelle
\begin{align*}
x^{\ast}\in G & := \left\{ x\in\mathbb{R}^n:g_i(x)=0\ (i=1,\ldots,n),\ h_j(x)=0\ (j=1,\ldots,p) \right\}\\
\left\{ \nabla(x^{\ast})\ (i\in I_0(x^{\ast}) \right. & \left. := \left\{ i\in\left\{1,\ldots,m\right\}|\ g_i(x^{\ast})=0 \right\}),\nabla h_j(x^{\ast})\ (j=1,\ldots,p) \right\} 
\end{align*}
beschreiben und für die lokale Minimumstelle $x^{\ast}$ von $f$ auf $G$  gilt $\Rightarrow\exists u^{\ast}\in\mathbb{R}^m$, $\exists\lambda^{\ast}\in\mathbb{R}^p$ mit der \ac{KTB}
\begin{align*}
	\nabla_x L(x^{\ast},u^{\ast}) & = \nabla f(x^{\ast})+\sum\limits_{i=1}^n u_i^{\ast}\nabla g_i(x^{\ast})+\sum\limits_{j=1}^p\lambda_j^{\ast}\nabla h_j(x^{\ast})=0
\end{align*}
mit $g(x^{\ast})\le 0$, $u^{\ast}\ge 0$, $u^{\ast T}g(x^{\ast})=0$, $h(x^{\ast})=0$ und dem \ac{KTP}: $\begin{pmatrix}x^{\ast}, & u^{\ast}, & \lambda^{\ast} \end{pmatrix}$
\end{exmp}
 
\begin{gegenexmp}\hspace{1cm}
\begin{minipage}{0.5\textwidth}
\begin{align*}
f(x) & = (x_1+1)^2+x^2_2\\
g_1(x) & = x_2-x^3_1 \le 0\\
g_2(x) & = -x_2 \le 0
\end{align*}
Minimumstelle: $x^{\ast}=(0,0)$, aber \ac{KTB} nicht erfüllt\\
$\Rightarrow \nabla g_1(x^{\ast})$, $\nabla g_2{x^{\ast}}$ linear abhängig

\end{minipage}
\begin{minipage}{0.5\textwidth}
	\centering
	\begin{tikzpicture}[auto, scale=1.5, >=latex']

	\draw[->] (-2,0) -- (2,0) node[right] {$x_1$};;
	\draw[->] (0,-2) -- (0,2) node[above] {$x_2$};;
	
   	\draw (-2,0) node[anchor=north] {-2};
   	\draw (-1,0) node[anchor=north] {1};
   	\draw (0,0) node[anchor=north] {0};
   	\draw (1,0) node[anchor=north] {1};
   	\draw (2,0) node[anchor=north] {2};

   	\draw (0,-1) node[anchor=east] {1};
   	\draw (0,1) node[anchor=east] {1};
   	\draw (0,2) node[anchor=east] {2};

	\draw (0,0) node[anchor=east] {$x^{\ast}$};
   	
   	\foreach \x in {-2,-1.5,...,1.5}
     	\draw (\x,0.02) -- (\x,-0.02);
   	
   	\foreach \y in {-2,-1.5,...,1.5}
     	\draw (0.02,\y) -- (-0.02,\y);

	\draw[thick,color=green] plot[samples=200, domain=-1.25:1.25] (\x,{\x*\x*\x});

	\draw[thick,color=blue] (-1,0) circle (0.5);
	\draw[thick,color=blue] (-1,0) circle (1);

\end{tikzpicture}

\end{minipage}
\end{gegenexmp}

\begin{satz}\label{satz:5}
Für die Aufgabe $\min\limits_{x\in G}f(x)$ mit $G:=\left\{ x\in\mathbb{R}^n|\ g_i(x)\le 0\ (i=1,\ldots,m) \right\}$ und $f$, $g_i$ ($i=1,\ldots,m$) stetig differenzierbar und konvex
gilt:
\begin{itemize}
  \item $x^{\ast}$ ist Lösung und $\exists\tilde{x}\in\mathbb{R}^n:g_i(\tilde{x})<0$ ($i=1,\ldots,m$) (Slater Bedingung)\\
  $Rightarrow \exists u^{\ast}\in\mathbb{R}^m:(x^{\ast},u^{\ast})$ ist \ac{KTB}
  \item $(x^{\ast},u^{\ast})$ sind \ac{KTP} $\Rightarrow x^{\ast}$ ist Lösung für alle
  \item Falls alle $g_i$ linear, d.h. $G=\left\{x\in\mathbb{R}^n:Ax\le b \right\}$ mit $A\in\mathbb{R}^{m\times n}$, $b\in\mathbb{R}^m$, so gilt
  \begin{align*}
  	x^{\ast}\text{ ist Lösung } &\Leftrightarrow \exists u^{\ast}\in\mathbb{R}^m:(x^{\ast},u^{\ast})\text{ ist \ac{KTP}}
  \end{align*}
  Bei zusätzlichen linearen \ac{GNB}, d.h. $G=\left\{x\in\mathbb{R}^n:Ax\le b,\ Cx=d \right\}$ mit $C\in\mathbb{R}^{p\times n}$, $d\in\mathbb{R}^p$:
  \begin{align*}
  	x^{\ast}\text{ ist Lösung } &\Leftrightarrow \exists u^{\ast}\in\mathbb{R}^m,\lambda^{\ast}\in\mathbb{R}^m:(x^{\ast},u^{\ast},\lambda^{\ast})\text{ ist \ac{KTP}}
  \end{align*}
\end{itemize}
\end{satz}

\begin{uea}
Aufgabe: 
\begin{itemize}
  \item[] Finden Sie alle Lösungen von $f(x)=-x_1-x_2\rightarrow\min\limits_{x\in\mathbb{R}^2}$ bei $g_1(x)=x_1+x_2^2\le 0$ (Skizze)!
\end{itemize}
Lösung: 
\end{uea}

\section{Anwendungen auf quadratische Optimierungsaufgaben}
\textbf{Aufgabe}
\begin{itemize}
	\item[] $\min\limits_{x\in\mathbb{R}^n} \frac12 x^TQx+q^Tx$ bei $Ax\le b$, $Q>0$
	\item[] mit $Q\in\mathbb{R}^{n\times n}$, $q\in\mathbb{R}^n$, $A\in\mathbb{R}^{m\times n}$, $b\in\mathbb{R}^m$ (convex) 
\end{itemize}
\subsection{Vorbereitungen}
\begin{enumerate}[label=(\arabic*)]
  \item $\min\limits_{x\in\mathbb{R}^n}f(x)$ mit $f(x)=\frac12 x^TQx+q^Tx$ und wegen $H f(x)=Q\ge 0$ ist $f$ convex auf $\mathbb{R}^n$. Damit gilt $x^{\ast}$ ist Lösung
  $\Leftrightarrow x^{\ast}$ erfüllt \ac{KTB} $\Leftrightarrow\nabla f(x^{\ast})=0$. Somit erhällt man die Lösung von (1-TODO) durch Lösen des linearen Gleichungssystems
  \begin{align*}
  	\nabla f(x) & = Qx+q =0
  \end{align*} 
  \underline{Beachte}: Lösung muss nicht existieren.
  \item $\min\limits_{x\in\mathbb{R}^n}\frac12 x^TQx+q^Tx$ bei $Cx=d$ mit $C\in\mathbb{R}^{p\times n}$, $d\in\mathbb{R}^p$. Analog zu (1): $x^{\ast}$ ist Lösung
  $\xLeftrightarrow{\text{Satz \ref{satz:5}}}\exists\lambda^{\ast}\in\mathbb{R}^p:(x^{\ast},\lambda^{\ast})$ ist \ac{KTB}. Mit $L(x,\lambda)=\frac12
  x^TQx+q^Tx+\lambda^T(Cx-d)$ liefert
  \begin{align*}
  \nabla L(x,\lambda) & = \begin{bmatrix}
  Qx+q+C^T\lambda \\
  Cx-d
  \end{bmatrix} = 0
  \end{align*} 
  unter den \ac{KTB} $\nabla_xL(x,\lambda)=0$ und $h(x)=0$ liefert das lineare Gleichungssystem
  \begin{align*}
  \begin{bmatrix}
  Q	& c^T\\ C	& 0 
  \end{bmatrix}\begin{bmatrix}
  x\\ \lambda
  \end{bmatrix} & = \begin{bmatrix}
  -q \\ d
  \end{bmatrix}
  \end{align*}
  \underline{Beachte}: Kann keine oder mehrdeutige Lösungen besitzen.
  \item Projektion auf Untervektorraum\\
  		$A\in\mathbb{R}^{m\times n}$, $m\ge n$, $\Rang A=n$, $b\in\mathbb{R}^m$\\
  		\begin{figure}[!htb]
			\centering
			\begin{tikzpicture}[scale=2.0]

	\draw[->] (-0.5,0) -- (3,0);
	\draw[->] (0,-0.5) -- (0,2);
	
	\draw[-] (-1,2) -- (0.25, -0.5);
	\draw[-] (-0.5,-0.25) -- (3, 1.5);
	
	\draw[->,color=blue] (2,2) -- node[anchor=east] {$I-P$} (-0.4,0.8);
	\draw[->,color=blue] (2,2) -- node[anchor=west] {$P$} (2.4, 1.2);
	
	\draw (2,2) node[branch] {};
	\draw (2,2) node[anchor=south] {$b$};
	\draw (2.4,1.2) node[anchor=west] {$Pb=A\widehat{x}$};
	\draw (-0.4,0.8) node[anchor=east] {$(I-P)b$};
	
	\rechterWinkel{0,0}{27}{.25}
	\rechterWinkel{2.4,1.2}{27}{.25}
		
\end{tikzpicture}

			\caption{Projektion}
			\label{fig:kap_1_projektion}
		\end{figure}
  		\begin{align}
  			\im A & := \left\{Ax:x\in\mathbb{R}^n \right\} \notag\\
  			(\im A)^{\bot} =\ker A^T & := \left\{y\in\mathbb{R}^m:A^Ty=0 \right\}\notag\\
  			A^T(b-A\widehat{x}) & = 0\notag\\
  			A^Tb & = A^TA\widehat{x}\notag\\
  			\widehat{x} & = (A^TA)^{-1}A^Tb\notag\\ 
  			\rightarrow P & = A(A^TA)^{-1}A^T\quad \ldots\text{ Projektor auf $\im A$} \label{eq:kap_1_projektion}
  		\end{align}
  		Projektion eines Vektors $v\in\mathbb{R}^n$ auf $M:=\left\{x\in\mathbb{R}^n|\ Cx=d \right\}$ mit $C\in\mathbb{R}^{p\times n}$, $p\le n$, $\Rang c = p$, siehe
  		\figureref{fig:kap_1_projektion} und man erhält den Projektor durch ersetzen von $A^T$ durch $C$ in \eqnref{eq:kap_1_projektion}
  		\begin{align*}
  			\mathcal{P} & = C^T(CC^T)^{-1}C\quad \ldots\text{ Projektor auf $M$}. 
  		\end{align*}
  		\begin{figure}[!htb]
			\centering
			\begin{tikzpicture}[auto, scale=2.0, >=latex']

	\draw[->] (-0.5,0) -- (3,0);
	\draw[->] (0,-0.5) -- (0,3);
	
	\draw[thick,color=black] plot[samples=200, domain=-0.5:3] (\x,{0.5*\x+1});
	\draw[thick,color=black] plot[samples=200, domain=-0.5:3] (\x,{0.5*\x});
	
	\draw[->,color=blue] (1,1.5) -- node[anchor=north] {$\mathcal{P}_v$} (1.5,1.75);
	\draw[dotted] (1.5,1.75) -- (1.25,2.25);
	\draw[->,color=blue] (1,1.5) -- node[anchor=east] {$v$} (1.25,2.25);
	
	\rechterWinkel{1.5,1.75}{27}{.25}
	
	\draw[<-] (1.25,0.625) -- node[anchor=east] {$\mathcal{P}$} (1,1.125);
	\draw[->, dotted] (0,0) -- (1,1.125);

	\rechterWinkel{1.25,0.625}{27}{.25}
	
	\draw (1.25,0.625) node[anchor=north] {$\mathcal{P}_v$};
	\draw (1,1.125) node[anchor=south] {$v$};
	\draw (2.5,1.25) node[anchor=west] {$\{x\in\mathbb{R}^n: Cx=0\}$};
	
\end{tikzpicture}

			\caption{Projektion}
			\label{fig:kap_1_projektion}
		\end{figure}
\end{enumerate}
\begin{uea}
Aufgabe:
\begin{itemize}
  \item[] Zeigen Sie, dass für $\min\limits_{x\in\mathbb{R}}$, d.h. $Q=0$, $q=1$, keine Lösung exisiert. 
\end{itemize}
Lösung:
\end{uea}
\begin{uea}
Aufgabe:
\begin{itemize}
  \item[] Wiederholen Sie den Algorithmus aus dem Beispiel mit $x^0=\begin{bmatrix}1\\ -1 \end{bmatrix}$. 
\end{itemize}
Lösung:
\end{uea}

\begin{uea}
Aufgabe:
\begin{itemize}
  \item[] Seien $A=\begin{bmatrix}
2\\1
\end{bmatrix}$, $b=\begin{bmatrix}
1\\2
\end{bmatrix}$. Berechnen Sie $Pb$, $(I-P)b$, $(I-P)P$, $Pb+(I-P)b$, $(Pb)^T(I-P)b$.
	\item[] Was ist das kleinste $\alpha\in\mathbb{R}$, so dass $\|Pb\|_2\le\alpha\|b\|_2$ erfüllt ist? 
\end{itemize}
Lösung:
\end{uea}


\begin{uea}
Aufgabe: 
\begin{itemize}
  \item[] Für $C=\begin{bmatrix}1 & 1 \end{bmatrix}$, $d=1$ projiziere $v_1=\begin{bmatrix}1 \\ 1 \end{bmatrix}$, $v_2=\begin{bmatrix}1 \\ 0 \end{bmatrix}$ auf $M$ (Skizze)!
  \item[] Für $C=\begin{bmatrix}1 & 1\\ 0 & 1 \end{bmatrix}$, $d=\begin{bmatrix} 1\\1\end{bmatrix}$ projiziere $v_1=\begin{bmatrix}1 \\ 1 \end{bmatrix}$, $v_2=\begin{bmatrix}1 \\ 0
  \end{bmatrix}$ auf $M$ (Skizze)!
\end{itemize}
Lösung: 
\end{uea}


\subsection{Aktive-Mengen-Strategie}
\textbf{Aufgabe}:
\begin{itemize}
  \item[] $f(x) = \frac12 x^TQx+q^Tx\rightarrow\min$ bei $Ax\le b$
\end{itemize}
Seien $Q\in\mathbb{R}^{n\times n}$, $Q>0$ possitiv definit, $q\in\mathbb{R}^n$, $A=\begin{bmatrix}a_1^T\\\vdots\\ a_m^T\end{bmatrix}\in\mathbb{R}^{m\times n}$, $b\in\mathbb{R}^m$.\\
Für $x\in G:=\left\{x\in\mathbb{R}^n|\ Ax\le b \right\}$ bezeichne $I_a(x)=\left\{i\in(1,\ldots,m):a_i^Tx=b_i\right\}$\footnote{Menge der im Punkt $x\in G$ aktiven Restriktionen} und
$M(I) = \left\{y\in\mathbb{R}^n|\ a_i^Ty=b_i, i\in I_a(x) \right\}$.\\
Sei $\left\{a_i|\ i\in I_a(x) \right\}$ linear unabhängig $\forall x\in G$.\\
\ac{NB}:
\begin{align*}
  g_i(x) := a_i^Tx -b_i & \le 0\\
  \nabla g_i(x) & = a_i
\end{align*}

\textbf{Algorithmus der Aktive-Mengen-Strategie}:
\begin{enumerate}[label=(S\arabic*)]
  \item Wähle ein $x^0\in G$, setze $I_0:=I_a(x^0)$ und setze $k:=0$.
  \item Falls Projektion von $\nabla f(x^k)=Qx^k+q$ auf $M(I_k)$ gleich dem Nullvektor ist, so sind die zu $x^k$ gehörigen Lagrange-Multiplikatoren
  $u_i^k$ aus $\nabla f(x^k)+\sum\limits_{i\in I_k}u_i^ka_i=0$ zu bestimmen.
  \begin{enumerate}[label=(S2\alph*)]
    \item Falls $u_i^k\ge 0$ mit $i\in I_k$, so ist $x^k$ Optimalstelle. Stop!
    \item Andernfalls sind der Index $r\in I_k$ mit $u_i^k<0$ zu bestimmen und es ist $I_k\leftarrow I_k\backslash\{r\}$\footnote{Interpretation: Entferne $r$ aus der Indexmenge $I_k$}
    zu setzen.
  \end{enumerate}
  \item Bestimme $y^k:=\argmin f(y)$\footnote{Bemerkung: $x^{\star}=\argmin f(x) \leftrightarrow x^{\star}\ldots$ Minimumstelle von $f$} bei $y\in
  M(I_k)$.\\
  		Setze $\alpha_k=\max\left\{\alpha\in[0,1]:x^k+\alpha(y^k-x^k)\in G\right\}$, $x^{k+1}=x^k+\alpha(y^k-x^k)$, $I_{k+1}:=I_a(x^{k+1})$ und gehe mit
  		$k\rightarrow k+1$ zu (S2)
\end{enumerate}

Bemerkung zu (S3):
\begin{itemize}
  \item[] Die im Punkt $x^k$ inaktive UNBen sind $a_j^Tx^k\le b_j$, $j\notin I_k$.
  \item[] Wir bestimmen das maximale $\beta > 0$ derart, dass $\beta(y^k-x^k)+x^k$ noch zulässig ist, d.h. $a_j^T(\beta(y^k-x^k)+x^k)\le b_j$.
  \item[] Dies ist wegen $a_j^Tx^k\le b_j$ und $\beta a_j^T(y^k-x^k)\le b_j-a_j^Tx^k$ stets für diejenigen $j$ erfüllt, für die $a_j^T(y^k-x^k)\le 0$ gilt. 
  \item[] Also ergibt sich das gesuchte \begin{align*}
  \beta & = \min\left\{\left.\ gfrac{b_j-a_j^Tx^k}{a_j^T(y^k-x^k)}\right|\ j\notin J_k\text{ mit } a_j^T(y^k-x^k)>0 \right\}.
  \end{align*}
  \item[] (S3) berechnet $\alpha_k=\min\left\{1,\beta\right\}$.
\end{itemize}

Bemerkung zum Algorithmus:
\begin{itemize}
  \item Die Voraussetzungen ``$C>0$'' und ``$\left\{a_i|\ i\in I_a(x)\right\}$ linear unabhängig $\forall k\in G$'' sichern die eindeutige Lösbarkeit des linearen Gleichungssystems in
  (S2) und (S3)
  \item Alle Iterationen $x^k$ sind zulässig. In der Praxis gilt bei jedem Durchlauf von (S3): $f(x^{k+1})<f(x^k)$. Der Algorithmus ist endlich (gutartiger, schneller Algorithmus).
\end{itemize}

\textbf{Finden eines zulässigen Startpunktes}:\\
Zur Aufgabe 
\begin{align}
	f(x) & =\frac12 x^TQx+q^Tx\rightarrow\min\limits_{x\in\mathbb{R}^n} \text{ bei } Ax\le b \label{eq:startpkt_1}
\end{align} 
bildet Aufgabe 
\begin{align}
	y & \rightarrow\min\limits_{\substack{x\in\mathbb{R}^n\\y\in\mathbb{R}^1}} \text{ bei } Ax\le b+\mathds{1}_m y,\ 0\le y \label{eq:startpkt_2}\\
	& \text{mit: } \mathds{1}_m\ldots \text{ Einheitsmatrix} \notag\\
	\Leftrightarrow \begin{bmatrix} 0 & \ldots & 0 & 1 \end{bmatrix}\begin{bmatrix} x\\y \end{bmatrix} & \rightarrow\min  \text{ bei } \begin{bmatrix} A & -\mathds{1}_m\\0 & -1
	\end{bmatrix}\begin{bmatrix} x\\y \end{bmatrix}\le \begin{bmatrix} b\\0 \end{bmatrix} \label{eq:startpkt_3}
\end{align}
Zulässiger Startwert:
\begin{itemize}
  \item[] $x^0$ beliebig und $y^0:=\max\left([0;\ Ax^0-b]\right)$
  \item[] $\begin{bmatrix}\overline{x}\\ \overline{y} \end{bmatrix}$ sei Lösung der linearen Optimierungsaufgabe \eqnref{eq:startpkt_3}.
  \item[] Falls $\overline{y}=0$, so gilt 
  $A\overline{x}\le b$, d.h. $\overline{x}$ ist zulässig von \eqnref{eq:startpkt_1}.
\end{itemize}

\textbf{Erweiterungen des Algorithmus}:
\begin{itemize}
  \item Falls $Q=0$, d.h. $f(x)=q^Tx\rightarrow\min$ bei $Ax\le b$, so liegt eine Aufgabe der linearen Optimierung vor. In (S3) wir dann $\alpha_k:=\max\left\{ \alpha\ge 0: x^k+\alpha
  s^k\in G \right\}$ wobei $s^k$ der auf $M(I_k)$ projizierte negative Gradient $-\nabla f(x^k)=-q$ ist.
  \item Treten UNB und GNB gleichzeitig auf, d.h. $f(x)=\frac12 x^TQx+q^Tx\rightarrow\min$ bei $Ax\le b$ und $Cx=d$ so werden die GNB wie aktive UNB behandelt.
 \end{itemize}

\begin{exmp}
Seien 
\begin{align*}
	A & =\begin{bmatrix}1 & \frac13\\0 & 1\\ -1 & 0 \end{bmatrix},\quad b=\begin{bmatrix}1\\1\\0 \end{bmatrix},\quad Q=\begin{bmatrix}2 & 0\\ 0 & 2
	\end{bmatrix},\quad q=\begin{bmatrix}-2\\-4 \end{bmatrix}.
\end{align*}
\begin{enumerate}[label=(S\arabic*)]
  \item Mit $x^0:=\begin{bmatrix}0\\0 \end{bmatrix}$ ist $I_0=\{3\}$.
  \item Projektion von $\nabla f(x^0)=\begin{bmatrix}-2\\-4 \end{bmatrix}$ auf $M(I_0)=\left\{\begin{bmatrix}0\\x_2 \end{bmatrix}:x_2\in\mathbb{R}
  \right\}$ ist $\begin{bmatrix}0\\-4 \end{bmatrix}\neq 0$.
  \item Man findet $y^0=\begin{bmatrix}0\\2 \end{bmatrix}$, $x^1=\begin{bmatrix}0\\1 \end{bmatrix}$, $I_1=\{2,3\}$.
  \item[(S2)] Projektion von $\nabla f(x^1)=\begin{bmatrix}-2\\-2 \end{bmatrix}$ auf $M(I_1)=\left\{\begin{bmatrix}0\\1
  \end{bmatrix}\right\}$\footnote{Bemerkung: d.h. nur ein Punkt! Projektion auf 1 Punkt ist immer der Nullvektor.} ist $\begin{bmatrix}0\\0
  \end{bmatrix}$ an der Stelle an der beide Restriktionen wirken mit $I_1=\{2,3\}$.\\
  Lagrange-Multiplikatoren $\begin{bmatrix}u_2^1\\u_3^1 \end{bmatrix}=\begin{bmatrix}2\\-2 \end{bmatrix}$, setze $I_1\leftarrow
  I_1\backslash\{3\}=\{2\}$.
  \item[(S3)] Man findet $y^1=\begin{bmatrix}1\\1 \end{bmatrix}$, $x^2=\begin{bmatrix}\frac23\\1 \end{bmatrix}$, $I_2=\{1,2\}$.
  \item[(S2)] Projektion von $\nabla f(x^2)=\begin{bmatrix}-\frac23\\-2 \end{bmatrix}$ auf $M(x^2)$ ist $\begin{bmatrix}0\\0 \end{bmatrix}$.\\
  		Lagrange-Multiplikatoren $\begin{bmatrix}u^2_1\\u_2^2 \end{bmatrix}=\begin{bmatrix}\frac23\\\frac{16}{9} \end{bmatrix}\ge 0$. Stop!
\end{enumerate}
\end{exmp}

\textbf{Grafische Darstellung der Optimierungsaufgabe und deren Lösung}:\\
\newline
\begin{minipage}[c]{0.5\textwidth}
Zielfunktion:
\begin{itemize}
  \item[] $f(x)=(x_1-1)^2+(x_2-2)^2-5$ 
\end{itemize}
Zulässiger Bereich:
\begin{enumerate}[label=(\arabic*)]
  \item $x_1+\frac{x_2}{3}\le 1$
  \item $x_2\le 1$
  \item $x_1 \ge 0$ 
\end{enumerate}
\end{minipage}
\hfill
\begin{minipage}[c]{0.5\textwidth}
Bild folgt
\end{minipage}

\begin{uea}
Aufgabe: 
\begin{itemize}
  \item[] Erweitern Sie den Algorithmus um die Auffindung eines zulässigen Startpunktes!
\end{itemize}
Lösung: 
\end{uea}

\begin{uea}
Aufgabe: 
\begin{itemize}
  \item[] Erweitern Sie den Algorithmus um die Aufgabe der linearen Optimierung und die Verwendung von GNB als aktive UNB!
\end{itemize}
Lösung: 
\end{uea}
