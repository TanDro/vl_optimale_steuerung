\section*{Übungen Kapitel 3}
\addcontentsline{toc}{section}{Übungen Kapitel 3} 
\label{sec:uebung_kapitel_3}

\subsection*{Das zeitdiskrete \NoCaseChange{\acl{LQR}}-Problem}
\addcontentsline{toc}{subsection}{Das zeitdiskrete \NoCaseChange{\acl{LQR}}-Problem}
\label{sec:uebung_zeitdiskretes_lqr_problem}
Überführen Sie die \exmpref{} und \exmpref{} aus \secref{} jedoch ohne Endbedingungen und Stellgrößenbeschränkungen in ein zeitdiskretes \ac{LQR}-Problem.

\subsection*{Beispiel der räumlichen Führung}
\addcontentsline{toc}{subsection}{Beispiel der räumlichen Führung} 
\label{sec:uebung_raeumliche_fuehrung}
\begin{enumerate}
  \item Überlegen Sie, warum in \eqnref{eqn:kap_3_bsp_rf_bsp_konst_regler} eine konstante Rückführung mit $F_{12}$ anstelle von $F_{10}$ zu besseren Ergebnissen führen könnte.
  \item Praktisch steht die Gierwinkelrate $\Delta\dot{\gamma}$ nicht direkt als Steuergröße zur Verfügung. Vielmehr kann diese nur zeitlich verzögert beeinflusst werden, was über die
  Differentialgleichung $T\cdot\Delta\ddot{\gamma}+\Delta\dot{\gamma}=u$ mit einer Zeitkonstante $T$ (deutlich unter einer Sekunde) modelliert werden kann. Stellen Sie die lineare
  zeitvariante Zustandsdifferentialgleichung für das resultierende System 2. Ordung auf und machen Sie sich anhand eines Beispiels durch Simulation klar, dass ein zeitlich konstanter
  Regler im Vergleich zum zeitvarianten \ac{LQR}-Regler zu deutlichen Stabilitätsproblemen führt.
\end{enumerate}