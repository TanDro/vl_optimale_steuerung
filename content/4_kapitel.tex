\chapter{\NoCaseChange{\acl{MPR}}}
\section{Einleitung}
Die im \secref{sec:3_3_zeitkont_opt_endl} beschriebene Vorgehensweise zur Lösung eines Optimalsteuerproblems auf Basis der Dynamsichen Programmierung ist selbst für lineare
Prozesse mit quadratischem Kostenfunktional mit einem hohen Aufwand verbunden, wenn Beschränkungen im Steuer- und Zustandsraum auftreten (was die
Anwendung des \ac{LQR}-Entwurfes verhindert) und dabei der Zustands- und Steuerraum keine geringe Dimension aufweisen oder der Optimeirungshorizont
nicht klein ist. Bei der numerischen Lösung stellt die Vorwärtsrechnung dann durch die Fallunterscheidungen hohe Anforderungen an die Rechenleistung
und die erforderliche Speicherung der berechneten Zustandsrückführungen $u_{\kappa}^{\ast}$ wird bei einer ausreichend feinen Diskretisierung des
Zustandsraumes im Allgemeinen einen hohen Speicherplatzbedarf verursachen.

Eine Alternative zu dieser Vorgehensweise ist die Berechnung der optimales Steuergrößen $u^{\ast}[0]$, $u^{\ast}[1]$, $\ldots$, $u^{\ast}[k-1]$ durch eine Lösung des
Optimierungsproblems ? für den gegebenen Anfangszustand $x[0]$ durch numerische Optimierung in Echtzeit. Da hierbei zum Zeitpunkt $\kappa=0$ eine Steuerfolge $u[\kappa]$ für den
gesamten Horizont $\kappa=0,1,\ldots,k-1$ berechnet wird, tritt anders als bei den Zustandsrückführungen $u_{\kappa}^{\ast}$ keine Rückkopplung des tatsächlich erreichten
Prozesszustandes mehr auf. Praktisch wird dann, z.B. aufgrund von Modellunbestimmtheiten oder einwirkenden Störgrößen, eine Abweichung der erreichten Zustände von der bei der
Optimeierung berechneten optimalen Zustandstrajektorie auftreten. Man wird deshalb die Lösung des Optimierungsproblems ? in jedem Zeitschritt mit dem
aktuellen Prozesszustand als Anfangszustand wiederholen und somit eine Rückführung, also eine Regelung, erreichen. Dieser Lösungszugang ist unter der
Bezeichnung \ac{MPR} bekannt und soll im Weiteren einführend für lineare Mehrgrößenprozesse mit Beschränkungen im Steuer- und Zustandsraum dargestellt werden.

\section{Formulierung der Regelungsaufgabe}
\subsection{Grundprinzip der prädiktiven Regelung und Begriffe}
\label{subsec:grundprinzip_mpr}
Gegeben sei der lineare zeitinvariante Prozess
\begin{align}
	\begin{split}\label{eqn:kap_4_zustandsraummodell}
	x[\kappa+1] & = A x[\kappa] + B u[\kappa] \\
	y[\kappa] & = Cx[\kappa]
	\end{split}
\end{align}
in diskreter Zeit. Mit $\kappa=0,1,2,\ldots$ werden die diskreten Zeitpunkte bezeichnet und $k$ bezeichnet den \textit{aktuellen Zeitpunkt}. Die Steuergrößen $u[\kappa]$ seien für
$\kappa < k$ und die Prozessausgänge $y[\kappa]$ für $\kappa \le k$ bekannt. Die Aufgabe besteht in der Bestimmung der aktuellen Steuergröße $u[k]$. Dazu bezeichne
\begin{align}
	y_f[k +i|k] & := CA^ix[k]\qquad (i\ge 0)
\end{align}
die freie Bewegung des Prozessausganges ab dem Zeitpunkt $k$ und 
\begin{align}
	\hat{y}[k+i|k] & := CA^ix[k]+C\sum\limits_{j=0}^{i-1}A^{i-j-1}B\hat{u}[k+j|k]\qquad (i=1,2,\ldots,H_p) \label{eqn:kap_4_praediktion}
\end{align}
die \textit{Prädiktion des Prozessausganges} über dem \textit{Prädiktionshorizont} der Länge $H_p$.
\begin{exmp}[Prädiktionshorizont $H_p=2$]
\begin{align*}
	x[k+2] & = Ax[k+1]+Bu[k+1]\\
	& = A^2x[k]+ABu[k]+Bu[k+1]\\
	y[k+2] & = CA^2x[k]+\underbrace{C\left( ABu[k]+Bu[k+1] \right)}_{C\sum\limits_{j=0}^1A^{1-j}Bu[k+j]}
\end{align*}
\end{exmp}
Weiter wird der \textit{Steuerhorizont} der Länge $H_u$ definiert, es gelte $H_p\ge H_u \ge 1$ und die Steuergröße verändere sich ab dem Zeitpunkt $k+H_u$ nicht mehr\footnote{Die
Bedingung $H_p\ge H_u$ sichert, dass die Auswirkungen eines Steuereingriffs auf den Prozessausgang über einen ausreichend großen Horizont $H_p-H_u$
berücksichtigt werden. Praktisch gilt aus diesem Grund sogar oft $H_p \gg H_u$.}, d.h. es sei
\begin{align}
	\hat{u}[k+j|k] & = \hat{u}[k+H_u-1|k]\qquad (H_u\le j\le H_p-1).\label{eqn:kap_4_steuerhorizont}
\end{align}
\begin{figure}[htb]
	\centering
	\begin{tikzpicture}[auto, >=latex'] 
	\node [input] () {};
\end{tikzpicture}

	\caption{Grundprinzip der prädiktiven Regelung}
	\label{fig:kap_4_grundprinzip_mpr}
\end{figure}
In \figref{fig:kap_4_grundprinzip_mpr} sind die bekannten Verläufe des Prozessausganges und der Steuergröße bis zum Zeitpunkt $k$ zu sehen. Wenn ab dem Zeitpunkt $k$ die Steuergrößen
$\hat{u}[k+i-1|k]$ ($i=1,2,\ldots,H_p$) auf den Prozess gegeben würden, so würde sich unter der Annahme eines Haltegliedes 0. Ordnung für die Steuergrößen gemäß
\eqnref{eqn:kap_4_praediktion} in den Abtastzeitpunkten der Prozessausgang $\hat{y}[k+i|k]$ ergeben. Durch ein numerisches Optimeirungsverfahren wird eine Steuerfolge so bestimmt, dass
vorgegebene Forderungen erfüllt werden. Beispielhaft für solche Forderungen sind in \figref{fig:kap_4_grundprinzip_mpr} konstante Schranken $\underline{u}$ und $\overline{u}$ für die
Steuergröße
\begin{align}
	\underline{u}&\le \hat{u}[k+i-1|k]\le \overline{u} \label{eqn:kap_4_ugnb_u}
\end{align}
und eine konstante obere Schranke $\overline{y}[k+i|k]$ und eine zeitabhängige untere Schranke $\underline{y}[k+i|k]$ für den Prozessausgang
\begin{align}
	\underline{y}[k+i|k]&\le \hat{y}[k+i|k]\le\overline{y}[k+i|k] \label{eqn:kap_4_ugnb_y}
\end{align}
darstellt. Darüber hinaus ist eine \textit{Sollwerttrajektorie} $s[\kappa]$ für den Prozessausgang eingezeichnet. Aus dieser Sollwerttrajektorie und
dem Wert des Prozessausganges zum Zeitpunkt $k$ wird die \textit{Referenztrajektorie} $r[k+i|k]$ abgeleitet, z.B. mit einer gegeben Zeitkonstante $T_c>0$ in der Form
\begin{align}
	r[k+i|k] & := s[k+i]-(s[k]-y[k])e^{-iT_c},
\end{align}
und die Zielstellung verfolgt, den Abstand (mit einem noch zu definierendem Abstandsmaß) des Prozessausganges von der Referenztrajektorie zu minimieren\footnote{Der Prozessausgang wird
nicht (direkt) der Sollwerttrajektorie, sondern der Referenztrajektorie nachgeführt. Die Referenztrajektorie gibt an, wie ein von der Sollwerttrajektorie abweichender Prozessausgang
wieder zur Sollwerttrajektorie zurückkehren soll. Die Verwendung der Referenztrajektorie anstelle der Sollwerttrajektorie zur Vorgabe des Verlaufes des Prozessausganges hat im
Wesentlichen eine praktische Motivation: Da die Regelabweichung $r[k+i|k]-\hat{y}[k+i|k]$ hinsichtlich des Betrages im Allgemeinen kleiner ist als $s[k+i]-\hat{y}[k+i|k]$ kann man
erwarten, dass bei den in der Praxis stets vorliegenden nichtlinearen Prozessen ein um die Referenztrajektorie linearisiertes Prozessmodell eine genauere Prozessbeschreibung darstellt,
als ein Modell, dass durch eine Linearisierung um die Sollwerttrajektorie gewonnen wird.}. Wenn eine optimale Steuerfolge $u^{\ast}[k+i-1|k]$ ($i=0,1,\ldots,H_u-1$) berechnet ist, so
wird das erste Glied der optimalen Steuerfolge als Steuergröße im Schritt $k$ ausgegeben. Im Schritt $k+1$ wird dann die gesamte Prozedur wiederholt. Der Prädiktionshorizont, der wie
auch der Steuerhorizont in seiner Länge unverändert bleibt, wird dabei mit dem aktuellen Zeitpunkt weitergeschoben. Man spricht von einem gleitenden
Horizont. Insgesamt ergibt sich die folgende Vorgehensweise.\\
\textsc{Prinzipieller Algorithmus der Modellprädiktiven Regelung}
\begin{enumerate}[label=(S\arabic*)]
  \item Minimiere über alle Steuerfolgen $u[k+i|k]$ ($i=0,1,\ldots,H_u-1$) ein Maß für den Abstand zwischen $r[k+i|k]$ und $\hat{y}[k+i|k]$ ($i=1,2,\ldots,H_p$) unter den
  Nebenbedingungen \eqnref{eqn:kap_4_ugnb_u} und \eqnref{eqn:kap_4_ugnb_y}.
  \item Setze für die Steuergröße $u[k]:=u^{\ast}[k|k]$.
  \item Warte einen Zeitschritt, setzte $k:=k+1$ und gehe zu (S1).
\end{enumerate}
In \cite{Maciejowski2002} werden folgende Merkmale einer modellprädiktiven Regelung hervorgehoben.
\begin{itemize}
  \item Vorhandensein eines expliziten Prozessmodells\\
  		Das Prozessmodell dient im Rahmen der Lösung des Optimierungsproblems im Schritt (S1) zur Bestimmung des für eine vorgegebene Steuerfolge $u[k+i|k]$ ($i=0,1,\ldots,H_u-1$)
  		resultierenden Prozessausganges. Das Prozessmodell muss nicht notwendig ein parametrisches Modell sein, sondern kann z.B. auch durch eine
  		Sprungantwort beschrieben sein.
  \item Optimierung in Echtzeit\\
  		Die optimale Steuerfolge, von der jeweils nur das erste Glied als Steuergröße an den Prozess auch tatsächlich ausgegeben wird, wird in jedem diskreten Zeitschritt durch ein
  		numerisches Optimeirungsverfahren in Echtzeit berechnet.
  \item Gleitender Optimierungshorizont\\
  		Im Gegensatz zur Vorgehensweise im Abschnitt ?, bei dem der aktuelle Zeitpunkt in jedem Schritt näher an den Endzeitpunkt heranrückt, gleiten hier der Steuer- und
  		Prädiktionshorizont mit dem aktuellen Zeitpunkt weiter. 
\end{itemize}

\subsection{Prozessbeschreibung}
Für die weiteren Darstellungen wird von den Steuergrößen $u[\kappa]$ zu den Steuergrößendifferenzen
\begin{align}
	\Delta u[\kappa] & := u[\kappa]-u[\kappa -1]
\end{align}
übergegangen, womit sich eine Regelkreisstruktur wie in \figref{fig:kap_4_steuergroessen_differenzen} ergibt\footnote{Dies hat in erster Linie prakische Gründe, denn da der
\ac{MPR}-Regler nur Steuergrößendifferenzen ("`Steuergrößenkorrekturen"') erarbeitet, kann z.B. ein stoßfreies Umschalten zwischen verschiedenen Reglern problemlos realisiert werden und auch eine Regelung um einen Arbeitspunkt
(wie z.B. bei der Regelung nichtlinearer Prozesse) ist einfacher realisierbar (\cite{Maciejowski2002}).}. Eine mögliche Zustandsraumdarstellung des
\ac{MPR}-Prozesses kann aus \eqnref{eqn:kap_4_zustandsraummodell} mit
\begin{align}
	\xi[\kappa] & := \begin{bmatrix}
	x[\kappa]\\ u[\kappa -1]
	\end{bmatrix}
\end{align}
in der Form ("`erweitertes Modell"')
\begin{align}
\begin{split}\label{eqn:kap_4_erweitertes_modell}
	\xi[\kappa+1] & = \begin{bmatrix}
	A & B\\ 0 & I
	\end{bmatrix}\xi[\kappa]+\begin{bmatrix}
	B\\ I
	\end{bmatrix}\Delta u[\kappa]\\
	y[\kappa] & = \begin{bmatrix}
	C & 0
	\end{bmatrix}\xi[\kappa]
\end{split}
\end{align}
gewonnen werden.
\begin{figure}[htb]
	\centering
	\begin{tikzpicture}[auto, >=latex'] 
	\node [input] () {};
\end{tikzpicture}

	\caption{Übergang von den Steuergrößen zu den Steuergrößendifferenzen}
	\label{fig:kap_4_steuergroessen_differenzen}
\end{figure}
Stellt man nun die Gleichung für die Prädiktion des Zustandsvektors auf
\begin{align}
	\hat{x}[k+i|k] & = A^ix[k]+\sum\limits_{j=0}^{i-1}A^{i-j-1}B\hat{u}[k+j|k]
\end{align}
und ersetzt die Steuergröße $\hat{u}[k+j|k]$ durch die Steuergrößendifferenz
\begin{align}
\begin{split}\label{eqn:kap_4_steuergroessendiff}
	\Delta\hat{u}[k|k] & = \hat{u}[k|k]-u[k-1]\\
	\Delta\hat{u}[k+i|k] & = \hat{u}[k+i|k]-\hat{u}[k+i-1|k]\quad\text{für }i=1,2,\ldots,H_u-1\\
	\Delta\hat{u}[k+i|k] & = 0\quad\text{für }i=H_u-1,\ldots,H_p-1
\end{split}
\end{align}
so erhält man aus \eqnref{eqn:kap_4_zustandsraummodell} die Darstellung des \ac{MPR}-Prozesses
\begin{align*}
\begin{bmatrix}
\hat{x}[k+1|k]\\
\vdots\\
\hat{x}[k+H_u|k]\\
\hat{x}[k+H_u+1|k]\\
\vdots\\
\hat{x}[k+H_p|k]
\end{bmatrix} & = \underbrace{\begin{bmatrix}
A\\ \vdots\\ A^{H_u}\\ A^{H_u+1}\\ \vdots\\ A^{H_p}
\end{bmatrix}}_{\Psi}x[k]+\underbrace{\begin{bmatrix}
B\\ \vdots \\ \sum_{i=0}^{H_u-1}A^iB\\ \sum_{i=0}^{H_u}A^iB\\ \vdots \\ \sum_{i=0}^{H_p-1}A^iB
\end{bmatrix}}_{\Upsilon}u[k-1]\\
&\quad +\underbrace{\begin{bmatrix}
B 						&	\ldots 	& 	0\\
\vdots					&	\ddots	& 	\vdots\\
\sum_{i=0}^{H_u-1}A^iB	& \ldots	& 	B\\
\sum_{i=0}^{H_u}A^iB	& \ldots	& 	AB+B\\
\vdots					&			& 	\vdots\\
\sum_{i=0}^{H_p-1}A^iB	& \ldots	& 	\sum_{i=0}^{H_p-H_u}A^iB
\end{bmatrix}}_{\Theta} \underbrace{\begin{bmatrix}
\Delta\hat{u}[k|k]\\
\vdots\\
\Delta\hat{u}[k+H_u-1|k]
\end{bmatrix}}_{\Delta\mathcal{U}[k]},\\
\begin{bmatrix}
\hat{y}[k+1|k]\\ \vdots\\ \hat{y}[k+H_p|k]
\end{bmatrix}[k]& = \begin{bmatrix}
C	&			&	\\
	& \ddots	&	\\
	&			& C
\end{bmatrix}\begin{bmatrix}
\hat{x}[k+1|k]\\ \vdots\\ \hat{x}{k+H_p|k}
\end{bmatrix},
\end{align*}
die man mit 
\begin{align*}
	\mathcal{Y}[k] & := \begin{bmatrix}
	\hat{y}[k+1|k]\\
	\vdots\\
	\hat{y}[k+H_p|k]
	\end{bmatrix},\ \Delta\mathcal{U}[k]:=\begin{bmatrix}
	\Delta \hat{u}[k|k]\\
	\vdots\\
	\Delta\hat{u}[k+H_u-1|k]
	\end{bmatrix}
\end{align*}
und entsprechender Defintion der Matrizen $\mathrm{\Upsilon}$, $\Psi$ und $\Theta$ kürzer in der Form % TODO
\begin{align}
\mathcal{Y}[k] & = \underbrace{\Psi x[k]+\Upsilon u[k-1]}_{\text{freie Bewegung}}+\Theta\Delta\mathcal{U}[k] \label{eqn:kap_4_zustandsraummodell_steuerdiff}
\end{align}
schreiben kann.

Mit der Gleichung \eqnref{eqn:kap_4_zustandsraummodell_steuerdiff} ist bei gegebenen Werten $x[k]$ und $u[k-1]$ die Prädiktion des Prozessausganges
als lineare Funktion der Steuergrößendifferenzen $\Delta\mathcal{U}[k]$ dargestellt. Im aktuellen Schritt $k$ ist die im vorhergehenden Schritt $k-1$
an den Prozess ausgegebene Steuergröße $u[k-1]$ bekannt. Der Zustand $x[k]$ wird im Weiteren ebendalls als bekannt vorausgesetzt. Sollte $x[k]$ nicht
direkt messbar sein, so wird $x[k]$ in \eqnref{eqn:kap_4_zustandsraummodell_steuerdiff} durch einen Schätzwert $\hat{x}[k|k]$ eines
Zustandsbeobachters ersetzt. Sollte im \ac{MPR}-Prozess \eqnref{eqn:kap_4_zustandsraummodell} darüber hinaus eine direkt gemessene Störgröße $d_m[k]$
oder eine durch einen erweiterten Beobachter ermittelte Störgröße mit dem Wert $\hat{d}[k|k]$ auftreten, so kommt auf der rechten Seite von
\eqnref{eqn:kap_4_zustandsraummodell_steuerdiff} zusätzlich ein Summand
\begin{align*}
	\mathcal{D}_m\begin{bmatrix}
	d_m[k]\\ \hat{d}_m[k+1|k]\\ \vdots \\ \hat{d}_m[k+H_p-1|k]
	\end{bmatrix}\quad\text{bzw.}\quad \mathcal{D}\begin{bmatrix}
	\hat{d}[k|k]\\ \hat{d}[k+1|k]\\ \vdots\\ \hat{d}[k+H_p-1|k]
	\end{bmatrix}
\end{align*}
mit den Matrizen $\mathcal{D}$ und $\mathcal{D}_m$ hinzu. Dabei bezeichnen $\hat{d}_m[k+i|k]$ und $\hat{d}[k+i|k]$ die Prädiktion der direkt gemessen bzw. beobachteten Störgröße, die
z.B. unter der Annahme konstanter Störgrößen gemäß $\hat{d}_m[k+i|k]:=d_m[k]$ bzw. $\hat{d}[k+i|k]:=\hat{d}[k|k]$ bestimmt werden ($i=1,\ldots,H_p-1$). Die Modellgleichung
\eqnref{eqn:kap_4_zustandsraummodell_steuerdiff} bleibt in diesem Fall linear in den Steuergrößendifferenzen $\Delta\mathcal{U}[k]$, so dass auch bei auftretenden Störgrößen die im
Weiteren beschriebene Vorgehensweise anwendbar ist.

\subsection{Darstellung des Gütefunktionals und der Beschränkungen für die Steuer- und Regelgrößen}
Die Bestimmung der Steuergrößen $\Delta\hat{u}[k+i|k]$ für den \ac{MPR}-Prozess ($i=0,1,\ldots,H_u-1$) erfolgt so, dass ein vorgegebenes Gütefunktional minimiert wird und dabei
vorgegebene Schranken für die Steuergrößen und für die Regelgrößen eingehalten werden.

Im Folgenden werden drei Klassen von Gütefunktionalen betrachtet, das quadratische Gütefunktional ($\|.\|_2$-Funktional)
\begin{align}
V[k] & = \sum\limits_{i=1}^{H_p}\left\|Q_y[i]\left(\hat{y}[k+i|k]-r[k+i|k]\right)\right\|_2^2+\sum\limits_{i=0}^{H_u-1}\left\|Q_{\Delta u}[i]\Delta\hat{u}[k+i|k]\right\|_2^2
\label{eqn:kap_4_quad_guetefunktional},
\end{align}
das 1-Norm-Gütefunktional ($\|.\|$)
\begin{align}
V[k] & = \sum\limits_{i=1}^{H_p}\left\|Q_y[i]\left(\hat{y}[k+i|k]-r[k+i|k]\right)\right\|_1+\sum\limits_{i=0}^{H_u-1}\left\|Q_{\Delta
u}[i]\Delta\hat{u}[k+i|k]\right\|_1,\label{eqn:kap_4_eins_norm_guetefunktional}
\end{align}
und das Maximum-Norm-Gütefunktional ($\|.\|_{\infty}$-Funktional)
\begin{align}
	V[k] & = \max\limits_{i=1,\ldots,H_p}\left\|Q_{y}[i]\left(\hat{y}[k+i|k]-r[k+i|k] \right)\right\|_{\infty}\label{eqn:kap_4_max_norm_guetefunktional_1}
\end{align}
bzw.
\begin{align}
	V[k] & = \max\limits_{i=0,\ldots,H_u-1}\left\|Q_{\Delta u}[i]\Delta\hat{u}[k+i|k] \right\|_{\infty}.\label{eqn:kap_4_max_norm_guetefunktional_2}
\end{align}
Hierbei sind die Gewichtsmatrizen $Q_{y}[i]$ positiv semidefinit und $Q_{\Delta u}[i]$ positiv definit, oft werden sie sogar als Diagonalmatrizen gewählt. Quadratische Gütefunktionale
sind bereits aus dem Abschnitt ? bekannt und haben wichtige Anwendungen bei statistische Aufgabenstellungen. Das 1-Norm-Gütefunktional tritt z.B. bei der Verbrauchsoptimierung auf,
während Probleme, bei denen der ungünstigste Fall (Worst-Case) betrachtet wird, oft auf Maximum-Norm-Gütefunktional führen.
\begin{exmp}\label{exmp:kap_4_bsp_siso_nb}
Falls für den Prozess \eqnref{eqn:kap_4_zustandsraummodell} bzw. das quadratische Funktional \eqnref{eqn:kap_4_quad_guetefunktional} speziell die Bedingungen 
\begin{align*}
H_u & = H_p,\quad C=I,\quad r[k+i|k]=0,\quad Q_y[i]=Q_y,\quad Q_{\Delta u}[i]=Q_{\Delta u}\ \forall i
\end{align*}
erfüllt sind, so kann \eqnref{eqn:kap_4_quad_guetefunktional} mit $\bar{Q}_x:=Q_y^TQ_y$ und $\bar{Q}_{\Delta u}:=Q^T_{\Delta u}Q_{\Delta u}$ in der Form
\begin{align}
\begin{split}\label{eqn:kap_4_bsp_1_kostenfunktionial}
V[k] & = \sum\limits_{i=1}^{H_p-1}\hat{x}[k+i|k]^T\bar{Q}_{x}\hat{x}[k+i|k]+\sum\limits_{i=0}^{H_p-1}\Delta\hat{u}[k+i|k]^T\bar{Q}_{\Delta u}\Delta\hat{u}[k+i|k]\\
& \quad +\hat{x}[k+H_p|k]^T\bar{Q}_{x}\hat{x}[k+H_p|k]
\end{split}
\end{align}
geschrieben werden. Abgesehen von dem Term $\hat{x}[k|k]^T\bar{Q}_{x}\hat{x}[k|k]$, der aber durch die Wahl der $\Delta\hat{u}[k+i|k]$ ($i=0,\ldots,H_p-1$) nicht beeinflusst wird, ist
das Gütefunktional \eqnref{eqn:kap_4_bsp_1_kostenfunktionial} gleich dem aus dem Abschnitt ? bekannten \ac{LQR}-Funktional (angewendet auf den \ac{MPR}-Prozess).
\end{exmp}
Die Beschränkungen der Regelgröße, der Steuergrößendifferenzen und der Steuergrößen werden nachfolgend durch lineare Nebenbedingungen in der Form
\begin{align}
M[k]\begin{bmatrix}
\mathcal{Z}[k]\\ \Delta\mathcal{U}[k]\\ \mathcal{U}[k]\\ 1
\end{bmatrix} & \le 0	\label{eqn:kap_4_lin_nebenbedingungen}
\end{align}
dargestellt.
\begin{exmp}
Gegeben seien ein \ac{SISO}-Prozess, das Gütefunktional
\begin{align*}
	V[k] & = \max\limits_{i=0,\ldots,H_u-1}\left|\Delta\hat{u}[k+i|k] \right|
\end{align*}
und die Nebenbedingungen
\begin{align}
\begin{split}\label{eqn:kap_4_bsp_1_nebenbedingungen}
	\left|\hat{y}[k+i|k]-r[k+H_p|k] \right| & \le \frac{1}{i+1}\left| y[k] - r[k+H_p|k] \right|\quad (i=1,\ldots,H_p)\\
	\hat{y}[k+H_p|k] & = r[k+H_p|k].
\end{split}
\end{align}
Durch das Gütefunktional wird das Maximum der Steuergrößenänderung minimiert. Diese Aufgabenstellung kann für ein mechanisches System dessen Steuergrößen Kräfte oder Momente sind, als
eine Minimierung des sogenannten maximalen Rucks (das ist die Ableitung der Kraft bzw. des Momentes) interpretiert werden.
\begin{figure}[htb]
	\centering
	\begin{tikzpicture}[auto, >=latex'] 
	\node [input] () {};
\end{tikzpicture}

	\caption{Veranschaulichung der Nebenbedingungen zum \exmpref{exmp:kap_4_bsp_siso_nb}}
	\label{fig:kap_4_bsp_nebenbedingungen}
\end{figure}
Die Nebenbedingungen der Aufgabe sind Forderungen an den Verlauf der Regelgröße. Wie in der \figref{fig:kap_4_bsp_nebenbedingungen} veranschaulicht, wird eine fortschreitende Annäherung
an den Wert $r[k+H_p|k]$ der Referenztrajektorie am Ende des Prädiktionshorizontes erzwungen sowie für den Zeitpunkt $k+H_p$ ein exaktes Erreichen dieses Wertes verlangt. Die
eingezeichnete Trajektorie $y^{(1)}[k+i|k]$ ist ein Beispiel für einen diesbezüglich zulässigen Verlauf. Stellt man zusätzlich die Monotonieforderung
\begin{align}
	\hat{y}[k+i+1|k] & \ge \hat{y}[k+i|k], \label{eqn:kap_4_monotonie_anf}
\end{align}
so darf die Regelgröße den Wert $r[k+H_p|k]$ nicht überschreiten. Der Verlauf $y^{(2)}[k+i|k]$ ist dann in diesem Sinne zulässig.
\end{exmp}
Es ist eine Übung unter \picref{sec:uebung_darstellung_guetefunktional}{Darstellung des Gütefunktionals und der Beschränkungen für die Steuer- und Regelgrößen} im Anhang zu finden.

\section{Darstellung der Steuerungsaufgabe als Optimierungsproblem und numerische Lösung}
\subsection{Aufgabe mit $\|\cdot\|_2$-Funktionial ohne Beschränukung der Steuer- und Regelgröße}
\label{subsec:aufgabe_quad_ohne_beschr}
Betrachtet werden Aufgaben, bei denen das Funktional \eqnref{eqn:kap_4_quad_guetefunktional} zu minimieren ist und als Nebenbedingung allein die Prozessgleichung
\eqnref{eqn:kap_4_zustandsraummodell_steuerdiff} auftritt.
Mit den Abkürzungen
\begin{align*}
	\mathcal{Q}_{y}[k] & :=\begin{bmatrix}
	Q_{y}[1]	& 			& 	\\
				& \ddots	&	\\
				&			& Q_{y}[H_p]
	\end{bmatrix},\ \mathcal{Q}_{\Delta\mathcal{U}}[k]:= \begin{bmatrix}
	Q_{\Delta\mathcal{U}}[0]	&			&	\\
								& \ddots	& 	\\
								&			& Q_{\Delta\mathcal{U}}[H_u-1]
	\end{bmatrix}
\end{align*}
und
\begin{align*}
	\mathcal{T}[k] & := \begin{bmatrix}
	r[k+1|k]\\
	\vdots\\
	r[k+H_p|k]
	\end{bmatrix}
\end{align*}
gilt
\begin{align}
	V(k) & = \left\| \mathcal{Q}_{y}\left(\mathcal{Z}[k]-\mathcal{T}[k]\right) \right\|_2^2 + \left\| \mathcal{Q}_{\Delta\mathcal{U}}\Delta\mathcal{U}[k] \right\|_2^2
\end{align}
Mit $\mathcal{E}[k]:=\mathcal{T}[k]-\Psi x[k]-\Upsilon u[k-1]$ wird die Differenz zwischen der Referenztrajektorie $\mathcal{T}[k]$ und der freien Bewegung $\Psi x[k]+\Upsilon u[k-1]$
des \ac{MPR}-Prozesses (nicht des Prozesses!) bezeichnet. Somit gilt
\begin{align}
\begin{split}\label{eqn:kap_4_quad_kostenfunktional_ohne_beschr}
	V[k] & = \left\| \mathcal{Q}_{y}\left(\Theta\Delta\mathcal{U}[k]-\mathcal{E}[k]\right) \right\|_2^2 + \left\| \mathcal{Q}_{\Delta\mathcal{U}}\Delta\mathcal{U}[k] \right\|_2^2\\
	& = \left\| \begin{bmatrix} \mathcal{Q}_y\left(\Theta\Delta\mathcal{U}[k] - \mathcal{E}[k]\right)\\ \mathcal{Q}_{\Delta\mathcal{U}\Delta\mathcal{U}[k]} \end{bmatrix}
	\right\|_2^2
\end{split}
\end{align}
und man erhält die Lösung $\Delta\mathcal{U}^{\ast}[k]$ der unbeschränkten Optimierungsaufgabe
\begin{align}
	\min\limits_{\Delta\mathcal{U}[k]}V[k]
\end{align}
als Quadratmittellösung eines überbestimmten linearen Gleichungssystems
\begin{align}
	\begin{bmatrix}
	\mathcal{Q}_y\Theta\\ \mathcal{Q}_{\Delta\mathcal{U}}
	\end{bmatrix}\Delta\mathcal{U}[k] & \cong \begin{bmatrix}
	\mathcal{Q}_y\mathcal{E}[k]\\ 0
	\end{bmatrix}. \label{eqn:kap_4_quadratmittelproblem}
\end{align}
Da $\mathcal{Q}_{\Delta\mathcal{U}}$ positiv definit ist, ist die Lösung eindeutig.
\begin{remark}
Die Matrix $A\in\mathbb{R}^{m\times n}$ ($m\ge n$) habe vollen Spaltenrang und es sei $b\in\mathbb{R}^m$. Dann ist im generischen Fall die
Lösungsmenge des Gleichungssystems $A x= b$ leer. Die Aufgabe der Bestimmung der Quadratmittellösung, i.Z.
\begin{align}
	A x & \cong b\quad \text{bzw.}\quad \left\| A x -b \right\|_2\rightarrow \min,
\end{align}
hat dann die eindeutige Lösung $x^{\ast}=A^+b$. Die Matrix $A^+:=\left(A^TA\right)^{-1}A^T$ heisst \textit{Pseudoinverse} zu $A$ und die Matrix $P:=A A^+$ heißt Projektor auf das Bild
$\Image A:=\left\{ A x : x\in\mathbb{R}^n \right\}$ von $A$. Die Berechnung von $x^{\ast}$ kann z.B. mit der Singulärwertzerlegung oder der
QR-Zerlegung von $A$ erfolgen. In \textsc{Matlab} verwendet man zur Bestimmung der Quadratmittellösung $x^{\ast}$ den Backslash-Operator: $x^{\ast}=A\backslash b$.
\end{remark} 
\begin{figure}[htb]
	\centering
	\begin{tikzpicture}[auto, >=latex'] 
	\node [input] () {};
\end{tikzpicture}

	\caption{Interpretation der Quadratmittellösung $A x\cong b$}
	\label{fig:kap_4_interp_quadratmittelloesung}
\end{figure}
\begin{remark}[Struktur des \ac{MPR}-Reglers]
Seien $A\in\mathbb{R}^{m\times n}, B=\begin{bmatrix} b_1 & \ldots & b_p \end{bmatrix}\in\mathbb{R}^{m\times p},\alpha_i\in\mathbb{R} (i=1,\ldots,p)$. Weiter sei $x$ Lösung von
\begin{align*}
	A x & \cong B\begin{bmatrix}
	\alpha_1\\ \vdots\\ \alpha_p
	\end{bmatrix}.
\end{align*}
Dann gilt
\begin{align}
	x & = A\backslash(B\begin{bmatrix}
	\alpha_1 \\ \vdots\\ \alpha_p
	\end{bmatrix})= \begin{bmatrix}
	A\backslash b_1 & \ldots & A\backslash b_p
	\end{bmatrix}\begin{bmatrix}
	\alpha_1 \\ \vdots\\ \alpha_p
	\end{bmatrix}.\label{eqn:kap_4_berechnung_x}
\end{align}
Aus \eqnref{eqn:kap_4_quadratmittelproblem} folgt damit
\begin{align*}
	\Delta\mathcal{U}^{\ast}[k] & = \begin{bmatrix}
	\Delta u^{\ast}[k|k]\\ \vdots\\ \Delta u^{\ast}[k+H_u-1|k]
	\end{bmatrix}=\bar{K}\mathcal{E}[k]\quad\text{mit }\bar{K}:=\begin{bmatrix}
	\mathcal{Q}_y\Theta\\ \mathcal{Q}_{\Delta\mathcal{U}}
	\end{bmatrix}\backslash\begin{bmatrix}
	\mathcal{Q}_y\\ 0
	\end{bmatrix}.
\end{align*}
Sei $K:=\bar{K}(1:n_{\mathcal{U}},:)$, wobei $n_{u}$ die Länge des Vektors $u[k]$ ist. Gemäß dem Schritt (S2) des Algorithmus der modellprädiktiven Regelung aus
\secref{subsec:grundprinzip_mpr} ist die im Schritt $k$ ausgegebene Steuergröße gleich
\begin{align*}
	\Delta u[k] & = \Delta u^{\ast}[k|k] = K\mathcal{E}[k]=K\left(\mathcal{T}[k]-\Psi x[k]-\Upsilon u[k-1]\right).
\end{align*}
Die Multiplikation der Signale $\mathcal{T}[k]$, $x[k]$ und $u[k-1]$ mit zeitunabhängigen Matrizen zur Ermittlung der optimalen Steuergröße $u^{\ast}[k]$ stellt ein lineares und
zeitinvariantes Regelgesetz dar. Die Struktur des realisierten Reglers ist in der \figref{fig:kap_4_struktur_mpr} dargestellt. Man stellt fest, dass durch die geschickte
Nutzung der Gleichung \eqnref{eqn:kap_4_berechnung_x} zur Berechnung der optimalen Steuergröße im jeweils aktuellen Schritt keine Online-Optimierung mehr erforderlich ist.
\end{remark}

\subsection{Aufgabe mit $\|\cdot\|_2$-Funktionial und Beschränkung}
\label{subsec:aufgabe_quad_mit_beschr}
Die in \secref{subsec:aufgabe_quad_ohne_beschr} betrachtete Aufgabe wird nun auf den Fall beschränkter Steuer- und Regelgrößen erweitert. Die Aufgabe besteht nun in der
Minimierung von 
\begin{align}
V[k] & = \left\| \begin{bmatrix}
	\mathcal{Q}_y \left(\Theta\Delta\mathcal{U}[k]-\mathcal{E}[k] \right)\\
	\mathcal{U}_{\Delta\mathcal{U}}\Delta\mathcal{U}[k]
\end{bmatrix} \right\|_2^2
\end{align}
\begin{figure}[htb]
	\centering
	\begin{tikzpicture}[auto, >=latex'] 
	\node [input] () {};
\end{tikzpicture}

	\caption{Struktur den \ac{MPR}-Reglers zur Minimierung von \eqnref{eqn:kap_4_quad_guetefunktional} bei unbeschränkten Steuer- und Regelgrößen}
	\label{fig:kap_4_struktur_mpr}
\end{figure}
unter der Nebenbedingung \eqnref{eqn:kap_4_lin_nebenbedingungen}, d.h.
\begin{align}
	M[k]\begin{bmatrix}
	\mathcal{Z}[k]\\ \Delta\mathcal{U}[k]\\ \mathcal{U}[k]\\ 1
	\end{bmatrix}\le 0. \label{eqn:kap_4_nebenbedinungen_beschr}
\end{align}
Mit $\mathcal{Z}[k]=\Psi x[k]+\Upsilon u[k-1]+\Theta\Delta\mathcal{U}[k]$ und
\begin{align*}
\mathcal{U}[k] & = \begin{bmatrix}
I & \ldots & 0\\
\vdots & \ddots & \cdots\\
I & \ldots & I
\end{bmatrix}\Delta\mathcal{U}[k]+\begin{bmatrix}
u[k-1]\\ \vdots\\ u[k-1]
\end{bmatrix}
\end{align*}
kann man \eqnref{eqn:kap_4_nebenbedinungen_beschr} in die Form
\begin{align}
	\Xi\Delta\mathcal{U}[k] & \le \xi\label{eqn:kap_4_lin_nebenbedinung_form}
\end{align}
mit einer Matrix $\Xi=\Xi[k]$ und einem Vektor $\xi = \xi[k]$ überführen. Weiter gilt füt die Zielfunktion
\begin{align*}
	V[k] & = \left( \Theta\Delta\mathcal{U}[k]-\mathcal{E}[k] \right)^T\mathcal{Q}_y^T\mathcal{Q}_y\left( \Theta\Delta\mathcal{U}[k]-\mathcal{E}[k]
	\right)+\Delta\mathcal{U}[k]^T\mathcal{Q}^T_{\Delta u}\mathcal{Q}_{\Delta u}\Delta\mathcal{U}[k]\\
	& = \Delta\mathcal{U}[k]^T H \Delta\mathcal{U}[k]-\Delta\mathcal{U}[k]^TG+F
\end{align*}
mit $H:=\Theta^T\mathcal{Q}_y^T\mathcal{Q}_y\Theta+\mathcal{Q}^T_{\Delta u}\mathcal{Q}_{\Delta u}$,
$G:=2\Theta^T\mathcal{Q}_y^T\mathcal{Q}_y\mathcal{E}[k]$ und $F:=\mathcal{E}[k]^T\mathcal{Q}_y^T\mathcal{Q}_y\mathcal{E}[k]$. Also ist $V[k]$ eine quadratische Zielfunktion und mit der linearen Nebenbedingung
\eqnref{eqn:kap_4_lin_nebenbedinung_form} ist
\begin{align}
	\min\limits_{\Delta\mathcal{U}[k]}V[k] & \text{ bei } \Xi\Delta\mathcal{U}[k]\le \xi\label{eqn:kap_4_quad_opt_aufgabe_lin_nebenbedingungen}
\end{align}
eine Aufgabe der Quadratischen Optimierung. Die Lösung kann im Allgemeinen nur durch numeirsche Verfahren bestimmt werden. Der zulässige Bereich der Aufgabe ist aufgrund der
ausschließlich linearen Nebenbedingungen konvex und es gibt weiter
\begin{align*}
\frac{\d^2 V[k]}{\d\Delta\mathcal{U}[k]^2} & = 2 H \ge 0.
\end{align*}
Damit ist \eqnref{eqn:kap_4_quad_opt_aufgabe_lin_nebenbedingungen} ein konvexes Optimeirungsproblem, d.h. falls eine Lösung dieser Aufgabe exisitert, ist jede lokale Minimalstelle auch
stets eine globale Minimalstelle. Darüber hinaus ist aufgrund von $\mathcal{Q}_{\Delta u}>0$ sogar $H>0$ erfüllt und daher ist die Zielfunktion $V[k]$
streng konvex, d.h. im Falle der Existenz ist die Lösung von \eqnref{eqn:kap_4_quad_opt_aufgabe_lin_nebenbedingungen} dann sogar eindeutig. Unter diesen Umständen kann die Lösung $\Delta\mathcal{U}^{\ast}[k]$
der Aufgabe \eqnref{eqn:kap_4_quad_opt_aufgabe_lin_nebenbedingungen} numerisch effizient mit Algorithmen der Quadratischen Optimierung bestimmt werden\footnote{In \textsc{Matlab} ist zur Lösung
der Aufgabe der Quadratischen Programmierung die Routine \lstinline[columns=fixed]{quadprog} implementiert, die speziell für den durch Ungleichungsbeschränkungen gekennzeichneten
Problemtyp \eqnref{eqn:kap_4_quad_opt_aufgabe_lin_nebenbedingungen} einer Aktiven-Mengen-Strategie nach \cite{Gill1981} verwendet. Der Aufruf erfolgt mit
$\Delta\mathcal{U}[k]=\text{\lstinline[columns=fixed]{quadprog}}(2H,G,\Xi,\xi)$.}.

\subsection{Aufgabe mit $\|.\|_1$- und $\|.\|_{\infty}$-Funktionial}
Bei Aufgaben mit dem Funktional \eqnref{eqn:kap_4_eins_norm_guetefunktional} oder \eqnref{eqn:kap_4_max_norm_guetefunktional_1} bzw. \eqnref{eqn:kap_4_max_norm_guetefunktional_2} muss
man nicht zwischen dem Fall unbeschränkter Steuer- und Regelgrößen und dem Fall beschränkter Steuer- und Regelgrößen unterscheiden, da sich in jedem Fall ein Optimierungsproblem mit
Beschränkungen ergibt. Formt man die Zielfunktion $V[k]$ analog dem \secref{subsec:aufgabe_quad_mit_beschr} um, so erhält man für das Gütefunktional
\eqnref{eqn:kap_4_eins_norm_guetefunktional}
\begin{align}
V[k] & = \left\| \begin{bmatrix}
\mathcal{Q}_y\left( \Theta\Delta\mathcal{U}[k]-\mathcal{E}[k] \right)\\ \mathcal{Q}_{\Delta\mathcal{U}}\Delta\mathcal{U}[k]
\end{bmatrix} \right\|_1 = \left\| \begin{bmatrix}
\mathcal{Q}_y\Theta\\ \mathcal{Q}_{\Delta\mathcal{U}}
\end{bmatrix}\Delta\mathcal{U}[k]-\begin{bmatrix}
\mathcal{Q}_y\mathcal{E}[k]\\ 0
\end{bmatrix} \right\|_1
\end{align}
und für die Gütefunktionale \eqnref{eqn:kap_4_max_norm_guetefunktional_1} und \eqnref{eqn:kap_4_max_norm_guetefunktional_2} erhält man
\begin{align}
	V[k] & = \left\| \mathcal{Q}_y\Theta\Delta\mathcal{U}[k]-\mathcal{Q}_y\mathcal{E}[k] \right\|_{\infty}
\end{align}
bzw.
\begin{align}
	V[k] & = \left\| \mathcal{Q}_{\Delta\mathcal{U}}\Delta\mathcal{U}[k] \right\|_{\infty}.
\end{align}
Die Minimierung von $V[k]$ unter der Nebenbedingung \eqnref{eqn:kap_4_lin_nebenbedinung_form} kann dann als Aufgabe der Linearen Optimierung dargestellt werden\footnote{Die numerische
Lösung von Aufgaben der Linearen Optimierung kann wie für Aufgaben der Quadratischen Optimierung durch eine Aktive-Mengen-Strategie erfolgen. In \textsc{Matlab} erfolgt die Lösung
von $\min\limits_{x\in\mathbb{R}^n}c^Tx$ bei $Ax\le b$ durch $x=\text{\lstinline[columns=fixed]{linprog}}(c,A,b)$}, denn mit Matrizen $F$ und $C$ sowie Vektoren $f$ und $d$ passender
Dimension kann die Aufgabe
\begin{align*}
	\min\limits_{x\in\mathbb{R}^n}\left\| F x-f \right\|_1
\end{align*}
durch
\begin{align*}
	\min\limits_{\substack{\mu\in\mathbb{R}^1\\\eta\in\mathbb{R}^m\\x\in\mathbb{R}^n }}\mu & \text{ bei } \mu\ge 0,\ \unity^T\eta\le \mu,\ -\eta\le Fx-f\le\eta,\ Cx\le d
\end{align*}
und die Aufgabe 
\begin{align*}
	\min\limits_{x\in\mathbb{R}^n}\left\| F x-f \right\|_{\infty}
\end{align*}
durch
\begin{align*}
	\min\limits_{\substack{\delta\in\mathbb{R}^1\\x\in\mathbb{R}^n }}\delta & \text{ bei } \delta\ge 0,\ -\unity^T\delta\le Fx-f\le\unity\delta,\ Cx\le d
\end{align*}
gelöst werden. Hierbei ist $\unity:=\begin{bmatrix}1 & \ldots & 1 \end{bmatrix}^T$.

\section{Stabilität prädiktiver Regelungen}
\subsection{Problembeschreibung}
\label{subsec:kap_4_problembeschreibung}
Der in \secref{subsec:grundprinzip_mpr} beschriebene prinzipielle Algorithmus der modellprädiktiven Regelung stellt eine Rückführung dar, womit
potenziell die Gefahr der Instabilität des entstehenden Regelkreises besteht. Zur Erläuterung wird als eine einfache Form des quadratischen Gütefunktionals das Funktional
\begin{align}
	V[k] & = \sum\limits_{i=1}^{H_p}\left\| \hat{x}[k+i|k] \right\|_2^2
\end{align}
betrachtet, wobei für die Länge des Steuer- und des Prädiktionshorizonts $H_u=H_p$ gelten soll. Beschränkungen hinsichtlich der Steuer- und Zustandsgröße sollen nicht vorliegen. Im
Zeitpunkt $k$ wird die optimale Steuerfolge $(\Delta u^{\ast}[k+i|k])_{i=0,\ldots,H_p-1}$ berechnet, die dann die optimale Zustandstrajektorie $(x^{\ast}[k+i|k])_{i=1,\ldots,H_p}$
liefert. Gemäß dem Algorithmus der modellprädiktiven Regelung wird dann $\Delta u[k]:=\Delta u^{\ast}[k|k]$ gesetzt und mit dieser Steuergröße ergibt
sich (unter der Annahme, dass keine Störgrößen oder Modellunbestimmtheiten auftreten) der Folgezustand $x[k+1]=x^{\ast}[k+1|k]$. Würde man dann im
Zeitpunkt $k+1$ das Gütefunktional
\begin{align}
	\tilde{V}[k] & = \sum\limits_{i=1}^{H_p-1}\left\| \hat{x}[k+1+i|k+1] \right\|_2^2
\end{align}
minimieren, so würde nach dem \textsc{Bellman}schen Optimalitätsprinzip für die optimale Zustandstrajektorie $\tilde{x}^{\ast}[k+1+i|k+1]=x^{\ast}[k+1+i|k]$ ($i=1,\ldots,H_p-1$) gelten.
Jedoch löst man im Zeitpunkt $k+1$ die Aufgabe mit dem Gütefunktional
\begin{align}
	V[k+1] & = \sum\limits_{i=1}^{H_p}\left\| \hat{x}[k+1+i|k+1] \right\|_2^2
\end{align}
und bestimmt damit den im Allgemeinen von $\tilde{x}^{\ast}[k+1+i|k+1]$ abweichenden Verlauf $x^{\ast}[k+1+i|k+1]$. Weiter gilt dann $x[k+2]=x^{\ast}[k+2|k+1]$. Setzt man diese
Überlegung fort, so erhält man eine Trajektorie $x[k+i]$ ($i=1,\ldots,H_p$) des Zustandes im \ac{MPR}-Regelkreis, die im Normalfall von der zum Zeitpunkt $k$ bestimmten optimalen
Prädiktion $x^{\ast}[k+i|k]$ abweichen wird (siehe \figref{fig:kap_4_abweichung_opt_tatsaechlich_verlauf}).

Hätte der Prädiktionshorizont eine unendliche Länge, so würde der tatsächliche Verlauf der Zustandstrajektorie $x[k+i]$ mit dem Verlauf der optimalen Prädiktion $x^{\ast}[k+i|k]$
übereinstimmen (falls keine Störgrößen wirken und keine Modellunbestimmtheiten auftreten). Bei einer endlichen Länge werden $x[k+i]$ und $x^{\ast}[k+i|k]$ jedoch im Allgemeinen
voneinander verschieden sein, so dass sich dann nicht nur die Güte des Regelkreises (gegenüber der aufgrund der Vorhersage $x^{\ast}[k+i|k]$ über dem Prädiktionshorizont erwarteten
Güte) verschlechtern wird, sondern sogar ein instabiler Regelkreis resultieren kann.
\begin{figure}[htb]
	\centering
	\begin{tikzpicture}[auto, >=latex'] 
	\node [input] () {};
\end{tikzpicture}

	\caption{Zur Abweichung zwischen der optimalen Trajektorie $x^{\ast}[k+i|k]$ und dem tatsächlichen Verlauf $x[k+i]$ ($i=1,\ldots,H_p$)}
	\label{fig:kap_4_abweichung_opt_tatsaechlich_verlauf}
\end{figure}
\begin{exmp}
Gegeben seien der stabiler Prozess
\begin{align*}
	x[k+1] & = \begin{bmatrix}
	0 & 0\\ 1 & 0
	\end{bmatrix}x[k]+\begin{bmatrix}
	1\\ 0
	\end{bmatrix}u[k]
\end{align*}
und das quadratische Gütefunktional
\begin{align*}
	V[k] & = \hat{x}[k+1|k]^T\begin{bmatrix}
	1 & 2\\ 2& 6
	\end{bmatrix}\hat{x}[k+1|k].
\end{align*}
Beschränkungen seien nicht vorhanden. Mit den Prozessgleichungen $\hat{x}_1[k+1|k]=\hat{u}[k|k]$ und $\hat{x}_2[k+1|k]=x_1[k|k]=x_1[k]$ erhält man
\begin{align*}
	V[k] & = \hat{x}_1[k+1|k]^2+6\hat{x}_2[k+1|k]^2+4\hat{x}_1[k+1|k]\hat{x}_2[k+1|k]\\
		 & = \hat{u}[k|k]^2+6x_1[k]^2+4\hat{u}[k|k]x_1[k]
\end{align*}
und weiter
\begin{align*}
\frac{\d V[k]}{\d\hat{u}[k|k]} & = 2\hat{u}[k|k]+4 x_1[k]=0\quad \Leftrightarrow\quad \underbrace{\hat{u}[k|k]}_{=:u^{\ast}[k]}=-2x_1[k].
\end{align*}
Der mit diesem (optimalen) Regelgesetz geschlossene Regelkreis wird durch die Differenzengleichung $x[k+1]=\begin{bmatrix}-2 & 0\\ 1 & 0 \end{bmatrix}x[k]$ beschrieben und ist
offensichtlich instabil.
\end{exmp}

\subsection{Stabilität des geschlossenen Kreises bei endlichem Optimierungshorizont durch eine Endbedingung für den Prozesszustand}
\label{subsec:kap_4_stab_geschl_rk}
Der folgende Satz liefert eine erste Aussage über die Stabilität einer modellprädiktiven Regelung.
\begin{satz}\label{satz:kap_4_stabilitaet}
Gegeben seien der Prozess $x[\kappa +1]=f(x[\kappa],u[\kappa])$ mit stetigem $f$ und einem stationären Punkt für $(x[\kappa],u[\kappa])=(0,0)$. Das Gütefunktional sei
\begin{align*}
	V[k] & = \sum\limits_{i=1}^{H_p}l\left( \hat{x}[k+i|k],\hat{u}[k+i-1|k] \right)
\end{align*}
mit stetigem $l$, $l(x,u)\ge 0$ und $l(x,u)=0$ genau dann, wenn $(x,u)=(0,0)$. Als Endbedingung sei $\hat{x}[k+H_p|k]=0$ gegeben und die
Beschränkungen seien $\hat{u}[k+i|k]\in U$ und $\hat{x}[k+i|k]\in X$ für gegebene Mengen $U$ und $X$. Für jedes $k$ existiert eine optimale Stellfolge $\left(u^{\ast}[k+1|k]\right)_{i=0,1,\ldots,H_p-1}$ und es sei $u^{\ast}[k|k]$
die im Schritt $k$ auf den Prozess aufgeschaltete Steuergröße.

Dann ist $(x,u)=(0,0)$ eine asymptotische stabile Gleichgewichtslage des geschlossenen Kreises.
\end{satz}
\begin{proof}
O.B.d.A. sei $f(x[k],0)\neq 0$ für $x[k]\neq 0$. Die Funktion
\begin{align*}
	\left( x[k],\hat{u}[k|k],\ldots,\hat{u}[k+H_p-1|k] \right) \mapsto V[k]
\end{align*}
ist stetig und damit ist der Minimalwert
\begin{align*}
	\mathcal{V}(x[k]) & := \min\limits_{\hat{u}[k|k],\ldots,\hat{u}[k+H_p-1|k]}V[k]
\end{align*}
stetig in $x[k]$. Wir zeigen jetzt, das $\mathcal{V}$ eine strikte Ljapunov-Funktion für $x[k+1]=f(x[k],u^{\ast}[k|k])$ ist.
\begin{enumerate}
  \item "`$\mathcal{V}$ ist positiv definit"':
  \begin{itemize}
    \item Aus $l\ge 0$ folgt die Ungleichung $\mathcal{V}(x)\ge 0$.
    \item Sei $x[k]=0$. Setze $u^{\ast}[k|k]:=0$, dann folgt $x^{\ast}[k+1|k]=0$, setze $u^{\ast}[k+1|k]:=0$, dann ist $x^{\ast}[k+2|k]=0$ usw. Somit gilt $\mathcal{V}(x[k])=0$.\\
    Umgekehrt sei $\mathcal{V}(x[k])=0$. Wegen $f(x[k],0)\neq 0$, $l\ge 0$ und $l=0 \Leftrightarrow (x,u)=(0,0)$ ist dann $x[k]=0$.\\
    Also gilt $\mathcal{V}(x)\Leftrightarrow x=0$.
  \end{itemize}
  \item "`$\mathcal{V}(x[k+1])<\mathcal{V}(x[k])$ für $x[k]\neq 0$"':
  \begin{align*}
  	\mathcal{V}(x[k+1]) & = \min\limits_{\hat{u}[k+1|k],\ldots,\hat{u}[k+H_p|k]}\sum\limits_{i=1}^{H_p}l\left( \hat{x}[k+i+1|k],\hat{u}[k+i|k] \right)\\
  	& = \min\limits_{\ldots}\left\{ \sum\limits_{i=1}^{H_p}l\left( \hat{x}[k+i|k],\hat{u}[k+i-1|k] \right) - l\left( \hat{x}[k+1|k],\hat{u}[k|k] \right) + \right. \\
  	& \qquad \qquad \qquad \left.\vphantom{\sum\limits_{i=1}^{H_p}} l\left( \hat{x}[k+H_p+1|k],\hat{u}[k+H_p|k] \right) \right\}\\
  	& \le \min\limits_{\hat{u}[k+1|k],\ldots,\hat{u}[k+H_p-1|k]}\left\{ \sum\limits_{i=1}^{H_p}l\left( \hat{x}[k+i|k],\hat{u}[k+i-1|k] \right) - \right.\\
  	& \qquad \qquad \qquad\qquad \qquad\qquad\quad \left.\vphantom{\sum\limits_{i=1}^{H_p}} l\left( \hat{x}[k+1|k],\hat{u}[k|k] \right) \right\}+\\
  	& \qquad \qquad \min\limits_{\hat{u}[k+H_p|k]}\sum\limits_{i=1}^{H_p}l\left( \hat{x}[k+H_p+1|k],\hat{u}[k+H_p|k] \right)\\
  	& = \underbrace{ \min\limits_{\hat{u}[k|k],\hat{u}[k+1|k],\ldots,\hat{u}[k+H_p-1|k]}V[k] }_{=\mathcal{V}(x[k])}-\underbrace{ l\left( \hat{x}[k+1|k],u^{\ast}[k|k] \right) }_{>0\text{
  	wegen }f(x[k],0)\neq 0}+\\
  	& \qquad\qquad \underbrace{ \min\limits_{\hat{u}[k+H_p|k]}\sum\limits_{i=1}^{H_p}l\left( \hat{x}[k+H_p+1|k],\hat{u}[k+H_p|k] \right) }_{=0}\\
  	& < \mathcal{V}(x[k]).
  \end{align*}
\end{enumerate}
\end{proof}
\begin{remark}
Für den \satzref{satz:kap_4_stabilitaet} gibt es allgemeinere Formulierungen. So kann dieser Satz z.B. auf die Endbedingung
$\hat{x}[k+H_p|k]=r[k+H_p|k]\neq 0$, d.h. auf den Fall beliebiger Referenztrajektorien, und auf die Bewertung der Steuergrößendifferenz $\Delta u$
(anstelle bzw. zusätzlich zur Steuergröße $u$) erweitert werden (siehe \cite{Maciejowski2002}).
\end{remark}
Die Voraussetzung der Existenz einer optimalen Steuerfolge mit der in $H_p$ Schritten der Zustand $x=0$ erreicht wird, ist eine sehr starke Voraussetzung, die z.B. bei einem kurzen
Horizont $H_p$ und bei Steuergrößenbeschränkungen häufig nicht erfüllt ist. Wie im nächsten Abschnitt gezeigt wird, kann diese Voraussetzung insbesondere für lineare Prozesse
abgeschwächt werden, indem der Endzustand lediglich geeignet gewichtet wird.

\subsection{Stabilität des geschlossenen Kreises bei endlichem Steuerhorizont durch eine Gewichtung des Endzustandes}
Wir betrachten zunächst die Aufgabe mit quadratischem Gütefunktional bei unendlichem Optimierungshorizont (d.h. einem unendlichen Steuer- und Prädiktionshorizont)
\begin{align}
	V[k] & = \sum\limits_{i=1}^{\infty}\left\| Q_x\hat{x}[k+i|u] \right\|_2^2 + \sum\limits_{i=0}^{\infty}\left\|Q_{\Delta u}[i]\Delta\hat{u}[k+i|k]
	\right\|_2^2\rightarrow \min .
\end{align}
Nach den Überlegungen in \secref{subsec:kap_4_problembeschreibung} gilt aufgrund des unendlichen Horizonts
\begin{align}
	x[k+i] & = x^{\ast}[k+i|k]\qquad (i=1,2,\ldots),\label{eqn:kap_4_x_normal_geschaetzt}
\end{align}
d.h. es kann aus dem Verlauf der optimalen Prädiktion auf das Verhalten (speziell auf die Stabilität) des \ac{MPR}-Regelkreises geschlossen werden, denn falls $V[k]$ einen endlichen
Wert aufweist, so impliziert dies $x[k+i]\rightarrow 0$ für $i\rightarrow\infty$.
\begin{figure}[htb]
	\centering
	\begin{tikzpicture}[auto, >=latex'] 
	\node [input] () {};
\end{tikzpicture}

	\caption{Optimale Trajektorie $x^{\ast}[k+i|k]$ und dem tatsächlicher Verlauf $x[k+i]$ bei unendlichem Horizont}
	\label{fig:kap_4_abweichung_opt_tatsaechlich_verlauf_unendl}
\end{figure}
Betrachtet man hingegen einen endlichen Steuerhorizont und einen unendlichen Prädiktionshorizont
\begin{align}
	V[k] & = \sum\limits_{i=1}^{\infty}\left\| Q_y\hat{y}[k+i|k] \right\|_2^2 + \sum\limits_{i=0}^{H_u-1}\left\| Q_{\Delta u}[i]\Delta
	\hat{u}[k+i|k]\right\|_2^2\rightarrow\min,
\end{align}
so gilt aufgrund von $H_u<\infty$ die Gleichung \eqnref{eqn:kap_4_x_normal_geschaetzt} im Allgemeinen nicht mehr. Im Weiteren wird vorausgesetzt, dass der Prozess stabil sei, das Paar
$(A,Q_yC_y)$ beobachtbar ist und für die Steuergröße am Ende des Steuerhorizontes $\hat{u}[k+H_u|k]=0$ gilt, was dann wegen \eqnref{eqn:kap_4_steuerhorizont} auch
$\hat{u}[k+H_u+i|k]=0\ \forall i\ge 0$ impliziert. Somit gelten
\begin{align*}
	\hat{y}[k+H_u+i|k] & =  C_yA^i\hat{x}[k+H_u|k]\qquad (i=1,2,\ldots)
\end{align*}
und
\begin{align*}
	\sum\limits_{i=H_u}^{\infty}\left\| Q_y\hat{y}[k+i|k]\right\|_2^2 & = \hat{x}[k+H_u|k]^T\underbrace{ \left(\sum\limits_{i=0}^{\infty}\left(A^T\right)^iC_y^TQ_y^TQ_yC_yA^i\right)
	}_{=:\bar{Q}}\hat{x}[k+H_u|k].
\end{align*}
Da die Matrix $A$ stabil ist, existiert die Matrix $\bar{Q}$ und kann als eindeutige Lösung der Ljapunov-Gleichung
\begin{align}
	A^T\bar{Q}A & = \bar{Q}-C_y^TQ_y^TQ_yC_y \label{eqn:kap_4_ljapunov_glg}
\end{align}
bestimmt werden. Da $(A,Q_yC_y)$ beobachtbar ist, ist $\bar{Q}$ positiv definit. Damit gilt
\begin{align}
\begin{split}\label{eqn:kap_4_guetefunktional_bedingung}
	V[k] & = \hat{x}[k+H_u|k]^T\bar{Q}\hat{x}[k+H_u|k]\\
	& \quad + \sum\limits_{i=1}^{H_u-1}\left\| Q_y\hat{y}[k+i|k]\right\|_2^2+\sum\limits_{i=0}^{H_u-1}\left\| Q_{\Delta u}[i]\Delta \hat{u}[k+i|k]\right\|_2^2 < +\infty  
\end{split}
\end{align}
also ist insbesondere auch
\begin{align}
	\mathcal{V}(x[k]) & := \min\limits_{\Delta\hat{u}[k|k],\ldots,\Delta\hat{u}[k+H_u-1|k]}V[k] < +\infty.
\end{align}
Man findet 
\begin{align*}
	\mathcal{V}(x[k+1]) & \le \mathcal{V}(x[k])-\left\| Q_y y^{\ast}[k+1|k]\right\|_2^2-\left\| Q_{\Delta u}\Delta u^{\ast}[k|k]\right\|_2^2,
\end{align*}
also gilt $\mathcal{V}(x[k+1])\le \mathcal{V}(x[k])$. Weiter wird in \cite{Maciejowski2002} gezeigt, dass es im Fall $x[k]\neq 0$ ein $i_0\in\left\{ 1,\ldots,n_x \right\}$ gibt, wobei
$n_x$ die Länge des Vektors $x[k]$ bezeichnet, so dass sogar $\mathcal{V}(x[k+i_0])<\mathcal{V}(x[k])$ erfüllt ist. Wegen $\mathcal{V}\ge 0$ und $\mathcal{V}$ stetig, ist $\mathcal{V}$ eine
Ljapunov-Funktion zur Gleichgewichtslage $x=0$. Und bei geeigneter Indizierung ist $\mathcal{V}$ sogar eine strikte Ljapunov-Funktion. Weiter folgt aus $\left\| x[k]
\right\|_2\rightarrow+\infty$ dann $\mathcal{V}(x[k])\rightarrow +\infty$. Also ist der geschlossene Regelkreis asymptotisch stabil!
\begin{remark}
Wegen \eqnref{eqn:kap_4_guetefunktional_bedingung} ergibt sich eine Aufgabe mit endlichem Prädiktionshorizont bzw. Steuerhorizont und einer Gewichtung des Endzustandes durch die
positiv definite Matrix $\bar{Q}$. Diese Gewichtungsmatrix kann gemäß \eqnref{eqn:kap_4_ljapunov_glg} durch das \textsc{Matlab}-Kommando
$\bar{Q}=\text{\lstinline[columns=fixed]{dlyap}}\left(A^T,C_y^TQ_y^TQ_yC_y\right)$ bestimmt werden. Die Gewichtung des Endzustandes ist praktisch eine wesentlich schwächere
Voraussetzung als das Einführen einer Endbedingung der Form $\hat{x}[k+H_u|k]=0$ wie in \secref{subsec:kap_4_stab_geschl_rk}.\\
Die beschriebene Vorgehensweise kann auf instabile Prozesse, auf von Null verschiedene Referenztrajektorien, auf beschränkte Steuer- und Regelgrößen sowie auf $\|.\|_1$- und
$\|.\|_{\infty}$-Gütefunktionale erweitert werden (siehe \cite{Maciejowski2002}).
\end{remark}
