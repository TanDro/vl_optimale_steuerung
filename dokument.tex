%% Dokumentenklasse (Koma Script) -----------------------------------------
\documentclass[%
   %final,      % fertiges Dokument
	 % --- Paper Settings ---
   paper=a4,%
   paper=portrait, % landscape
   pagesize, % driver
   % --- Base Font Size ---
   fontsize=13bp,%
	 % --- Koma Script Version ---
   version=last, %
 ]{scrreprt} % Classes: scrartcl, scrreprt, scrbook

%%%% Dokument/PDF Metadaten
\title{Vorlesung - Optimale Steuerung kontinuierlicher Prozesse}
\author{Jens Wurm}

% Encoding der Dateien (sonst funktionieren Umlaute nicht)
% Fuer Linux -> utf8
% Fuer Windows, alte Linux Distributionen -> latin1

% Empfohlen latin1, da einige Pakete mit utf8 Zeichen nicht
% funktionieren, z.B: listings, soul.
%\usepackage[latin1]{inputenc}
%\usepackage[ansinew]{inputenc}
\usepackage[utf8]{inputenc}
%\usepackage{ucs}
%\usepackage[utf8x]{inputenc}

%%% Preambel
\input{latex_base/thesis/preambel/settings}
\input{latex_base/thesis/preambel/preambel}
%
%%%% Neue Befehle
\input{latex_base/thesis/macros/newcommands}
\input{latex_base/general/macros/TableCommands}

%% Dokument Beginn
%% Gliederung:
%%		Deckblatt
%%		Inhaltsverzeichnis..................I
%%		Abkürzungsverzeichnis...............II
%%		Verwendete Formelm..................III
%%		Verwendete Indizies.................IV
%%		Abbildungsverzichnis................V
%%		(Tabellenverzeichnis................VI)
%%		1. Einleitung.......................1
%%		2. .................................3
%%		.
%%		.
%%		.
%%		Literatur- und Quellenverzeichnis...56
%%		Erklärung...........................60
%%		(Danksagung.........................61)
%%		Anhang..............................A-1

\begin{document}
% Deckblatt
% \subject{Diplomarbeit \\ Universität <einfügen>}
% \title{<Titel einfügen>}
% \author{<Autor einfügen>}
% \date{<Datum einfügen>}
% \maketitle

% \begin{titlepage}
% 	\mbox{}\vspace{5\baselineskip}\\
% 	\sffamily\huge
% 	\centering
% 	<Titel einfügen>
% 	\vspace{3\baselineskip}\\
% 	\rmfamily\Large
% 	Diplomarbeit \\ Universität <einfügen>
% 	\vspace{2\baselineskip}\\
% 	\rmfamily\Large
% 	<Autor einfügen>
% 	\vspace{1\baselineskip}\\
% 	<Datum einfügen>
% \end{titlepage}


\begin{titlepage}
	\sffamily\huge
	\centering
	Vorlesung
	\vspace{3\baselineskip}\\
	\rmfamily\huge\bfseries
	\centering
	Optimale Steuerung kontinuierlicher Prozesse
	\vspace{8\baselineskip}\\
	\rmfamily\small
	Dr.-Ing. Dipl.-Math. R. Bartholomäus\\
\end{titlepage}


%\thispagestyle{empty}
%\chapter*{Erklärung der Selbstständigkeit}
%\thispagestyle{empty}
%Hiermit versichere ich, die vorliegende Arbeit selbstständig verfasst und keine anderen als die angegebenen Quellen und Hilfsmittel benutzt sowie die Zitate deutlich kenntlich gemacht zu haben.
%\vspace{4\baselineskip}\\
%<Ort einfügen>, den <Datum einfügen> \hfill <Autor einfügen>
%\vspace{4\baselineskip}\\
%\clearpage
%\mbox{}\thispagestyle{empty}

\cleardoublepage
\frontmatter
\cleardoublepage
% Inhaltsverzeichnis in den PDF-Links eintragen
\tableofcontents
\cleardoublepage
% Abkuerzungsverzeichnis
\chapter*{Abkürzungsverzeichnis}
\addcontentsline{toc}{chapter}{Abkürzungsverzeichnis}
\begin{acronym}[LabVIEW] %<--in Klammern das laengste Wort
	\acro{NB}{Nebenbedingung}
	\acro{GNB}{Gleichungsnebenbedingungen}
	\acro{UNB}{Ungleichungsnebenbedingungen}
	\acro{KTB}{Kuhn-Tucker-Bedingungen}
	\acro{KTP}{Kuhn-Tucker-Punkt}
	\acro{LICQ}{linear independence constraint qualification}
	\acro{LQR}{linear-quadratischer Regler}
	\acro{AW}{Anfangswert}
	\acro{MPR}{Modellprädiktive Regelung}
	\acro{SISO}{Single Input Single Output}
	\acro{PN}{Proportionalnavigation}
	\acro{ARE}{algebraische Riccati-Gleichung}
	\acro{DARE}{diskrete algebraische Riccati-Gleichung}
	\acro{HJB}{Hamilton-Jacobi-Bellman}
	\acro{DP}{Dynamische Programmierung}
	\acro{AB}{Anfangsbedingung}
	\acro{EB}{Endbedingung}
	\acro{RB}{Randbedingung}
\end{acronym}
\cleardoublepage
% Verzeichnis der Formelzeichen
\addchap[Verzeichnis der Formelzeichen]{Verzeichnis der Formelzeichen}
\begin{acronym}[LabVIEW] %<--in Klammern das laengste Wort
	\acro{H}[\ensuremath{H}]{Hesse-Matrix}
	\acro{L}[\ensuremath{L(\dot,\dot)}]{Lagrange-Funktion}
	\acro{Ham}[\ensuremath{\Ham}]{Hamilton-Operator}
	\acro{nable}[\ensuremath{\nabla}]{Nabla-Operator (Gradient)}
	\acro{im}[\ensuremath{\im A}]{Bild der Matrix A}
	\acro{kern}[\ensuremath{\ker A}]{Kern der Matrix A}
\end{acronym}
%\cleardoublepage
% Verzeichnis der verwendeten Indizes
%\input{content/0_3_indizes}
%\cleardoublepage
% Verzeichnis der verwendeten Symbole
\chapter*{Symbolverzeichnis}
\addcontentsline{toc}{chapter}{Symbolverzeichnis}
\begin{acronym}[LabVIEW] %<--in Klammern das laengste Wort
	\acro{Notation}{\textbf{Bedeutung}}
	\acro{dotx}[\ensuremath{\dot{x},\ \ddot{x}\ bzw.\ x^{(k)}}]{erste, zweite bzw. k-te Ableitung}
	\acro{tx}[\ensuremath{\tilde{x}}]{Schätzvektor}
	\acro{hx}[\ensuremath{\hat{x}}]{Schätzfehlervektor}
	\acro{hx}[\ensuremath{e}]{Folgefehlervektor}
	\acro{Symbol}{\textbf{Bedeutung}}
	\acro{RN}[\ensuremath{\mathbb{R},\mathbb{N}}]{Menge der reellen bzw. natürlichen Zahlen}
	\acro{t}[\ensuremath{t}]{Zeit}
	\acro{x}[\ensuremath{\mathbold{x}}]{Zustandsvektor}
	\acro{z}[\ensuremath{\mathbold{z}}]{transformierte Zustandsvektor}
	\acro{u}{Systemeingang}
	\acro{v}{virtueller Eingang}
	\acro{faus}[\ensuremath{\mathbold{y}_f}]{flacher Ausgang}
	\acro{yhx}[\ensuremath{y = h(x)}]{Systemausgang, bzw. Regelgröße}
	\acro{f}[\ensuremath{f}]{System-Vektorfelder}
	\acro{A}[\ensuremath{A}]{Systemmatrix (lin. System)}
	\acro{n}[\ensuremath{n}]{Systemordnung}
	\acro{r}[\ensuremath{r}]{relatriver Grad}
	\acro{s}[\ensuremath{s}]{Gleitfläche}
	\acro{tau}[\ensuremath{\vect{\tau}}]{Tangentialvektor}
	\acro{nu}[\ensuremath{\vect{\nu}}]{Normalenvektor}
\end{acronym}
\cleardoublepage
% Abbildungs- und Tabellenverzeichnis
\listoffigures
\cleardoublepage
\listoftables
\cleardoublepage

% Hauptteil
\mainmatter
\chapter{Endlichdimensionale Optimierungsprobleme}
\begin{figure}[htb]
	\centering
	\begin{tikzpicture}[auto, scale=0.65, every node/.style={scale=0.65}, node distance=2.0cm,>=latex']
	\node [input, node distance=2.5cm] (input) {};
	\node [block, node distance=2.5cm, right of=input] (prozess) {$P$};
	\node [output, node distance=2.5cm, right of=prozess] (output) {};

	\draw [->] (input) -- node {$u\in U$} (prozess);
	\draw [->] (prozess) -- node {$y\in Y$}  (output);
		
\end{tikzpicture}

	\caption{Steuerung}
	\label{fig:kap_1_steuerung}
\end{figure}
Die Eingänge $u$ können in ihren möglichen Werten auf eine Menge möglicher Werte $U$ beschränkt werden.
\begin{align*}
K(u,y) & \rightarrow \min\limits_u\rightsquigarrow u^{\star}\\
K & \ldots \text{Kostenfunktion}\\
u^{\star} & \ldots \text{Steuerfunktion}
\end{align*}
Optimierung bzgl. eines zeitlich beschränkten Zeitraumes:\\
Möglichkeit
\begin{itemize}
  \item Ausdehnung auf unendliche Zeit
  \item Verschiebung des Intervalls $[t_0,t_e]$ nach jedem Zeitpunkt
\end{itemize}
um einen Übergang zu einer kontinuierlichen Regelung zu ermöglichen.
\begin{figure}[htb]
	\centering
	\begin{tikzpicture}[auto, scale=0.65, every node/.style={scale=0.65}, node distance=2.0cm,>=latex']
	\draw[->] (0,0) -- (6,0) node[anchor=north] {$t$};
	\draw (0,0) node[anchor=north] {$t_a$}
		  (3,0) node[anchor=north] {$t_e$};
	\draw (0, 0.1) -- (0,-0.1);
	\draw (3, 0.1) -- (3,-0.1);
	
	\draw (12,0) node{endlicher Optimierungshorizont}
		  (12,-2) node{mitführen des Optimierungshorizontes};
		  
	\draw (0,-1) -- (3,-1);
	\draw (0, -0.9) -- (0,-1.1);
	\draw (3, -0.9) -- (3,-1.1);
	
	\draw (1,-2) -- (4,-2);
	\draw (1, -1.9) -- (1,-2.1);
	\draw (4, -1.9) -- (4,-2.1);

	\draw (2,-3) -- (5,-3);
	\draw (2, -2.9) -- (2,-3.1);
	\draw (5, -2.9) -- (5,-3.1);
\end{tikzpicture}

	\caption{Optimierungshorizont}
	\label{fig:kap_1_horizont}
\end{figure}
\section{Wiederholung aus der Analysis: Optimierung unter Nebenbedingungen}
\begin{defi}
Sei $F:D\subset\mathbb{R}^n\rightarrow\mathbb{R}$. Eine Stelle $x_0\in D$ heißt
\begin{itemize}
  \item globale Minimumstelle von $f$ auf $D$, wenn $f(x_0)\le f(x)\forall x\in D$
  \item lokale Minimumstelle von $f$ auf $D$, wenn Umgebung $U_{\epsilon}(x_0):=\left\{y\in\mathbb{R}^n|\ \|y-y_0\|<\epsilon\right\}$ von $x_0$ existiert, so dass $f(x_0)\le
  f(x)\forall x\in D\cap U_{\epsilon}(x_0)$
  \item isolierte lokale Minimumstelle von $f$ auf $D$, wenn Umgebung $U_{\epsilon}(x_0)$ existiert, so dass $f(x_0)<f(x)\forall x\in D\cap U_{\epsilon}(x_0), x\neq
  x_0$
  \item Analog Maximum mit $\ge$ statt $\le$ und $>$ statt $<$.
  \item Minimumstellen und Maximumstellen treten gemeinsam auf: Extremstellen
\end{itemize}
\end{defi}
\begin{exmp}\hspace{1cm}\\
\begin{minipage}[c][][c]{0.3\textwidth}
\centering
\begin{tikzpicture}[auto, scale=0.65, every node/.style={scale=0.65}, node distance=2.0cm,>=latex']
	\draw[->] (0,0) -- (6,0);
	\draw	(1,0) node[anchor=north] {$a$}
			(2,0) node[anchor=north] {$x_1$}
			(3,0) node[anchor=north] {$x_2$}
			(4,0) node[anchor=north] {$x_3$}
			(5,0) node[anchor=north] {$b$};

	\draw[->] (0,0) -- (0,4);
	
	\draw[dotted] (1,0) -- (1,4);
	\draw[dotted] (2,0) -- (2,4);
	\draw[dotted] (3,0) -- (3,4);
	\draw[dotted] (4,0) -- (4,4);
	\draw[dotted] (5,0) -- (5,4);
	
	\draw [black] plot [smooth] coordinates {(1,1) (2,3.5) (3,0.75) (4,3.45) (4.5,3.75) (5,4)};
\end{tikzpicture}

\end{minipage}
\hfill
\begin{minipage}[c][][c]{0.7\textwidth}
\begin{align*}
f &: [a,b]\subset\mathbb{R}\rightarrow \mathbb{R}\\
a &: \text{isolierte lokale Minimumstelle}\\
x_1 &: \text{isolierte lokale Maximumstelle}\\
x_2 &: \text{globale und isolierte lokale Minimumstelle}\\
x_3 &: \text{lokale Minimum- und Maximumstelle (nicht isoliert)}
\end{align*}
\end{minipage}
\end{exmp}
\begin{satz}\label{satz:1}
Sei $f:D\in\mathbb{R}^n\rightarrow\mathbb{R}, x_0\in\Inter D$\footnote{$\Inter\ldots $ interior (Mengenoperator)} und $\nabla f(x_0)$ vorhanden. Dann gilt: $x_0$ ist eine lokale
Extremstelle $\rightarrow \nabla f(x_0)=0$.
\end{satz}
\begin{defi}
 Eine Stelle $x_0$ mit $\nabla f(x_0)=0$ heißt kritische Stelle (stationärer Punkt) von $f$.
\end{defi}
\begin{satz}\label{satz:2}
Sei $D\subset\mathbb{R}^n$ offen, $f\in C^2(D,\mathbb{R}),x_0\in D$ und $\nabla f(x_0)=0$. Dann gilt 
\begin{align*}
H f(x_0)=\left\{\begin{tabular}{c}pos. definit\\neg. definit\\ indefinit \end{tabular} \right\}\Rightarrow f\text{ ist in }x_0\left\{\begin{tabular}{c}isol. lokale Minimumstelle\\
isol. lokale Maximumstelle\\
keine lokale Extremstelle \end{tabular} \right\}
\end{align*}
\end{satz}
Es ist eine Übung unter \picref{sec:uebung_kapitel_1_def_extrem}{Definitheit und Extremstellen} im Anhang zu finden.
\begin{defi}
Sei $D\subset\mathbb{R}^n$ offen, $f:D\rightarrow\mathbb{R},h_1,\ldots,h_m:D\rightarrow\mathbb{R}$. $\overline{x}\in D$ heisst lokale oder globale Extremstelle von $f$ unter den
\ac{GNB} $h_1(x)=0,\ldots,h_m(x)=0$, wenn $\overline{x}$ lokale oder globale Extremstelle von $f$ auf
\begin{align*}
M &:=\left\{x\in D|\ h_1(x)=\ldots=h_m(x)=0\right\}
\end{align*}
ist.
\end{defi}
\begin{satz}\label{satz:3}
(Lagrange Multiplikatorregel) Seien $f,h_1,\ldots,h_m$ stetig differenzierbar, $\overline{x}\in M$ und $\nabla h_1(\overline{x}),\ldots,\nabla h_m(\overline{x})$ linear unabhängig, dann
ist $\overline{x}$ eine lokale Extremstelle von $f$ unter den \ac{GNB}. So gibt es Zahlen $\lambda_1,\ldots,\lambda_m\in\mathbb{R}$ (Lagrangesche Multiplikatoren), so dass
\begin{align*}
\nabla f(\overline{x}) & = \sum\limits_{i=1}^m\lambda_i\nabla h_i(\overline{x})
\end{align*}
gilt.\footnote{Hinweis: Dimension von $h$ ist $n$ für linear unabhängige $h_1,\ldots,h_m$. Für $h_1,\ldots,h_m$
muss gelten $m\le n$. Meist ist sogar $m<n$, da sich sonst alle $h_1,\ldots,h_m$ in genau einem Punkt schneiden würden. Eine Optimierung wäre dann sinnlos.}
\end{satz}
\textbf{Technik zum Auflisten von Extremstellen}:
\begin{enumerate}
  \item Definiere Langrange-Funktion $F(x,\lambda_1,\ldots,\lambda_m):=f(x)+\sum\limits_{i=1}^m\lambda_i\nabla h_i(x)$.
  \item Kanidaten für lokale Extremstelle unter den Lösungen von\\ $F'(x,\lambda_1,\ldots,\lambda_m)=0$ suchen:
	\begin{align*}
	\nabla F(x,\lambda_1,\ldots,\lambda_m)=\begin{bmatrix}
	\nabla f(x)+\sum\lambda\nabla h_i(x)\\
	h_1(x)\\
	\vdots\\
	h_m(x)
	\end{bmatrix} = 0\in\mathbb{R}^{n+m}.
	\end{align*}
	D.h. durch Einführung der Lagrange-Funktion kann man die Extremstellenbedingung für unbeschränkte Probleme verwenden. 
\end{enumerate}
Geometrische Interpretation im Fall $n=2$, $m=1$:
\begin{itemize}
  \item[] Der Tangentialraum $T_{\overline{x}}M=\left\{v\in\mathbb{R}^n|\ \nabla h_i^T(\overline{x})v=0\ (i=1,\ldots,m) \right\} $ in $\overline{x}$ an die Menge $M$ ist gleich dem
  Tangentialraum $T_{\overline{x}}N=\left\{ v\in\mathbb{R}^n|\ \nabla f(\overline{x})^Tv=0 \right\}$ in $\overline{x}$ an die Niveaumenge $N :=\left\{x\in D\subset\mathbb{R}^n|\
  f(x)=f(\overline{x})\right\}$
	\begin{figure}[htb]
	\centering
		\begin{tikzpicture}[auto, >=latex'] 
	
	\draw (-0.5,1) node[anchor=south] (dh1) {$\nabla h_1(\overline{x})$};
	\draw (0.5,-1) node[anchor=north] (df) {$\nabla f(\overline{x})$};
	\draw (1,0.5) node[anchor=west] (tang) {$\overline{x}+T_{\overline{x}}M=\overline{x}+T_{\overline{x}}N$};
	\draw (0,0) node[branch] (dx) {};
	\draw (0.1,0) node[anchor=west] (x) {$\overline{x}$};
	\draw (0.5,1) node[anchor=south] (M) {$M$};
	
	\draw [dashed] (-1,-0.5) -- (1,0.5);
	\draw [->] (0,0) -- (-0.5,1);
	\draw [->] (0,0) -- (0.5,-1);
	
	\draw[line width=0.6pt] (-1,0.25) .. controls (0,-0.25) .. (0.5,1);
	
	\draw[line width=0.2pt] (-1,-0.6) .. controls (0,0.1) .. (1,0.4);
	\draw[line width=0.2pt] (-1,-0.1) .. controls (0,0.6) .. (1,0.9);
	\draw[line width=0.2pt] (-1,-1) .. controls (0,-0.3) .. (1,0);
	
\end{tikzpicture}

		\caption{Auffinden von Extremstellen: Geometrische Interpretation im Fall $n=2$, $m=1$}
		\label{fig:kap_1_tangraum}
	\end{figure}
\end{itemize}

\begin{enumerate}[label=\arabic*)]
  \item $G\subset\mathbb{R}^n$: $\Inter G:=\left\{x\in\mathbb{R}^n|\ \exists\epsilon>0:U_{\epsilon}(x)\subset G \right\}$: Menge aller inneren Punkte von $G$. Wenn $G$ offen:
  $\Leftrightarrow G=\Inter G$
 	\begin{figure}[htb]
	\centering
		\begin{tikzpicture}[auto, >=latex'] 
	\draw plot[smooth cycle] coordinates{(0,0) (1,0) (2,1) (1,2)};
	
	\draw (1.5,1.5) node[anchor=north] {$G$};
	\draw (2,1) node[branch] (online) {};
	\draw (1.8,0.1) node[branch] (outline) {};
	\draw (1,1) node[branch] {};
	
	\draw (3,1) node[anchor=west] (txt) {nicht zu $\Inter G$};
	
	\draw [->] (txt) -- (online);
	\draw [->] (txt) -- (outline);
	
\end{tikzpicture}

		\caption{Visualisierung von $\Inter G$}
		\label{fig:kap_1_inter_g}
	\end{figure}
  \item $G$ offen. $C^k(G):=$ Menge aller $k$-mal stetig differenzierbaren Funktionen auf $G$. Auch: $C^k(G,\mathbb{R}^m)$ zur Angabe des Bildraums.
  \item $D\subset\mathbb{R}^n$ offen, $f\in C^2(D,\mathbb{R})\Rightarrow H f(x)=\begin{bmatrix}
  f_{11}(x) & \ldots & f_{1n}(x) \\
  \vdots 	&		 & \vdots\\
  f_{n1}(x) & \ldots & f_{nn}(x)
  \end{bmatrix}$ symmetrisch\footnote{Andere Schreibweise: $\nabla^2 f$, $\frac{\partial^2 f}{\partial x^2}$, $\nabla_{xx}f$}\\
  Sei $A$ reell und symmetrisch, d.h. $A=A^T$\\
  Dann $A$ possitiv definit, i.Z. $A>0:\Leftrightarrow x^TAx>0\forall x\neq 0\Leftrightarrow \lambda_i(A)>0\forall i$ und $A$ positiv demidefinit, i.Z. $A\ge 0:\Leftrightarrow x^TAx\ge
  0\forall x\Leftrightarrow \lambda_i(A)\ge 0\forall i$.\\
  Negativ definit bzw. semidefinit analog.\\
  $A$ indefinit: $\Leftrightarrow x^TAx$ kann sowohl positive als auch negative Werte annehmen. $\Leftrightarrow$ Es gibt positive und negative Eigenwerte von $A$.
  \item $M\subset \mathbb{R}^n$.\\
		$\bd M = \cl M\backslash\Inter M$: Rand von $M$\\
		$\cl M =\left\{x\in\mathbb{R}^n|\ \exists(x_n)_{n\in N}\subset M:x_n\rightarrow x \right\}$: Abschluss der Menge $M$ (auch: Menge der Berührungspunkte)\\
		Beispiel: \begin{tabular}[t]{lr}
			$M=[0,1)$ & $\cl M=[0,1]$\\
					  & $\Inter M = (0,1)$\\
					  & $\bd M =\{0,1\}$
		\end{tabular}
  \item \begin{tabular}[t]{rl}
  $\left\{v_1,\ldots,v_n \right\}$ linear unabhängig &:$\Leftrightarrow [\lambda_1 v_1+\ldots+\lambda_n v_n=0\Rightarrow \lambda_1=\ldots=\lambda_n=0]$\\
  $\left\{v \right\}$ linear unabhängig &:$\Leftrightarrow [\lambda v=0\Rightarrow \lambda = 0]\Leftrightarrow v\neq 0$
  \end{tabular}
\end{enumerate}
\begin{exmp}
Gesucht sind die lokalen Extremstelle von 
\begin{align*}
	f(x,y) & = 4x^2-3xy \text{ auf } K=\left\{(x,y)^T\in\mathbb{R}|\ x^2+y^2\le 1 \right\}.
\end{align*}
Nach Satz \ref{satz:1} findet man 
\begin{align*}
	\begin{bmatrix}
	0\\0
	\end{bmatrix} & = \nabla f = \begin{bmatrix}
	8x-3y\\-3x
	\end{bmatrix}\\
	\begin{bmatrix}
	8x-3y\\-3x
	\end{bmatrix}\Leftrightarrow \begin{bmatrix}
	x\\y
	\end{bmatrix} & = \begin{bmatrix}
	0\\0
	\end{bmatrix}
\end{align*}
kritische Stelle von $f$ im Inneren von $K$, jedoch
\begin{align*}
		H f(x,y) & = \begin{bmatrix}
		8 & -3\\ -3 & 0
		\end{bmatrix} \text{ mit } \det\left(H f(0,0)\right) = -9 < 0
\end{align*}
Determinante ist Produkt der Eigenwerte $\rightarrow \lambda_1>0$, $\lambda_2<0\rightarrow$ es negative Eigenwerte. $H f(0,0)$ ist indefinit, d.h. $(0;0)$ keine lokalen Extremstellen
existieren, nach Satz \ref{satz:2}.\\ 
Andererseits ist $f$ stetig auf der kompakten Menge $K$, nach Satz von Weierstrass besitzt $f$ auf dieser Menge Minimum und Maximum. Diese müssen
auf dem Rand liegen.
\begin{align*}
	\bd K  & = \left\{(x,y)^T|\ h(x,y):=x^2+y^2-1=0\right\}\\
	\Inter K &=\left\{(x,y)^T\in\mathbb{R}^2|\ x^2+y^2<1 \right\}\\
	\cl(\Inter K) & = K
\end{align*}
Da $\bd K$\footnote{$\bd K\ldots$ Rand von $K$} keine offene Menge ist, ist Satz \ref{satz:1} nicht anwendbar. Mit
\begin{align*}
M & = \bd K \text{ ist } \nabla h(x,y) = 2\begin{bmatrix}
x\\y
\end{bmatrix} \neq 0\ \forall (x,y)^T\in M 
\end{align*}
linear unabhängig. Nach Satz \ref{satz:3} existiert $\lambda\in\mathbb{R}$ mit\\
\begin{minipage}{\dimexpr.5\linewidth-1em\relax} 
  \begin{tabular}{ll}
  (1) & 	$8x-3y=\lambda\cdot 2x$ \\
  (2) & 	$-3x=\lambda\cdot 2y$ 
  \end{tabular} 
\end{minipage}% 
\begin{minipage}[0pt]{2em} 
  $\left.\mbox{\rule{0pt}{\baselineskip}}\right\}$ 
\end{minipage}% 
\begin{minipage}{\dimexpr.5\linewidth-1em\relax} 
$\Delta F(x,y) = H f(x,y)$
\end{minipage}
\begin{minipage}{\dimexpr.5\linewidth-1em\relax} 
  \begin{tabular}{ll}
  (3) & 	$x^2+y^2-1=0$ 
  \end{tabular} 
\end{minipage}% 
\begin{minipage}[0pt]{2em} 
  $\mbox{\rule{0pt}{0.35\baselineskip}}$ 
\end{minipage}% 
\begin{minipage}{\dimexpr.5\linewidth-1em\relax} 
Erfüllung der \ac{NB}en
\end{minipage}
Lösung:
\begin{alignat*}{8}
	x_{1/2} & = \pm\frac{1}{\sqrt{10}}, & \quad & y_{1/2} & = \pm\frac{3}{\sqrt{10}}, & \quad & \lambda_{1/2}=-\frac12, & \quad & F(x_{1/2},y_{1/2}) & = -\frac12\\
	x_{3/4} & = \mp\frac{3}{\sqrt{10}}, & \quad & y_{3/4} & = \pm\frac{1}{\sqrt{10}}, & \quad & \lambda_{3/4}=\frac92, & \quad & F(x_{3/4},y_{3/4}) & = \frac92
\end{alignat*}
Technik:
\begin{align*}
	F(x,y,\lambda) & = f(x,y) +\lambda h(x,y) \Rightarrow \begin{bmatrix}
	F_x\\ F_y \\ F_{\lambda}
	\end{bmatrix} = 0
\end{align*}
Abbildung \ref{fig:kap_1_beispiel_extremstelle} stellt die Niveaulinien der Ausgangsfunktion und der Berechnungen dar. 
\begin{figure}[htb]
	\centering
	\begin{tikzpicture}[auto, scale=2.0, >=latex']
	\draw[->] (-2,0) -- (2,0) node[right] {$x$};
	\draw[->] (0,-2) -- (0,2) node[above] {$y$};
	
   	\draw (-2,0) node[anchor=north] {-2};
   	\draw (-1,0) node[anchor=north] {1};
   	\draw (0,0) node[anchor=north] {0};
   	\draw (1,0) node[anchor=north] {1};
   	\draw (2,0) node[anchor=north] {2};

   	\draw (0,-2) node[anchor=east] {-2};
   	\draw (0,-1) node[anchor=east] {1};
   	\draw (0,1) node[anchor=east] {1};
   	\draw (0,2) node[anchor=east] {2};
   	
   	\foreach \x in {-2,-1.5,...,1.5}
     	\draw (\x,0.02) -- (\x,-0.02);
   	
   	\foreach \y in {-2,-1.5,...,1.5}
     	\draw (0.02,\y) -- (-0.02,\y);

	\draw[thick,color=black] plot[samples=200, domain=-1.5:1.5] (\x,{4/3*\x});

	\draw[thick,color=yellow] plot[samples=100, domain=-1.7:-0.22] (\x,{4/3*\x-1/(2*\x)});
	\draw[thick,color=yellow] plot[samples=100, domain=0.22:1.7] (\x,{4/3*\x-1/(2*\x)});
	
	\draw[thick,color=green] plot[samples=100, domain=-1.4:-0.09] (\x,{4/3*\x+1/(6*\x)});
	\draw[thick,color=green] plot[samples=100, domain=0.09:1.4] (\x,{4/3*\x+1/(6*\x)});
	
	\draw[thick,color=red] plot[samples=100, domain=-2:-0.55] (\x,{4/3*\x-9/(6*\x)});
	\draw[thick,color=red] plot[samples=100, domain=0.55:2] (\x,{4/3*\x-9/(6*\x)});
	
	\draw[thick,color=blue] (0,0) circle (1);
\end{tikzpicture}

	\caption{Niveaulinien der Funktion $f$: $f=-\frac12$ (Grün), $f=0$ (Schwarz), $f=1$ (Gelb), $f=\frac92$ (Rot), $x^2+y^2=1$ (Blau)}
	\label{fig:kap_1_beispiel_extremstelle}
\end{figure}
\end{exmp}

\begin{defi}
Sei $D\in\mathbb{R}^n$ offen, $f:D\rightarrow\mathbb{R},g_1,\ldots,g_m:D\rightarrow\mathbb{R},\overline{x}\in D$ heisst (lokale bzw. globale) Extremstelle von $f$ unter der \ac{UNB}
$g_1(x)\le 0,\ldots,g_m(x)\le 0$, wenn $\overline{x}$ (lokale oder globale) Extremstelle von $f$ auf $G:=\left\{x\in D|\ g_1(x)\le 0,\ldots,g_m(x)\le 0\right\}$.
\end{defi}
\begin{defi}
$I_0(x):=\left\{ i\in\left\{ 1,\ldots,m \right\}|\ g_i(x)=0 \right\}$: Indexmenge der für $x\in G$ aktiven Restriktionen (für $x\in\int G$ gilt $I_0(x)\in\emptyset$)
\end{defi}
$L(x,u)=f(x)+\sum\limits_{i=1}^{m}u_ig_i(x)$: Lagrange-Funktion zur Aufgabe $\min\limits_{x\in G}f(x)$
\begin{satz}\label{satz:4}
	(Kuhn-Tucker Bedinungen 1. Ordnung):\\
	Sei $f,g_1,\ldots,g_m$ stetig differenzierbar, $\overline{x}\in G$ und $\nabla g_i(x)$ mit $i\in I_0(\overline{x})$ linear unabhängig. Ist $\overline{x}$ lokale Minimumstelle von $f$
	unter \ac{UNB}, so gibt es $a_1,\ldots,a_m\in\mathbb{R}$ (Lagrange-Multiplikatoren) so, dass gilt
	\begin{align*}
	\nabla_x L(\overline{x},u) & = \nabla f(\overline{x})+\sum\limits_{i=1}^m u_i\nabla g_i(\overline{x})= 0\quad\text{mit } \begin{bmatrix}
	g_i(\overline{x})\le 0\\ u_i\ge 0\\ u_ig_i(\overline{x})=0
	\end{bmatrix},\ i=1,\ldots,m
	\end{align*}
\end{satz}
\textbf{Geometrische Interpretation der Kuhn-Tucker-Bedingungen}\\

\begin{figure}[!htb]
	\centering
	\begin{tikzpicture}[auto, >=latex'] 
	\draw (0,-0.25) .. controls (1,-0.5) .. (3,0.5);
	\draw (-0.5,-2.25) .. controls (1,-2.5) and (2.5,-2) .. (4,-0.5);
	\draw (2,0.25) .. controls (3,0) and (3.5,-0.25) .. (3.75,-1.25);
	
	\draw (0,-0.25) node[anchor=east] {$g_3=0$};
	\draw (-0.5,-2.25) node[anchor=east] {$g_1=0$};
	\draw (2,0.25) node[anchor=east] {$g_2=0$};
	\draw (-0.25,-1.25) node[anchor=east] {$G$};
	
	\draw (1,-1.3) circle (0.4);
	\draw (1,-1.3) circle (0.8);
	\draw (1,-1.3) circle (1.2);
	\draw (1,-1.3) circle (0.05);
	\draw (1,-1.3) node[anchor=west] {$\overline{x}$};

	\draw (0.8,-3) node[anchor=west] {$f= const.$};
	
\end{tikzpicture}

	\caption{Optimalstelle mit $I_0(\overline{x})=\emptyset$}
	\label{fig:kap_1_ktb_opt_0}
\end{figure}

\begin{figure}[!htb]
	\centering
	\begin{tikzpicture}[auto, >=latex'] 
	\draw (0,-0.25) .. controls (1,-0.5) .. (3,0.5);
	\draw (-0.5,-2.25) .. controls (1,-2.5) and (2.5,-2) .. (4,-0.5);
	\draw (2,0.25) .. controls (3,0) and (3.5,-0.25) .. (3.75,-1.25);
	
	\draw (0,-0.25) node[anchor=east] {$g_3=0$};
	\draw (-0.5,-2.25) node[anchor=east] {$g_1=0$};
	\draw (2,0.25) node[anchor=east] {$g_2=0$};
	\draw (-0.25,-1.25) node[anchor=east] {$G$};

	\draw[->] (0.9,-2.25) -- (1.1,-3.45);
	\draw[->] (0.9,-2.25) -- (1,-2.85);
	
	\draw (1.1,-3.45) node[anchor=north] {$\nabla g_1(\overline{x})$};
	\draw (1,-2.85) node[anchor=west] {$-\nabla f(\overline{x})$};

	\draw (1.1,-3.45) circle (0.4);
	\draw (1.1,-3.45) circle (0.8);
	\draw (1.1,-3.45) circle (1.2);
	\draw (0.9,-2.25) circle (0.05);
	\draw (0.9,-2.25) node[anchor=south] {$\overline{x}$};
	
	\draw (2.2,-3.55) node[anchor=west] {$f= const.$};
	
\end{tikzpicture}

	\caption{Optimalstelle mit $I_0(\overline{x})=\{1\}$}
	\label{fig:kap_1_ktb_opt_1}
\end{figure}

\begin{figure}[!htb]
	\centering
	\begin{tikzpicture}[auto, >=latex'] 
	\draw (0,-0.25) .. controls (1,-0.5) .. (3,0.5);
	\draw (-0.5,-2.25) .. controls (1,-2.5) and (2.5,-2) .. (4,-0.5);
	\draw (2,0.25) .. controls (3,0) and (3.5,-0.25) .. (3.75,-1.25);
	
	\draw (0,-0.25) node[anchor=east] {$g_3=0$};
	\draw (-0.5,-2.25) node[anchor=east] {$g_1=0$};
	\draw (2,0.25) node[anchor=east] {$g_2=0$};
	\draw (-0.25,-1.25) node[anchor=east] {$G$};

	\draw[->] (3.625,-0.85) -- (4.77,-1.2);
	\draw[->] (3.625,-0.85) -- (4.5, -1.7);
	\draw[->] (3.625,-0.85) -- (4.6, -0.6);
	
	\draw (4.77,-1.2) node[anchor=west] {$-\nabla f(\overline{x})$};
	\draw (4.5, -1.7) node[anchor=west] {$\nabla g_1(\overline{x})$};
	\draw (4.6, -0.6) node[anchor=west] {$\nabla g_2(\overline{x})$};

	\draw (4.77,-1.2) circle (0.4);
	\draw (4.77,-1.2) circle (0.8);
	\draw (4.77,-1.2) circle (1.2);
	\draw (3.625,-0.85) circle (0.05);
	\draw (3.625,-0.85) node[anchor=east] {$\overline{x}$};

	\draw (4.77,0) node[anchor=south] {$f= const.$};
	
\end{tikzpicture}

	\caption{Optimalstelle mit $I_0(\overline{x})=\{1,2\}$}
	\label{fig:kap_1_ktb_opt_1_2}
\end{figure}
\begin{enumerate}[label=\alph*)]
  \item zu \figureref{fig:kap_1_ktb_opt_0}: $u_ig_i(\overline{x})=0\Rightarrow u_1=u_2=u_3=0\Rightarrow \nabla f(\overline{x})=0$
  \item zu \figureref{fig:kap_1_ktb_opt_1}: $u_ig_i(\overline{x})=0\Rightarrow u_2=u_3=0\Rightarrow \nabla f(\overline{x})+u_1\nabla g_1(\overline{x})=0$
  \item zu \figureref{fig:kap_1_ktb_opt_1_2}: $u_ig_i(\overline{x})=0\Rightarrow u_3=0\Rightarrow \nabla f(\overline{x})+u_1\nabla g_1(\overline{x})+u_2\nabla g_2(\overline{x})=0$ mit
  $u_1$, $u_2\ge 0$\\
\end{enumerate}
\begin{remark}
	$x$ ist Maximumstelle von $f$ genau dann, wenn $x$ Minimumstelle von $-f$ ist.  
\end{remark} 
\begin{gegenexmp}\hspace{1cm}\\
\begin{minipage}{0.5\textwidth}
\begin{align*}
f(x) & = (x_1+1)^2+x^2_2\\
g_1(x) & = x_2-x^3_1 \le 0\\
g_2(x) & = -x_2 \le 0
\end{align*}
Minimumstelle: $(0,0)$
\end{minipage}
\begin{minipage}{0.5\textwidth}
\centering
\begin{tikzpicture}[auto, scale=1.5, >=latex']

	\fill[pattern=north east lines wide, pattern color=red] (0,0) rectangle (2, -0.2);

	\draw[->] (-2,0) -- (2,0) node[right] {$x_1$};;
	\draw[->] (0,-2) -- (0,2) node[above] {$x_2$};;
	
   	\draw (-2,0) node[anchor=north] {-2};
   	\draw (-1,0) node[anchor=north] {1};
   	\draw (0,0) node[anchor=north] {0};
   	\draw (1,0) node[anchor=north] {1};
   	\draw (2,0) node[anchor=north] {2};

   	\draw (0,-1) node[anchor=east] {1};
   	\draw (0,1) node[anchor=east] {1};
   	\draw (0,2) node[anchor=east] {2};
   	
   	\foreach \x in {-2,-1.5,...,1.5}
     	\draw (\x,0.02) -- (\x,-0.02);
   	
   	\foreach \y in {-2,-1.5,...,1.5}
     	\draw (0.02,\y) -- (-0.02,\y);

	\draw[thick,color=green] plot[samples=200, domain=-1.25:1.25] (\x,{\x*\x*\x});
	\draw[thick,color=green,pattern=north west lines wide, pattern color=green] plot[samples=200, domain=0:1.25] (\x,{\x*\x*\x});

	\draw[thick,color=blue] (-1,0) circle (0.5);
	\draw[thick,color=blue] (-1,0) circle (1);

\end{tikzpicture}

\end{minipage}
\end{gegenexmp}
\begin{remark}
  Die Forderung nach der Unabhängigkeit der Vektoren in $\nabla h_1(\overline{x}),\ldots,\nabla h_m(\overline{x})$ in Satz \ref{satz:3} bzw. $\nabla g_i(\overline{x})$ mit
  $i\in I_0(\overline{x})$ in Satz \ref{satz:4} ist eine sogenannte \textit{Regularitätsbedingung} (\ac{LICQ}). Es gibt andere Regularitätsbedingungen, z.B. die Forderung das
  $h_1,\ldots,h_m$ bzw. $g_1,\ldots,g_m$ affine lineare Funktionen sind, d.h. eine Darstellung der Form 
  \begin{align*}
  	M = \left\{ x\in\mathbb{R}^n:\ Ax=b \right\} & \text{ bzw. } G=\left\{ x\in\mathbb{R}^n:\ Ax\le b \right\}
  \end{align*}
  mit der Matrix $A\in\mathbb{R}^{m\times n}$ und dem Vektor $b\in\mathbb{R}^m$ möglich ist.
\end{remark}
Für die Aufgabe $\min f(x)$ bei $g_i(x)\le 0$ ($i=1,\ldots,m$), $h_j(x)=0$ ($j=1,\ldots,p$) mit stetig differenzierbaren $f$, $g_i$, $h_j$ lautet die \ac{KTB}:
\begin{align}
	\nabla_x L(x,u,\lambda) & = \nabla f(x)+\sum\limits_{i=1}^m u_ig_i(x)+\sum\limits_{j=1}^p\lambda_j h_j(x) \label{eq:kap_1_ktp}\\
	& h(x) = 0,\ u\ge 0,\ g(x)\le 0,\ u_ig_i(x)=0\ (i=1,\ldots,m) \notag
\end{align}
Ein Punkt $L(x,u,\lambda)$ der Gleichung \eqnref{eq:kap_1_ktp} erfüllt, heisst \ac{KTP}.
\begin{exmp}
Gesucht ist die optimale Lösung der Aufgabe $f(x)=-x^2_1+x_2^2\rightarrow\min$ bei $(x_1+1)(x_2+2)\le 4$ mit $x_1\ge 0$, $x_2\ge 0$.\\
Mit 
\begin{align*}
	g_1(x) & = (x_1+1)(x_2+2)-4\le 0,\\
	g_2(x) & = -x_1\le 0,\\
	g_3(x) & = -x_2\le 0 
\end{align*}
erhält man
\begin{align*}
	L(x,u) & = f(x)+\sum\limits_{i=1}^3u_ig_i(x)
\end{align*}
und
\begin{align*}
\nabla_x L(x,u) & = \begin{bmatrix}
-2x_1+u_1(x_2+2)-u_2\\
2x_2+u_1(x_1+1)-u_3
\end{bmatrix}.
\end{align*}
Die $5$ nichtlinearen Gleichungen $\nabla_x L(x,u)=0$, $u_ig_i(x)=0$ mit $i=1,2,3$ haben die reelen Lösungen:
\begin{itemize}
	\item $u_1=0$, $u_2=0$, $u_3=0$, $x_1=0$, $x_2=0$
	\item $u_1=-4$, $u_2=16$, $u_3=0$, $x_1=1$, $x_2=2$ erfällt wegen Nichterfüllen der Ungleichung 
	\item $u_1=1$, $u_2=0$, $u_3=2$, $x_1=1$, $x_2=0$
\end{itemize}
Wegen $f(0,0)=0$, $f(1,0)=-1$ ist $x=\begin{pmatrix} 1 & 0 \end{pmatrix}^T$ die gesuchte Minimumstelle.
\begin{figure}[!htb]
	\centering
	\begin{tikzpicture}[auto, scale=2.0, >=latex']

	\fill[pattern=north east lines wide, pattern color=red] (0,0) rectangle (2, -0.2);
	\fill[pattern=north east lines wide, pattern color=orange] (0,0) rectangle (-0.2, 2.5);

	\draw[->] (-2,0) -- (2,0) node[right] {$x_1$};;
	\draw[->] (0,-1) -- (0,2.5) node[above] {$x_2$};;
	
   	\draw (-2,0) node[anchor=north] {-2};
   	\draw (-1,0) node[anchor=north] {1};
   	\draw (0,0) node[anchor=north] {0};
   	\draw (1,0) node[anchor=north] {1};
   	\draw (2,0) node[anchor=north] {2};

   	\draw (0,-1) node[anchor=east] {1};
   	\draw (0,1) node[anchor=east] {1};
   	\draw (0,2) node[anchor=east] {2};
   	
   	\foreach \x in {-2,-1.5,...,1.5}
     	\draw (\x,0.02) -- (\x,-0.02);
   	
   	\foreach \y in {-1,-0.5,...,2}
     	\draw (0.02,\y) -- (-0.02,\y);

	\draw[thick,color=black] plot[samples=200, domain=-1:2] (\x,{\x});

	\draw[thick,color=yellow] plot[samples=200, domain=1:2] (\x,{sqrt(\x*\x-1)});
	\draw[thick,color=yellow] plot[samples=200, domain=1:1.4] (\x,{-sqrt(\x*\x-1)});
	
	\draw[thick,color=green] plot[samples=200, domain=-1.75:1.75] (\x,{sqrt(\x*\x+1)});
	
	\draw[thick,color=blue,pattern=north east lines wide, pattern color=blue] plot[samples=200, domain=0:1] (\x,{4/(\x+1)-2});


\end{tikzpicture}

	\caption{Beispiel mit \ac{KTB}: $f=-1$ (Gelb), $f=0$ (Schwarz), $f=1$ (Grün), $g_1$ (Blau), $g_2$ (Rot), $g_3$ (Orange)}
	\label{fig:kap_1_ktp}
\end{figure}
\end{exmp}

\textit{Konvexe Optimierungsprobleme}:\\
$M\subset\mathbb{R}^n$ konvex $\Leftrightarrow \left[x_1,x_2\in M\Rightarrow \lambda x_1+(1-\lambda)x_2\in M\forall\lambda\in[0,1]\right]$
\begin{figure}[!htb]
	\centering
	\subfloat[konvex]{\begin{tikzpicture}[auto, >=latex'] 
	\draw plot[smooth cycle] coordinates{(0,0) (1,0) (2.5,1) (1,2.5) (0,1)};
	
	\draw (0.8,2) node[anchor=north] {$M$};
	\draw (0.5,0.5) node[anchor=north] {$x_1$};
	\draw (1.5,1.5) node[anchor=south] {$x_2$};
	\draw (0.5,0.5) node[branch] (x1) {};
	\draw (1.5,1.5) node[branch] (x2) {};
	
	\draw [->] (x2) -- node {$\lambda$} (x1);
	
\end{tikzpicture}
}\qquad
	\subfloat[nicht konvex]{\begin{tikzpicture}[auto, >=latex'] 
	\draw plot[smooth cycle] coordinates{(0,0) (1,0) (2,1) (3,0) (4,0) (4,1) (3,2) (1,2) (0,1)};
	
	\draw (2,2) node[anchor=north] {$M$};
	\draw (3.5,0.8) node[anchor=south] {$x_1$};
	\draw (0.5,0.2) node[anchor=south] {$x_2$};
	\draw (3.5,0.8) node[branch] (x1) {};
	\draw (0.5,0.2) node[branch] (x2) {};
	
	\draw [dashed] (x2) -- (x1);
	
\end{tikzpicture}
}
	\caption{Konvexität}
	\label{fig:kap_1_konvex}
\end{figure}

Sei $M\subset\mathbb{R}^n$ konvex: $f:M\rightarrow\mathbb{R}$ konvex auf $M$, wenn 
\begin{align*}
	\left[ x_1,x_2\in M \Rightarrow M(\lambda x_1+(1-\lambda)x_2)\right. & \left.\le \lambda f(x_1)+(1-\lambda)f(x_2)\forall\lambda\in[0,1]\right]
\end{align*}
\begin{figure}[!htb]
	\centering
	\subfloat[$f$ konvex]{\begin{tikzpicture}[auto, >=latex'] 

	\draw[->] (0,0) -- (5,0) node[right] {$x$};
	\draw[->] (0,-0.02) -- (0,4) node[above] {$f(x)$};
	
	\draw[dotted] (4.1,0) -- node[anchor=west] {$f(x_1+(1-\lambda)x_2)$} (4.1,4); 
	\draw (0, 0.5) node[anchor=east] {$f(x_2)$};
	\draw (4.1, 4) node[anchor=west] {$f(x_1)\forall\lambda\in(0,1)$};
	
	\draw[black] plot[samples=200, domain=0:4] (\x,{0.5*exp(0.5*\x)});
		
	\draw (4.1, 0.02) -- (4.1, -0.02);
	\draw (0, -0.02) node[anchor=north] {$0$};
	\draw (4.1, -0.02) node[anchor=north] {$1$};

\end{tikzpicture}
}\qquad
	\subfloat[$f$ nicht konvex]{\begin{tikzpicture}[auto, >=latex'] 

	\draw[->] (0,0) -- (4,0) node[right] {$x$};
	\draw[->] (0,0) -- (0,4) node[above] {$f(x)$};
	
	\draw[black] plot[samples=200, domain=0:4] (\x,{2.0+sin(6*\x r-pi)});
			
\end{tikzpicture}
}
	\caption{Konvexe Funktionen}
	\label{fig:kap_1_konvex_fkt}
\end{figure}

\textit{streng konvex auf} $M$:\\
Sei $M\subset\mathbb{R}^n$ offen, konvex, $f\in C^2(M,\mathbb{R})$:
\begin{itemize}
  \item $f$ konvex auf $M\Leftrightarrow\forall x\in M:H f(x)\ge 0$
  \item $f$ streng konvex auf $M\Leftrightarrow\forall x\in M: H f(x)>0$
\end{itemize}
Falls $M$ konvex und $f$ konvex auf $M$, so heisst $\min\limits_{x\in M}f(x)$ konvexe Optimierungsaufgabe und es gilt:
\begin{itemize}
  \item Falls $x^{\ast}\in M$ eine lokale Minimumstelle ist, so ist $x^{\ast}$ auch globale Minimumstelle auf $M$.
  \item Ist $f$ streng konvex, so gibt es höchstens ein globales Minimum von $f$ über $M$.
\end{itemize}
\begin{exmp}
Die streng konvexe Funktion $f(x)=e^x$ soll minimiert werden $\min\limits_{x\in\mathbb{R}^1}e^x$. Es soll gezeigt werden, dass höchstens ein globales Minimum von $f$ über $M$ existiert.
Dabei ist zu beachten:
\begin{align*}
\left.
 \begin{tabular}{rl}
 $f:\mathbb{R}^n$ & $\rightarrow \mathbb{R}^1$\\
 $g_i:\mathbb{R}^n$ & $\rightarrow \mathbb{R}^1$\\
 $h_j:\mathbb{R}^n$ & $\rightarrow \mathbb{R}^1$
 \end{tabular}\right\} & \text{stetig differenzierbar mit }i=1,\ldots,n\text{ und } j=1,\ldots,p
\end{align*}
Damit lässt sich die lokale Minimumstelle
\begin{align*}
x^{\ast}\in G & := \left\{ x\in\mathbb{R}^n:g_i(x)=0\ (i=1,\ldots,n),\ h_j(x)=0\ (j=1,\ldots,p) \right\}\\
\left\{ \nabla(x^{\ast})\ (i\in I_0(x^{\ast}) \right. & \left. := \left\{ i\in\left\{1,\ldots,m\right\}|\ g_i(x^{\ast})=0 \right\}),\nabla h_j(x^{\ast})\ (j=1,\ldots,p) \right\} 
\end{align*}
beschreiben und für die lokale Minimumstelle $x^{\ast}$ von $f$ auf $G$  gilt $\Rightarrow\exists u^{\ast}\in\mathbb{R}^m$, $\exists\lambda^{\ast}\in\mathbb{R}^p$ mit der \ac{KTB}
\begin{align*}
	\nabla_x L(x^{\ast},u^{\ast}) & = \nabla f(x^{\ast})+\sum\limits_{i=1}^n u_i^{\ast}\nabla g_i(x^{\ast})+\sum\limits_{j=1}^p\lambda_j^{\ast}\nabla h_j(x^{\ast})=0
\end{align*}
mit $g(x^{\ast})\le 0$, $u^{\ast}\ge 0$, $u^{\ast T}g(x^{\ast})=0$, $h(x^{\ast})=0$ und dem \ac{KTP}: $\begin{pmatrix}x^{\ast}, & u^{\ast}, & \lambda^{\ast} \end{pmatrix}$
\end{exmp}
 
\begin{gegenexmp}\hspace{1cm}\\
\begin{minipage}{0.5\textwidth}
\begin{align*}
f(x) & = (x_1+1)^2+x^2_2\\
g_1(x) & = x_2-x^3_1 \le 0\\
g_2(x) & = -x_2 \le 0
\end{align*}
Minimumstelle: $x^{\ast}=(0,0)$, aber \ac{KTB} nicht erfüllt\\
$\Rightarrow \nabla g_1(x^{\ast})$, $\nabla g_2{x^{\ast}}$ linear abhängig

\end{minipage}
\begin{minipage}{0.5\textwidth}
	\centering
	\begin{tikzpicture}[auto, scale=1.5, >=latex']

	\draw[->] (-2,0) -- (2,0) node[right] {$x_1$};;
	\draw[->] (0,-2) -- (0,2) node[above] {$x_2$};;
	
   	\draw (-2,0) node[anchor=north] {-2};
   	\draw (-1,0) node[anchor=north] {1};
   	\draw (0,0) node[anchor=north] {0};
   	\draw (1,0) node[anchor=north] {1};
   	\draw (2,0) node[anchor=north] {2};

   	\draw (0,-1) node[anchor=east] {1};
   	\draw (0,1) node[anchor=east] {1};
   	\draw (0,2) node[anchor=east] {2};

	\draw (0,0) node[anchor=east] {$x^{\ast}$};
   	
   	\foreach \x in {-2,-1.5,...,1.5}
     	\draw (\x,0.02) -- (\x,-0.02);
   	
   	\foreach \y in {-2,-1.5,...,1.5}
     	\draw (0.02,\y) -- (-0.02,\y);

	\draw[thick,color=green] plot[samples=200, domain=-1.25:1.25] (\x,{\x*\x*\x});

	\draw[thick,color=blue] (-1,0) circle (0.5);
	\draw[thick,color=blue] (-1,0) circle (1);

\end{tikzpicture}

\end{minipage}
\end{gegenexmp}

\begin{satz}\label{satz:5}
Für die Aufgabe $\min\limits_{x\in G}f(x)$ mit $G:=\left\{ x\in\mathbb{R}^n|\ g_i(x)\le 0\ (i=1,\ldots,m) \right\}$ und $f$, $g_i$ ($i=1,\ldots,m$) stetig differenzierbar und konvex
gilt:
\begin{itemize}
  \item $x^{\ast}$ ist Lösung und $\exists\tilde{x}\in\mathbb{R}^n:g_i(\tilde{x})<0$ ($i=1,\ldots,m$) (Slater Bedingung)\\
  $Rightarrow \exists u^{\ast}\in\mathbb{R}^m:(x^{\ast},u^{\ast})$ ist \ac{KTB}
  \item $(x^{\ast},u^{\ast})$ sind \ac{KTP} $\Rightarrow x^{\ast}$ ist Lösung für alle
  \item Falls alle $g_i$ linear, d.h. $G=\left\{x\in\mathbb{R}^n:Ax\le b \right\}$ mit $A\in\mathbb{R}^{m\times n}$, $b\in\mathbb{R}^m$, so gilt
  \begin{align*}
  	x^{\ast}\text{ ist Lösung } &\Leftrightarrow \exists u^{\ast}\in\mathbb{R}^m:(x^{\ast},u^{\ast})\text{ ist \ac{KTP}}
  \end{align*}
  Bei zusätzlichen linearen \ac{GNB}, d.h. $G=\left\{x\in\mathbb{R}^n:Ax\le b,\ Cx=d \right\}$ mit $C\in\mathbb{R}^{p\times n}$, $d\in\mathbb{R}^p$:
  \begin{align*}
  	x^{\ast}\text{ ist Lösung } &\Leftrightarrow \exists u^{\ast}\in\mathbb{R}^m,\lambda^{\ast}\in\mathbb{R}^m:(x^{\ast},u^{\ast},\lambda^{\ast})\text{ ist \ac{KTP}}
  \end{align*}
\end{itemize}
\end{satz}
Es ist eine Übung unter \picref{sec:uebung_kapitel_1_ktb}{Kuhn-Tucker-Bedingung} im Anhang zu finden.

\section{Anwendungen auf quadratische Optimierungsaufgaben}
\textbf{Aufgabe}
\begin{itemize}
	\item[] $\min\limits_{x\in\mathbb{R}^n} \frac12 x^TQx+q^Tx$ bei $Ax\le b$, $Q>0$
	\item[] mit $Q\in\mathbb{R}^{n\times n}$, $q\in\mathbb{R}^n$, $A\in\mathbb{R}^{m\times n}$, $b\in\mathbb{R}^m$ (convex) 
\end{itemize}
\subsection{Vorbereitungen}
\begin{enumerate}[label=(\arabic*)]
  \item $\min\limits_{x\in\mathbb{R}^n}f(x)$ mit $f(x)=\frac12 x^TQx+q^Tx$ und wegen $H f(x)=Q\ge 0$ ist $f$ convex auf $\mathbb{R}^n$. Damit gilt $x^{\ast}$ ist Lösung
  $\Leftrightarrow x^{\ast}$ erfüllt \ac{KTB} $\Leftrightarrow\nabla f(x^{\ast})=0$. Somit erhällt man die Lösung von (1-TODO) durch Lösen des linearen Gleichungssystems
  \begin{align*}
  	\nabla f(x) & = Qx+q =0
  \end{align*} 
  \textit{Beachte}: Lösung muss nicht existieren.
  \item $\min\limits_{x\in\mathbb{R}^n}\frac12 x^TQx+q^Tx$ bei $Cx=d$ mit $C\in\mathbb{R}^{p\times n}$, $d\in\mathbb{R}^p$. Analog zu (1): $x^{\ast}$ ist Lösung
  $\xLeftrightarrow{\text{Satz \ref{satz:5}}}\exists\lambda^{\ast}\in\mathbb{R}^p:(x^{\ast},\lambda^{\ast})$ ist \ac{KTB}. Mit $L(x,\lambda)=\frac12
  x^TQx+q^Tx+\lambda^T(Cx-d)$ liefert
  \begin{align*}
  \nabla L(x,\lambda) & = \begin{bmatrix}
  Qx+q+C^T\lambda \\
  Cx-d
  \end{bmatrix} = 0
  \end{align*} 
  unter den \ac{KTB} $\nabla_xL(x,\lambda)=0$ und $h(x)=0$ liefert das lineare Gleichungssystem
  \begin{align*}
  \begin{bmatrix}
  Q	& c^T\\ C	& 0 
  \end{bmatrix}\begin{bmatrix}
  x\\ \lambda
  \end{bmatrix} & = \begin{bmatrix}
  -q \\ d
  \end{bmatrix}
  \end{align*}
  \underline{Beachte}: Kann keine oder mehrdeutige Lösungen besitzen.\\
  Es ist eine Übung unter \picref{sec:uebung_bsp_quad_opt}{Beispiele zur Quadratischen Optimierung} im Anhang zu finden.
  \item Projektion auf Untervektorraum\\
  		$A\in\mathbb{R}^{m\times n}$, $m\ge n$, $\Rang A=n$, $b\in\mathbb{R}^m$\\
  		\begin{figure}[!htb]
			\centering
			\begin{tikzpicture}[scale=2.0]

	\draw[->] (-0.5,0) -- (3,0);
	\draw[->] (0,-0.5) -- (0,2);
	
	\draw[-] (-1,2) -- (0.25, -0.5);
	\draw[-] (-0.5,-0.25) -- (3, 1.5);
	
	\draw[->,color=blue] (2,2) -- node[anchor=east] {$I-P$} (-0.4,0.8);
	\draw[->,color=blue] (2,2) -- node[anchor=west] {$P$} (2.4, 1.2);
	
	\draw (2,2) node[branch] {};
	\draw (2,2) node[anchor=south] {$b$};
	\draw (2.4,1.2) node[anchor=west] {$Pb=A\widehat{x}$};
	\draw (-0.4,0.8) node[anchor=east] {$(I-P)b$};
	
	\rechterWinkel{0,0}{27}{.25}
	\rechterWinkel{2.4,1.2}{27}{.25}
		
\end{tikzpicture}

			\caption{Projektion}
			\label{fig:kap_1_projektion}
		\end{figure}
  		\begin{align}
  			\Image A & := \left\{Ax:x\in\mathbb{R}^n \right\} \notag\\
  			(\Image A)^{\bot} =\ker A^T & := \left\{y\in\mathbb{R}^m:A^Ty=0 \right\}\notag\\
  			A^T(b-A\hat{x}) & = 0\notag\\
  			A^Tb & = A^TA\hat{x}\notag\\
  			\hat{x} & = (A^TA)^{-1}A^Tb\notag\\ 
  			\rightarrow P & = A(A^TA)^{-1}A^T\quad \ldots\text{ Projektor auf $\Image A$} \label{eq:kap_1_projektion}
  		\end{align}
  		Es ist eine Übung unter \picref{sec:uebung_proj_untervekraum}{Projektion auf Unterverktorraum} im Anhang zu finden.
  		
  		Projektion eines Vektors $v\in\mathbb{R}^n$ auf $M:=\left\{x\in\mathbb{R}^n|\ Cx=d \right\}$ mit $C\in\mathbb{R}^{p\times n}$, $p\le n$, $\Rang c = p$, siehe
  		\figureref{fig:kap_1_projektion} und man erhält den Projektor durch ersetzen von $A^T$ durch $C$ in \eqnref{eq:kap_1_projektion}
  		\begin{align*}
  			\mathcal{P} & = C^T(CC^T)^{-1}C\quad \ldots\text{ Projektor auf $M$}. 
  		\end{align*}
  		\begin{figure}[!htb]
			\centering
			\begin{tikzpicture}[auto, scale=2.0, >=latex']

	\draw[->] (-0.5,0) -- (3,0);
	\draw[->] (0,-0.5) -- (0,3);
	
	\draw[thick,color=black] plot[samples=200, domain=-0.5:3] (\x,{0.5*\x+1});
	\draw[thick,color=black] plot[samples=200, domain=-0.5:3] (\x,{0.5*\x});
	
	\draw[->,color=blue] (1,1.5) -- node[anchor=north] {$\mathcal{P}_v$} (1.5,1.75);
	\draw[dotted] (1.5,1.75) -- (1.25,2.25);
	\draw[->,color=blue] (1,1.5) -- node[anchor=east] {$v$} (1.25,2.25);
	
	\rechterWinkel{1.5,1.75}{27}{.25}
	
	\draw[<-] (1.25,0.625) -- node[anchor=east] {$\mathcal{P}$} (1,1.125);
	\draw[->, dotted] (0,0) -- (1,1.125);

	\rechterWinkel{1.25,0.625}{27}{.25}
	
	\draw (1.25,0.625) node[anchor=north] {$\mathcal{P}_v$};
	\draw (1,1.125) node[anchor=south] {$v$};
	\draw (2.5,1.25) node[anchor=west] {$\{x\in\mathbb{R}^n: Cx=0\}$};
	
\end{tikzpicture}

			\caption{Projektion}
			\label{fig:kap_1_projektion}
		\end{figure}
  		Es ist eine Übung unter \picref{sec:uebung_proj_vektor}{Projektion eines Vektors} im Anhang zu finden.
\end{enumerate}

\subsection{Aktive-Mengen-Strategie}
Die Aufgabe besteht in
\begin{align}
  f(x) = \frac12 x^TQx+q^Tx\rightarrow\min\text{ bei } Ax\le b.
\end{align}
Seien $Q\in\mathbb{R}^{n\times n}$, $Q>0$ possitiv definit, $q\in\mathbb{R}^n$, $A=\begin{bmatrix}a_1^T\\\vdots\\ a_m^T\end{bmatrix}\in\mathbb{R}^{m\times n}$, $b\in\mathbb{R}^m$.\\
Für $x\in G:=\left\{x\in\mathbb{R}^n|\ Ax\le b \right\}$ bezeichne $I_a(x)=\left\{i\in(1,\ldots,m):a_i^Tx=b_i\right\}$\footnote{Menge der im Punkt $x\in G$ aktiven Restriktionen} und
$M(I) = \left\{y\in\mathbb{R}^n|\ a_i^Ty=b_i, i\in I_a(x) \right\}$.\\
Sei $\left\{a_i|\ i\in I_a(x) \right\}$ linear unabhängig $\forall x\in G$.\\
\ac{NB}:
\begin{align*}
  g_i(x) := a_i^Tx -b_i & \le 0\\
  \nabla g_i(x) & = a_i
\end{align*}

\subsubsection{Algorithmus der Aktive-Mengen-Strategie}
\begin{enumerate}[label=(S\arabic*)]
  \item Wähle ein $x^0\in G$, setze $I_0:=I_a(x^0)$ und setze $k:=0$.
  \item Falls Projektion von $\nabla f(x^k)=Qx^k+q$ auf $M(I_k)$ gleich dem Nullvektor ist, so sind die zu $x^k$ gehörigen Lagrange-Multiplikatoren
  $u_i^k$ aus $\nabla f(x^k)+\sum\limits_{i\in I_k}u_i^ka_i=0$ zu bestimmen.
  \begin{enumerate}[label=(S2\alph*)]
    \item Falls $u_i^k\ge 0$ mit $i\in I_k$, so ist $x^k$ Optimalstelle. Stop!
    \item Andernfalls sind der Index $r\in I_k$ mit $u_i^k<0$ zu bestimmen und es ist $I_k\leftarrow I_k\backslash\{r\}$\footnote{Interpretation: Entferne $r$ aus der Indexmenge $I_k$}
    zu setzen.
  \end{enumerate}
  \item Bestimme $y^k:=\argmin f(y)$\footnote{Bemerkung: $x^{\star}=\argmin f(x) \leftrightarrow x^{\star}\ldots$ Minimumstelle von $f$} bei $y\in
  M(I_k)$.\\
  		Setze $\alpha_k=\max\left\{\alpha\in[0,1]:x^k+\alpha(y^k-x^k)\in G\right\}$, $x^{k+1}=x^k+\alpha(y^k-x^k)$, $I_{k+1}:=I_a(x^{k+1})$ und gehe mit
  		$k\rightarrow k+1$ zu (S2)
\end{enumerate}
\begin{remark}[zu (S3)]
  Die im Punkt $x^k$ inaktive UNBen sind $a_j^Tx^k\le b_j$, $j\notin I_k$.
  Wir bestimmen das maximale $\beta > 0$ derart, dass $\beta(y^k-x^k)+x^k$ noch zulässig ist, d.h. $a_j^T(\beta(y^k-x^k)+x^k)\le b_j$.
  Dies ist wegen $a_j^Tx^k\le b_j$ und $\beta a_j^T(y^k-x^k)\le b_j-a_j^Tx^k$ stets für diejenigen $j$ erfüllt, für die $a_j^T(y^k-x^k)\le 0$ gilt. 
  Also ergibt sich das gesuchte \begin{align*}
  \beta & = \min\left\{\left.\ gfrac{b_j-a_j^Tx^k}{a_j^T(y^k-x^k)}\right|\ j\notin J_k\text{ mit } a_j^T(y^k-x^k)>0 \right\}.
  \end{align*}
  (S3) berechnet $\alpha_k=\min\left\{1,\beta\right\}$.
\end{remark}
\begin{remark}[zum Algorithmus]\hspace{1mm}
\begin{itemize}
  \item Die Voraussetzungen ``$C>0$'' und ``$\left\{a_i|\ i\in I_a(x)\right\}$ linear unabhängig $\forall k\in G$'' sichern die eindeutige Lösbarkeit des linearen Gleichungssystems in
  (S2) und (S3)
  \item Alle Iterationen $x^k$ sind zulässig. In der Praxis gilt bei jedem Durchlauf von (S3): $f(x^{k+1})<f(x^k)$. Der Algorithmus ist endlich (gutartiger, schneller Algorithmus).
\end{itemize}
\end{remark}
\subsubsection{Finden eines zulässigen Startpunktes}
Zur Aufgabe 
\begin{align}
	f(x) & =\frac12 x^TQx+q^Tx\rightarrow\min\limits_{x\in\mathbb{R}^n} \text{ bei } Ax\le b \label{eq:startpkt_1}
\end{align} 
bildet Aufgabe 
\begin{align}
	y & \rightarrow\min\limits_{\substack{x\in\mathbb{R}^n\\y\in\mathbb{R}^1}} \text{ bei } Ax\le b+\mathds{1}_m y,\ 0\le y \label{eq:startpkt_2}\\
	& \text{mit: } \mathds{1}_m\ldots \text{ Einheitsmatrix} \notag\\
	\Leftrightarrow \begin{bmatrix} 0 & \ldots & 0 & 1 \end{bmatrix}\begin{bmatrix} x\\y \end{bmatrix} & \rightarrow\min  \text{ bei } \begin{bmatrix} A & -\mathds{1}_m\\0 & -1
	\end{bmatrix}\begin{bmatrix} x\\y \end{bmatrix}\le \begin{bmatrix} b\\0 \end{bmatrix} \label{eq:startpkt_3}
\end{align}
Zulässiger Startwert:
\begin{itemize}
  \item[] $x^0$ beliebig und $y^0:=\max\left([0;\ Ax^0-b]\right)$
  \item[] $\begin{bmatrix}\overline{x}\\ \overline{y} \end{bmatrix}$ sei Lösung der linearen Optimierungsaufgabe \eqnref{eq:startpkt_3}.
  \item[] Falls $\overline{y}=0$, so gilt 
  $A\overline{x}\le b$, d.h. $\overline{x}$ ist zulässig von \eqnref{eq:startpkt_1}.
\end{itemize}
Es ist eine Übung unter \picref{sec:uebung_finden_startwert}{Finden eines zulässigen Startwertes für den Aktiven Mengen Algorithmus} im Anhang zu finden.


\subsubsection{Erweiterungen des Algorithmus}
\begin{itemize}
  \item Falls $Q=0$, d.h. $f(x)=q^Tx\rightarrow\min$ bei $Ax\le b$, so liegt eine Aufgabe der linearen Optimierung vor. In (S3) wir dann $\alpha_k:=\max\left\{ \alpha\ge 0: x^k+\alpha
  s^k\in G \right\}$ wobei $s^k$ der auf $M(I_k)$ projizierte negative Gradient $-\nabla f(x^k)=-q$ ist.
  \item Treten UNB und GNB gleichzeitig auf, d.h. $f(x)=\frac12 x^TQx+q^Tx\rightarrow\min$ bei $Ax\le b$ und $Cx=d$ so werden die GNB wie aktive UNB behandelt.
\end{itemize}
Es ist eine Übung unter \picref{sec:uebung_erweiterung_algo}{Erweiterung des Aktiven Mengen Algorithmus} im Anhang zu finden.
 
\begin{exmp}\label{exmp:kap_1_ama}
Seien 
\begin{align*}
	A & =\begin{bmatrix}1 & \frac13\\0 & 1\\ -1 & 0 \end{bmatrix},\quad b=\begin{bmatrix}1\\1\\0 \end{bmatrix},\quad Q=\begin{bmatrix}2 & 0\\ 0 & 2
	\end{bmatrix},\quad q=\begin{bmatrix}-2\\-4 \end{bmatrix}.
\end{align*}
\begin{enumerate}[label=(S\arabic*)]
  \item Mit $x^0:=\begin{bmatrix}0\\0 \end{bmatrix}$ ist $I_0=\{3\}$.
  \item Projektion von $\nabla f(x^0)=\begin{bmatrix}-2\\-4 \end{bmatrix}$ auf $M(I_0)=\left\{\begin{bmatrix}0\\x_2 \end{bmatrix}:x_2\in\mathbb{R}
  \right\}$ ist $\begin{bmatrix}0\\-4 \end{bmatrix}\neq 0$.
  \item Man findet $y^0=\begin{bmatrix}0\\2 \end{bmatrix}$, $x^1=\begin{bmatrix}0\\1 \end{bmatrix}$, $I_1=\{2,3\}$.
  \item[(S2)] Projektion von $\nabla f(x^1)=\begin{bmatrix}-2\\-2 \end{bmatrix}$ auf $M(I_1)=\left\{\begin{bmatrix}0\\1
  \end{bmatrix}\right\}$\footnote{Bemerkung: d.h. nur ein Punkt! Projektion auf 1 Punkt ist immer der Nullvektor.} ist $\begin{bmatrix}0\\0
  \end{bmatrix}$ an der Stelle an der beide Restriktionen wirken mit $I_1=\{2,3\}$.\\
  Lagrange-Multiplikatoren $\begin{bmatrix}u_2^1\\u_3^1 \end{bmatrix}=\begin{bmatrix}2\\-2 \end{bmatrix}$, setze $I_1\leftarrow
  I_1\backslash\{3\}=\{2\}$.
  \item[(S3)] Man findet $y^1=\begin{bmatrix}1\\1 \end{bmatrix}$, $x^2=\begin{bmatrix}\frac23\\1 \end{bmatrix}$, $I_2=\{1,2\}$.
  \item[(S2)] Projektion von $\nabla f(x^2)=\begin{bmatrix}-\frac23\\-2 \end{bmatrix}$ auf $M(x^2)$ ist $\begin{bmatrix}0\\0 \end{bmatrix}$.\\
  		Lagrange-Multiplikatoren $\begin{bmatrix}u^2_1\\u_2^2 \end{bmatrix}=\begin{bmatrix}\frac23\\\frac{16}{9} \end{bmatrix}\ge 0$. Stop!
\end{enumerate}
Grafische Darstellung der Optimierungsaufgabe und deren Lösung\\
\begin{minipage}[c]{0.5\textwidth}
Zielfunktion:
\begin{align*}
  f(x)=(x_1-1)^2+(x_2-2)^2-5
\end{align*}
Zulässiger Bereich:
\begin{enumerate}[label=(\arabic*)]
  \item $x_1+\frac{x_2}{3}\le 1$
  \item $x_2\le 1$
  \item $x_1 \ge 0$ 
\end{enumerate}
\end{minipage}
\hfill
\begin{minipage}[c]{0.5\textwidth}
\centering
\begin{tikzpicture}[auto, scale=1.5,>=latex']
	\fill[color=green!20] (0,-0.1) -- (0,1) -- (0.6666,1.0) -- (1.0333,-0.1);

	\draw[->] (0,0) -- (2.5,0) node[right] {$x_1$};
	\draw	(1,0) node[anchor=north] {$1$}
			(2,0) node[anchor=north] {$2$};

	\draw[->] (0,0) -- (0,2.5)node[above] {$x_2$};
	\draw	(0,1) node[anchor=east] {$1$}
			(0,2) node[anchor=east] {$2$};
	
	\foreach \x in {1,...,2}
     	\draw (\x,0.02) -- (\x,-0.02);
   	
   	\foreach \y in {1,...,2}
     	\draw (0.02,\y) -- (-0.02,\y);

	\draw[dashed, color=gray] (1,2) circle (0.05);
	\draw[dashed, color=gray] (1,2) circle (0.25);
	\draw[dashed, color=gray] (1,2) circle (0.5);
	
	\draw[thin, dotted, color=gray] (0,2) -- (2,2);
	\draw[thin, dotted, color=gray] (2,2) -- (2,0);
	\draw[thin, dotted, color=gray] (1,0) -- (1,2);
	\draw[thin, dotted, color=gray] (0,1) -- (2,1);
	
	\draw[thick, color=green] (0,-0.1) -- (0,1.8);
	\draw[thick, color=green] (-0.1,1) -- (1.8,1);
	\draw[thick, color=green] (1.0333,-0.1) -- (0.4,1.8);
	
	\draw (0.8,0.5) node[anchor=west] {(1)};
	\draw (1.5,1.0) node[anchor=south] {(2)};
	\draw (0.0,0.5) node[anchor=east] {(3)};
	
	
\end{tikzpicture}

\end{minipage}
\end{exmp}
Es ist eine Übung unter \picref{sec:uebung_ama}{Aktiver Mengen Algorithmus} im Anhang zu finden.

\chapter{Klassische Verfahren zur optimalen Steuerung}
\section{Grundproblem der Variationsrechnung}
Die Aufgabe besteht im Auffinden einer stetig differenzierbaren Funktion $x:[t_0,t_e]\rightarrow\mathbb{R}$ mit den Randbedingungen $x(t_0)=x_0$, $x(t_e)=x_e$, so dass ein
Gütefunktional $J=\int\limits_{t_0}^{t_e}f(t,x,\dot{x}) dt$ minimal wird.

Mit der Vorgehensweise von Euler wird angenommen, man hätte eine optimale Lösung $x^{\ast}$ gefunden. Konstruiere eine einparametrige Schar von Vergleichskurven
$x(t)=x^{\ast}+\epsilon\tilde{x}(t)$, wobei $\epsilon\in(-\epsilon_0,+\epsilon_0)$ ein Parameter ($\epsilon>0$ gegeben) und $\tilde{x}$ eine gegebene, stetig differenzierbare Funktion
mit $\tilde{x}(t_0)=0$, $\tilde{x}(t_e)=0$ ist.
\begin{figure}[htb]
	\centering
	\begin{tikzpicture}[auto, >=latex'] 
	\node [input] () {};
\end{tikzpicture}

	\caption{Vorgehensweise von Euler}
	\label{fig:kap_2_vorg_euler}
\end{figure}
Die Funktion $\delta x^{\ast}:= \epsilon\tilde{x}$ heisst Variation von $x^{\ast}$ und das einsetzen in $J$ liefert 
\begin{align*}
	F(\epsilon) & := \int\limits_{t_0}^{t_e}f(t,x^{\ast}(t)+\epsilon\tilde{x}(t),\dot{x}^{\ast}(t)+\epsilon\dot{\tilde{x}}(t))dt.
\end{align*}
Sei $f$ zweifach stetig differenzierbar. Die Funktion $F:(-\epsilon_0,+\epsilon_0)\rightarrow\mathbb{R}$ hat für $\epsilon=0$ ein Minimum, also muss gelten
\begin{align*}
	\left.\frac{\td F}{\td \epsilon}\right|_{\epsilon_0} & = \int\limits_{t_0}^{t_e}\left[\frac{\d f}{\d x}\tilde{x}(t)+\frac{\d f}{\d\dot{x}}\dot{\tilde{x}}(t)
	\right]dt = 0.
\end{align*}
Mit partieller Integration 
\begin{align*}
	\int\limits_{t_0}^{t_e}\underbrace{\frac{\d F}{\d \dot{x}}\dot{\tilde{x}}(t)}_{u\cdot v'}dt & = \bigg[\underbrace{\frac{\d
	F}{\d\dot{x}}\tilde{x}}_{u\cdot v}\bigg]_{t=t_0}^{t=t_e}-\int\limits_{t_0}^{t_e} \underbrace{\frac{\td}{\td t}\frac{\d F}{\partial\dot{x}}\tilde{x}(t)}_{u'\cdot v}dt
\end{align*}
ergibt sich
\begin{align*}
	\int\limits_{t_0}^{t_e}\left[\frac{\d F}{\d x}-\frac{\td}{\td t}\frac{\d F}{\d\dot{x}} \right]\tilde{x}(t)dt & = 0
\end{align*}
und da $\tilde{x}$ (bis auf die Randwerte) beliebig ist, muss 
\begin{align}
	\frac{\d F}{\d x}-\frac{\td}{\td t}\frac{\d F}{\d\dot{x}} & = 0
\end{align}
gelten.
\section{Formulierung des Optimierungsproblems und Lösung}
Ausgehend vom Kostenfunktional
\begin{align*}
	J & = h\left(x(t_b),t_b \right) + \int\limits_{t_a}^{t_b}f_0\left(x(t),u(t),t \right)dt\rightarrow \min!
\end{align*}
kann der Prozess, die \ac{AB} und die \ac{EB} definiert werden\\
\begin{tabular}{ll}
Prozess: & $\dot{x}(t)=f\left(x(t),u(t),t \right)$\\
\ac{AB}: & $x(t_a)=x_a$ mit gegebenen $t_a$ und $x_a\in\mathbb{R}^n$\\
\ac{EB}: & mit $t_b$ frei und/oder $t_b$ gegeben:\\
		 & Fall A: $z\left(x(t_b) \right)=0$ mit gegebenen $z:\mathbb{R}^n\rightarrow\mathbb{R}^m$\\
		 & Fall B: $x(t_b)=x_b$ mit geg. $x_b\Leftrightarrow z(x(t_b)):=x(t_b)-x_b=0$\\
		 & Fall C: $x(t_b)$ frei, d.h. $z\equiv 0$ 
\end{tabular}
$h$, $f_0$, $f$, $z$ sind stetig differenzierbar bezüglich aller Argumente. Es gibt keine Beschränkung von $u(t)$ und $x(t)$ für $t\in\left(t_a, t_b
\right)$.\\
Falls $t_b$ gegeben ist und Fall B besteht, so ist $h(x(t_b),t_b)$ fest und kann aus Kostenfunktional gestrichen werden.

\subsection{Prinzip der Herleitung notwendiger Bedingungen}
Die Einführung der Lagrange-Multiplikatoren liefert
\begin{align*}
	\bar{J} & = h\left(x(t_b),t_b \right)+\int\limits_{t_a}^{t_b}\left[f_0\left(x(t),u(t),t \right)+\Psi(t)^T\left(f(x(t),u(t),t)-\dot{x}(t) \right)
	\right]dt + \lambda_a^T\left\{x_a - x(t_a) \right\} + \lambda_b^T z\left(x(t_b) \right)
\end{align*}
mit $\lambda_b\in\mathbb{R}^m$, $\lambda_a\in\mathbb{R}^n$.

Die Motivation ist, dass 
\begin{align*}
	J & = \int F_0(x(t))dt \rightarrow \min
\end{align*}
bei $\underbrace{f\left(x(t), u(t) \right)-\dot{x}(t)=0 }_{h\left(x(t),u(t) \right)\in\mathbb{R}^n }$ dargestellt werden kann in diskreter Form als
\begin{align*}
	J_{diskret} & = \sum\limits_j f_0(x(t_j))\rightarrow \min 
\end{align*}
bei $\left. h\left(x(t_j),u(t_j) \right)\right|_{\forall j}=0$ bzw.
\begin{align*}
	L_{diskret} & = \sum\limits_j f_0(x(t_j)) + \sum\limits_j \underbrace{\Psi_j^T}_{=:\Psi(t_j)} h\left(x(t_j),u(t_j) \right).
\end{align*}
Dies kann wiederrum dargestellt werden in der Form
\begin{align*}
	\bar{J} & = \int\left(f_0(x(t)) \right) + \Psi(t)^T\left(f(x(t),u(t))-\dot{x}(t) \right).
\end{align*}
Es wird die Hamilton-Funktion definiert mit
\begin{align*}
	\Ham\left(x(t),u(t),\Psi(t),t \right) & = f_0\left(x(t),u(t),t \right) + \Psi(t)^T f\left(x(t),u(t),t \right)
\end{align*}
und erhalten 
\begin{align}
	\bar{J} & = h\left(x(t_b),t_b \right) + \int\limits_{t_a}^{t_b}\left[H\left(x(t),u(t),\Psi(t),t \right)-\Psi(t)^T\dot{x}(t) \right]dt
	+\lambda_a^T\left\{x_a-x(t_a) \right\} + \lambda_b^T z\left(x(t_b\right).
\end{align}


\subsection{Prinzip der Herleitung notwendiger Bedingungen}

\subsection{Notwendige Bedingungen für Optimallösung}

\subsection{Numerische Lösung am Beispiel "`Fall C und fester Endzeit $t_b$"'}
\begin{itemize}
  \item Zur Lösung der "`Endwertaufgabe"' $\dot{\psi}=\nabla_x \Ham$ mit gegeben Endwert $\psi(t_b)$ und unter der Annahme, dass $x(t)$ und $u(t)$ gegeben sind.\\
  		Wir setzen $t=t(\tau)=t_b+t_a-\tau$ und mit 
  		\begin{align*}
  			\Psi(\tau) & := \psi(t(\tau)) = \psi(t_b+t_a-\tau)
  		\end{align*}
  		erhält man 
  		\begin{align*}
  			\dot{\Psi}(\tau) & = \frac{d\psi(t(\tau))}{d\tau} = \left.\frac{d\psi(t)}{dt}\right|_{t=t(\tau)}\frac{dt}{d\tau}\\
  			& = \left.-\frac{d\psi(t)}{dt}\right|_{t=t(\tau)}=\left.-\dot{\psi}(t)\right|_{t=t(\tau)}\\
  			& = \left.\nabla_x \Ham(x(t),u(t),\psi(t),t)\right|_{t=\tau}
  		\end{align*}
  		TODO BILD\\
  		Damit ist die Anfangswertaufgabe 
  		\begin{align*}
  			\dot{\Psi}(\tau) & = \nabla_x \Ham(x(t_b+t_a-\tau),u(t_b+t_a-\tau),\Psi(\tau),t_b+t_a-\tau)
  		\end{align*}
  		mit dem \ac{AW} $\Psi(t_a)=\psi(t_b)$ zu lösen für $\tau\in[t_a,t_b]$ und es gilt
  		\begin{align*}
  		\psi(t) & = \Psi(t_b+t_a-\tau).
  		\end{align*}
  \item Zur Lösung des nichtlinearen Gleichungssystems $\nabla_u \Ham =0$ unter der Verwendung des NEWTON-Verfahrens
\end{itemize}

\subsection{Anwendung zur Umformung von Optimierungsproblemen am Beispiel des \NoCaseChange{\acl{LQR}}-Problems}
Minimiere das Gütefunktional
\begin{align*}
	J  & = \frac12 x^T(t_b)Gx(t_b)+\frac12\int\limits_{t_a}^{t_b}\left[x^T(t)Q(t)x(t)+u^TR(t)u(t)\right]dt
\end{align*}
bei dem Prozess 
\begin{align*}
	\dot{x}(t) & = A(t)x(t)+B(t)u(t)
\end{align*}
mit den gegebenen Werten $x(t_a)=x_a$, $t_a$, $x_a$, $t_b$, wobei der Zustandswert zur Endzeit $x(t_b)$ frei ist. Weiterhin sollen $G\ge 0$, $Q(t)\ge 0$ (semipositiv defintit), $R(t)>0$
(positiv definit) und $Q(t)$, sowie $R(t)$ stetig differenzierbar sein.

\section{Maximumprinzip von Pontojaju}

\chapter{Optimale Zustandsrückführung}
\section{Zeitdiskrete Optimalsteuerung bei endlichem Zeithorizont}
Es wird der Prozess
\begin{align}
	x[\kappa + 1] & = f\left(x[\kappa], u[\kappa] \right)
\end{align}
betrachtet mit $x[\kappa]\in\mathbb{R}^n$, $u[\kappa]\in\mathbb{R}^m$, dem Prozesszustand $x[\kappa]:=x(\kappa T_A)$ mit der Abtastzeit $T_a$ zu
diskreten Zeitpunkten $\kappa=0,1,\ldots,k$ und gegebenem Anfangszustand $x[0]$.

Das Kostenfunktional wird mit
\begin{align}
	K\left(x[0];u[0],..,u[k-1] \right) & := \sum\limits_{\kappa=0}^{k-1}\underbrace{c\left(x[\kappa],u[\kappa] \right)}_{\begin{tabular}{c}inkrementale
	Kosten\\ im Schritt $\kappa$\end{tabular}} + \underbrace{K_k(x[k])}_{\begin{tabular}{c}Endwertkosten\end{tabular}}
\end{align}
angenommen.\\
Die Aufgabe besteht in der Minimierung des Kostenfunktionals
\begin{align}
	\min\limits_{u[0],\ldots,u[k-1]}K\left(x[0];u[0],\ldots,u[k-1]\right),
\end{align}
mit der optimalen Steuerfolge $u^{\ast} [0], \ldots , u^{\ast} [k-1]$ und der optimalen Zustandstrajektorie $x^{\ast} [0], \ldots , x^{\ast} [k-1]$.
\begin{figure}[htb]
	\centering
	\begin{tikzpicture}[auto, >=latex'] 
	\node [input] () {};
\end{tikzpicture}

	\caption{Darstellung des k-stetigen Entscheidungsprozesses}
	\label{fig:kap_3_entscheidungsprozess}
\end{figure}
Es werden die minimalen Restkosten im Schritt $\kappa=0,\ldots,k$ wie folgt definiert 
\begin{align}
	V\left(x[\kappa],\kappa \right) & := \min\limits_{u[\kappa],\ldots,u[k-1]}\left\{\sum\limits_{i=\kappa}^{k-1} c\left(x[i],u[i] \right) +
	K_k(x[k])\right\}.
\end{align}
Man beachte, dass $V\left(x[k],k\right)=K_k(x[k])$ ist und dann die Minimalkosten
\begin{align*}
	V\left(x[0],0 \right) & = \min\limits_{u[0],\ldots,u[k-1]} K\left(x[0];u[0],\ldots,u[k-1] \right)
\end{align*}
sind. Es gilt für $\kappa=k-1,\ldots,0$
\begin{align}
	V\left(x[\kappa],\kappa \right) & = \min\limits_{u[\kappa],\ldots,u[k-1]}\left\{ c\left(x[\kappa],u[\kappa] \right)+\sum\limits_{i=\kappa+1}^{k-1} \ldots
	\right\}\\
	& = \min\limits_{u[\kappa]}\left\{\min\limits_{u[\kappa + 1],\ldots,u[k-1]}\left(c\left(x[\kappa],u[\kappa] \right) + \sum\limits_{i=\kappa +
	1}^{k-1} \ldots\right) \right\}\\
	& = \min\limits_{u[\kappa]}\Bigg\{c\left(x[\kappa], u[\kappa] \right) + \min\limits_{u[\kappa
	+1],\ldots,u[k-1]}\underbrace{\sum\limits_{i=\kappa+1}^{k-1}\ldots}_{V\left(x[\kappa+1],\kappa+1 \right)} \Bigg\}\\
	& = \min\limits_{u[\kappa]}\left\{c\left(x[\kappa],u[\kappa] \right)+V\left(f\left(x[\kappa],u[\kappa] \right),\kappa+1 \right)
	\right\}.\label{eqn:kap_3_dp}
\end{align}
Gleichung \eqnref{eqn:kap_3_dp} entspricht der Gleichung der \ac{DP} und kann in Kurzform wie folgt dargestellt werden
\begin{align}
	V(x,\kappa) & = \min\limits_u\left(c(x,u)+V\left(f(x,u),\kappa +1 \right) \right)
\end{align}
\begin{remark}\hspace{1mm}
\begin{itemize}
  \item Optimale Steuerfolge $u^{\ast}[\kappa],\ldots,u^{\ast}[k-1]$ zum Erreichen der minimalen Restkosten $V\left(x[\kappa],\kappa \right)$ ist allein von $x[\kappa]$ abhängig, also
  ist die Kenntnis von $u[0],\ldots,u[k-1]$ nicht erforderlich.
  \item Es gilt das Optimalitätsprinzip von Bellmann:\\
  	Sei $u^{\ast}[\kappa],\ldots,u^{\ast}[k-1]$ optimale Steuerfolge für den Prozess mit Anfangszustand $x[\kappa]$, dann ist $u^{\ast}[\kappa+1],\ldots,u^{\ast}[k-1]$ optimale
  	Steuerfolge für den Prozess mit dem Anfangszustand $f\left(x[\kappa],u^{\ast}[\kappa] \right)$.
\end{itemize}
\end{remark}
Anwendung der Gleichung der \ac{DP} zur Bestimmung einer optimalen Steuerfolge

Rekursive Rückwärtsrechnungen
\begin{enumerate}[label=(S\arabic*)]
  \item Berechne
  \begin{align*}
  	V\left(x[k-1],k-1 \right) & = \min\limits_{u[k-1]}\left\{c(x[k-1],u[k-1]) + K_k\left(f(x[k-1],u[k-1]) \right) \right\}
  \end{align*}
  sowie Funktion
  \begin{align*}
  	\pi_{k-1} & := x[k-1]\mapsto u^{\ast}[k-1],
  \end{align*}
  d.h. 
  \begin{align*}
  	\pi_{k-1}\left(x[k-1] \right) & = u^{\ast}[k-1].
  \end{align*}
  \item Für $\kappa=k-2,\ldots,0$ berechne 
  \begin{align*}
  	V\left(x[\kappa],\kappa \right) & = \min\limits_{u[\kappa]}\left\{c\left(x[\kappa],u[\kappa] \right) + V\left(f(x[\kappa],u[\kappa],\kappa +1) \right) \right\},
  \end{align*}
  sowie
  \begin{align*}
  	\pi_{\kappa} & := x[\kappa]\mapsto u^{\ast}[\kappa].
  \end{align*}
\end{enumerate}
Rekursive Vorwärtsrechnung
\begin{enumerate}[label=(S\arabic*),resume] 
  \item Zu gegebenem $x[0]$ bestimme für $\kappa=0,1,\ldots,k-1$
  \begin{align*}
  	u^{\ast}[\kappa] & = \pi_{\kappa}(x[\kappa]).
  \end{align*}
\end {enumerate} 

\begin{remark}\hspace{1mm}
\begin{itemize}
  \item $\pi$ ist die Zustandsrückführung. Es ist die Prozessgleichung notwendig. Vor- und Rückwärtsrechnung durchführen, danach sind diese für beliebige Anfangswerte
  anwendbar. Die Rückführungen bleiben. Bei Änderung von einem $u[\kappa]$ muss nur neu eingesetzt werden, daher ist dieser Algorithmus sehr effizient.
  \item \ac{NB} der Form $u[\kappa]\in U_{\kappa}\subset\mathbb{R}^m$, $x[\kappa]\in X_{\kappa}\subset\mathbb{R}^n$ sind möglich.\\
  Speziell sind zum Beispiel Endwertbedingungen der Form $x[k]=x_{end}$ mit gegebenen $x_{end}$.
  \item Zeitvariante Prozesse $x[\kappa+1]=f(x[\kappa],u[\kappa],\kappa)$ und zeitvariante inkrementale Kosten $c(x[\kappa],u[\kappa],\kappa)$ sind möglich. Dann ist Kostenfunktion der
  \ac{DP}
  \begin{align*}
  	V(x,\kappa) & = \min\limits_u\left(c(x,u,\kappa)+V\left(f(x,u,\kappa), \kappa +1 \right) \right).
  \end{align*}
\end{itemize}
\end{remark}
\begin{exmp}\label{exmp:kap_3_zeitdis_opt_1}
Der Prozess ist gegeben mit
\begin{align*}
	x[\kappa + 1] & = \frac12\left(x[\kappa]+u[\kappa] \right)
\end{align*}
und der Anfangsbedingung $x[0]=4$, sowie dem Zeithorizont $k=3$. Der Endwert ist gegeben mit $x[3]=0$ und das Kostenfunktional mit
\begin{align}
	K\left(x[0];u[0],u[1],u[2] \right) & = \sum\limits_{\kappa=0}^2\left(x[\kappa]^2+u[\kappa]^2 \right).
\end{align}
Für die Berechnung der optimalen Steuerfolge und Zustandstrajektorie wird als erstes die Rückwärtsrechnung schrittweise durchgeführt. 
\begin{enumerate}[label=(S\arabic*)]
  \item Mit $k=3$ und keinen Endwertkosten $K_3$, da Endwert mit $x[3]=0$ gegeben, ergibt sich
\begin{align*}
	V\left(x[k-1],k-1\right) & = \min\limits_{u[k-1]}\left\{c\left(x[k-1],u[k-1]\right) + K_k\left(f(x[k-1],u[k-1]) \right) \right\}\\
	V\left(x[2],2\right) & = \min\limits_{u[2]}\left\{x[2]^2+u[2]^2\bigg| x[3]=0 \right\}\\
	x[3] & = \frac12 \left(x[2] + u[2] \right)\quad \Rightarrow\quad u[2] = -x[2]\\
	V\left(x[2],2\right) & = x[2]^2 + \left(-x[2]\right)^2 = 2x[2]^2
\end{align*}
Damit ist der optimale Steuerwert $u^{\ast}[2]=-x[2]$, $V\left(x[2],2\right)=2x[2]^2$ und die Zustandsrückführung
$\pi_2\left(x[2]\right)=-x[2]=u^{\ast}_2[2]$.
  \item Mit $\kappa=k-2=1$ lässt sich
  \begin{align*}
  	V\left(x[\kappa],\kappa\right) & = \min\limits_{u[\kappa]}\left\{c\left(x[\kappa],u[\kappa]\right)+V\left(f\left(x[\kappa],u[\kappa]
  	\right),\kappa+1 \right) \right\}\\ 
  	V\left(x[1], 1 \right) & = \min\limits_{u[1]}\underbrace{\left\{x[1]^2+u[1]^2 + V\left(x[2], 2 \right) \right\}
  	}_{=:\nu_1}\\
  	& = \min\limits_{u[1]}\underbrace{\left\{x[1]^2+u[1]^2 + 2\left(\frac12\left(x[1]+u[1] \right) \right)^2 \right\}
  	}_{=:\nu_1}\\
  	\frac{\d \nu_1}{\d u[1]} & = x[1]+3u[1]=0\quad \Rightarrow\quad u[1]=-\frac{x[1]}{3}\\
  	\left.\frac{\d^2 \nu_1}{\d u[1]^2}\right|_{u[1]} & = 3 > 0 \quad \Rightarrow\quad\text{Minimum!}\\
  	V\left(x[1], 1 \right) & = x[1]^2 + \left(-\frac13x[1]\right)^2 + 2\left(\frac12\left(x[1] -\frac13x[1] \right)\right)^2=\frac43x[1]^2
  \end{align*}  
  ermitteln. Damit ist $u^{\ast}[1]=-\frac13x[1]$, $V\left(x[1],1\right)=\frac43x[1]^2$ und die Zustandsrückführung
  $\pi_1\left(x[1]\right)=-\frac13x[1]=u^{\ast}_1[1]$.\\
  Weiter wird mit $\kappa=k-3=0$
  \begin{align*}
  	V\left(x[\kappa],\kappa\right) & = \min\limits_{u[\kappa]}\left\{c\left(x[\kappa],u[\kappa]\right)+V\left(f\left(x[\kappa],u[\kappa]
  	\right),\kappa+1 \right) \right\}\\ 
  	V\left(x[0], 0 \right) & = \min\limits_{u[0]}\underbrace{\left\{x[0]^2+u[0]^2 + V\left(x[1], 1 \right) \right\}
  	}_{=:\nu_0}\\
  	& = \min\limits_{u[0]}\underbrace{\left\{x[0]^2+u[0]^2 + \frac43\left(\frac12\left(x[0]+u[0] \right) \right)^2 \right\}
  	}_{=:\nu_0}\\
  	\frac{\d \nu_0}{\d u[0]} & = \frac23x[0]+\frac83u[0]=0\quad \Rightarrow\quad u[0]=-\frac{x[0]}{4}\\
  	\left.\frac{\d^2 \nu_0}{\d u[0]^2}\right|_{u[0]} & = 8 > 0 \quad \Rightarrow\quad\text{Minimum!}\\
  	V\left(x[0], 0 \right) & = x[0]^2 + \left(-\frac14x[0]\right)^2 + \frac43\left(\frac12\left(x[0] -\frac14x[0] \right)\right)^2=\frac54x[0]^2  
  \end{align*}
  der letzte Wert für die optimale Steuerfolge berechnet. Damit ergibt sich $u^{\ast}[0]=-\frac14x[0]$, $V\left(x[0],0\right)=\frac54x[0]^2$ und die
  Zustandsrückführung $\pi_0\left(x[0],0 \right)=-\frac14x[0]=u^{\ast}_0[0]$.
\end{enumerate}
Mit den Ergebnissen der Rückwärtsrechnung kann mittels der Vorwärtsrechnung die optimale Steuerfolge und
Zustandstrajektorie berechnet werden
\begin{align*}
& && & x^{\ast}[0] & = 4\\
u^{\ast}[0] & = -\frac14\cdot 4 = -1 && \Rightarrow  &x^{\ast}[1]& = \frac12\cdot(4-1)=\frac32\\
u^{\ast}[1] & = -\frac13\cdot\frac32=-\frac12 && \Rightarrow  &x^{\ast}[2]& = \frac12\cdot\left(\frac32 -\frac12\right)=\frac12\\
u^{\ast}[2] & = -\frac12 && \Rightarrow  &x^{\ast}[3]& = \frac12\cdot\left(\frac12 - \frac12\right)=0.
\end{align*}
\end{exmp}
\begin{exmp}\label{exmp:kap_3_zeitdis_opt_2}
Benutzt den selben Prozess wie \exmpref{exmp:kap_3_zeitdis_opt_1}. Hat aber keine Endwertbedingung, dafür Endwertkosten und zusätzliche
Steuergrößenbeschränkungen $-1\leq u[\kappa]\leq 1$ für $\kappa=0,1,2$ und das Kostenfunktional
\begin{align*}
	K\left(x[0];u[0],u[1],u[2]\right) & = \sum\limits_{\kappa=0}^2\left(x[\kappa]^2+u[\kappa]^2 \right) + x[3]^2.
\end{align*}
Für die Berechnung der optimalen Steuerfolge und Zustandstrajektorie wird als erstes die Rückwärtsrechnung schrittweise durchgeführt. 
\begin{enumerate}[label=(S\arabic*)]
  \item Mit $k=3$ ergibt sich
\begin{align*}
	V\left(x[2],2\right) & = \min\limits_{u[2]}\left\{c\left(x[2],u[2]\right) + K_3\left(f(x[2],u[2]) \right)\bigg| -1\leq u[2]\leq 1 \right\}\\
	V\left(x[2],2\right) & = \min\limits_{u[2]}\bigg\{\underbrace{x[2]^2+u[2]^2+\left(\frac12\left(x[2]+u[2]\right) \right)^2}_{=:\nu_2} \bigg| -1\leq
	u[2] \leq 1 \bigg\}\\
	\frac{\d \nu_2}{\d u[2]} & = x[2]+5u[2]=0\quad \Rightarrow \quad u[2]=-\frac{x[2]}{5}\\
	\left.\frac{\d^2 \nu_2}{\d u[2]^2}\right|_{u[2]} & = 5 > 0 \quad \Rightarrow\quad\text{Minimum!}\\
	u^{\ast}[2] & = \left\{\begin{array}{rl}
	1 & \text{für } x[2]\leq -5\\
	-\frac{x[2]}{5} & \text{für } x[2]\in(-5,5)\\
	-1 & \text{für } x[2]\geq 5
	\end{array} \right.\\
	V\left(x[2],2\right) & = \left\{\begin{array}{rl}
	\frac54x[2]^2+\frac12x[2]+\frac54 & \text{für } x[2]\leq -5\\
	\frac65x[2]^2 & \text{für } x[2]\in(-5,5)\\
	\frac54x[2]^2-\frac12x[2]+\frac54 & \text{für }x[2]\geq 5
	\end{array}
	\right.
\end{align*}
	\item Mit $\kappa = k - 2 = 1$ ergibt sich
\begin{align*}
	V\left(x[1],1\right) & = \min\limits_{u[1]}\left\{c\left(x[1],u[1]\right) + K_2\left(f(x[1],u[1]) \right)\bigg| -1\leq u[1]\leq 1 \right\}\\
	V\left(x[1],1\right) & = \left\{\begin{array}{l}
	\min\limits_{u[1]}\bigg\{\underbrace{x[1]^2+u[1]^2+\frac54x[2]^2+\frac12x[2]+\frac54}_{=:\nu_{1.1}} \bigg| -1\leq u[1]\leq 1\bigg\}\\
	\min\limits_{u[1]}\bigg\{\underbrace{x[1]^2+u[1]^2+\frac65x[2]^2}_{=:\nu_{1.2}} \bigg| -1\leq u[1]\leq 1\bigg\}\\
	\min\limits_{u[1]}\bigg\{\underbrace{x[1]^2+u[1]^2+\frac54x[2]^2-\frac12x[2]+\frac54}_{=:\nu_{1.3}} \bigg| -1\leq u[1]\leq 1\bigg\}
	\end{array}\right.
\end{align*}
\end{enumerate}
Analog wie in \exmpref{exmp:kap_3_zeitdis_opt_1} weiterrechnen. Durch die Steuergrößenbeschränkung müssen für jedes $_nu_1,i$ mit $i=1,\ldots,3$
jeweils drei Fuktionen ausgewertet werden und im nächsten Schritt wieder weitere drei pro Funktion.
\end{exmp}

\subsection{Numerische Durchführung des Algorithmus der Dynamsichen Programmierung}
Sei $x[\kappa+1] = f\left(x[\kappa],u[\kappa] \right)$ mit $x[\kappa]\in\mathbb{R}^1$, $u[\kappa]\in\mathbb{R}^1$ und das Kostenfunktional
\begin{align*}
	K\left(x[0];u[0],\ldots,u[\kappa-1] \right) & = \sum\limits_{k=0}^{\kappa-1}c\left(x[k],u[k] \right)+K_k([k]).
\end{align*}
Dann kann über eine diskrete Darstellung von $x[\kappa]$ in Form eines Gitters, wie in \figref{fig:kap_3_dyn_prog_xgitter} dargestellt, sowie von
$u[\kappa]$ auf die Restkosten $V(x_i,\kappa)$ geschlossen werden, vgl. \tabref{tab:kap_3_dyn_prog_restkosten}. Mit den diskreten Restkosten kann dann
auf die optimale Steuerfolge $u^{\ast}[x_i,\kappa]$ geschlossen werden, siehe \tabref{tab:kap_3_dyn_prog_optu}. Bei der Betrachtung von $x_1$ im
Schritt $k-1$ werden die Restkosten mit der \textsc{NumPy}-Funktion \lstinline[columns=fixed]{interp} ausgewertet
\begin{figure}[!htb]
	\centering
	\begin{tikzpicture}[auto, >=latex']
	\draw[->] (0,0) -- (5.5,0) node[right] {$\kappa$};
	\draw	(0,0) node[anchor=north] {$0$}
			(1,0) node[anchor=north] {$1$}
			(2.5,-0.15) node[anchor=north] {$\ldots$}
			(4,0) node[anchor=north] {$k-1$}
			(5,0) node[anchor=north] {$k$};
	\draw (0,0.04) -- (0,-0.04);
	\draw (1,0.04) -- (1,-0.04);
	\draw (4,0.04) -- (4,-0.04);
	\draw (5,0.04) -- (5,-0.04);

	\draw[->] (0,0) -- (0,4.5);
	\draw	(0,1) node[anchor=east] {$x_1$}
			(0,2) node[anchor=east] {$x_2$}
			(-0.15,3) node[anchor=east] {$\vdots$}
			(0,4) node[anchor=east] {$x_N$};
	\draw (0.04,0) -- (-0.04,0);
	\draw (0.04,1) -- (-0.04,1);
	\draw (0.04,2) -- (-0.04,2);
	\draw (0.04,4) -- (-0.04,4);

	\draw[dashed] (1,0) -- (1,4);
	\draw[dashed] (4,0) -- (4,4);
	\draw[dashed] (5,0) -- (5,4);

	\draw[dashed] (0,1) -- (5,1);
	\draw[dashed] (0,2) -- (5,2);
	\draw[dashed] (0,4) -- (5,4);

	\draw[->, color=green] (4,1) -- node[anchor=west] {$u_1$} (5,2);
		
\end{tikzpicture}

	\caption{Darstellung der Diskretisierung von $x[\kappa]$ in Gitterform}
	\label{fig:kap_3_dyn_prog_xgitter}
\end{figure}
\begin{align*}
	V\left(x_1, k-1 \right) & = \min\limits_{j=1,\ldots,M} \left\{c\left(x_1,u_j \right)\right.\\
	& \quad + \left.\text{\lstinline[columns=fixed]{interp}}\left(f(x_1,u_j),\left[x_1,\ldots,x_N\right],\left[K_k[x_1],\ldots,K_k[x_N] \right] \right)
	\right\}\\
	& \rightarrow u^{\ast}(x_1,k-1).
\end{align*}
Wenn $x_i$ im Schritt $\kappa$ mit $i=1,\ldots,N$ und $\kappa=k-1,\ldots,0$ betrachtet wird, folgt
\begin{align*}
	V\left(x_i,\kappa \right) & = \min\limits_{j=1,\ldots,N}\left\{c\left(x_i,u_j \right)\right.\\
	&\quad + \left.\text{\lstinline[columns=fixed]{interp}}\left(f(x_i,u_j), \left[x_1,\ldots,x_N \right],\left[V(x_1,\kappa+1),\ldots,V(x_N,\kappa+1)
	\right] \right) \right\}\\
	& \rightarrow u^{\ast}(x_i,\kappa).
\end{align*}
\begin{table}[htb]
\caption{Berechnung der diskretisierten Restkosten $V(x_i,\kappa)$}
\centering
\begin{tabular}{|c|cccc|}
	\hline
	\diagbox{$i$}{$\kappa$} 	& $0$ 	& $\ldots$ 	& $k-1$ 		& $k$\\\hline
	$1$							&		&			& $V(x_1,k-1)$	& $K_k(x_1)$\\
	$\vdots$					&		&			& $\vdots$		& $\vdots$\\
	$N$							&		&			& $V(x_N,k-1)$	& $K_k(x_N)$\\\hline
\end{tabular}
\label{tab:kap_3_dyn_prog_restkosten}
\end{table}
\begin{table}[htb]
\caption{Berechnung der diskretisierten optimalen Steuerfolge $u^{\ast}(x_i,\kappa)$}
\centering
\begin{tabular}{|c|ccc|}
	\hline
	\diagbox{$i$}{$\kappa$} 	& $0$ 	& $\ldots$ 	& $k-1$\\\hline
	$1$							& 		& 			& $u^{\ast}(x_1,k-1)$\\
	$\vdots$					& 		& 			& $\vdots$\\
	$N$							& 		& 			& $u^{\ast}(x_N,k-1)$\\\hline
\end{tabular}
\label{tab:kap_3_dyn_prog_optu}
\end{table}
\begin{exmp}
Der Prozess ist beschrieben durch
\begin{align}
	x[\kappa + 1] & = \frac{1}{2}\left(x[\kappa]+u[\kappa] \right)
\end{align}
mit der Anfangsbedingung $x[0] = 4$ und dem zeithorizont $k=3$. Die Steuergrößenbeschränkungen sind gegeben mit $-1\leq u[\kappa]\leq 1$ für $\kappa =
0,1,2$. Das Kostenfunktional ist gegeben mit
\begin{align}
	K\left(x[0];u[0],u[1],u[2] \right) & = \sum\limits_{\kappa = 0}^2\left(x[\kappa]^2+u[\kappa]^2 \right) + x[3]^2.
\end{align}
\begin{lstlisting}[style=PythonStyle, caption=Numerische Algorithmus der \ac{DP} am \exmpref{exmp:kap_3_zeitdis_opt_2}, label=code:kap_3_sec_1_num_dp] 
import numpy as np

# Rueckwaertsrechnung
x0 = 4
K = 3
uGitter = np.arange(-1, 1.1, 0.1)
xGitter = np.arange(0, 4.1, 0.1)
nu = len(uGitter)
nx = len(xGitter)

V = np.zeros([nx, K+1])
V[:, K] = xGitter**2
U = np.zeros([nx, K])

for k in np.arange(K, 0, -1)-1:
    for i in range(nx):
        Kosten = np.array([])
        for j in range(nu):
            xFolge = (xGitter[i] + uGitter[j])/2
            tKosten = xGitter[i]**2 + uGitter[j]**2 +\
                np.interp(xFolge,
                          xGitter,
                          V[:, k+1],
                          np.inf,
                          np.inf)
            Kosten = np.append(Kosten, tKosten)
        minval = np.min(Kosten)
        minind = np.argmin(Kosten)
        V[i][k] = minval
        U[i][k] = uGitter[minind]

# Vorwaertsrechnung
x = np.array([x0])
u = np.array([])
for k in range(K):
    u = np.append(u, np.interp(x[k], xGitter, U[:, k]))
    x = np.append(x, (x[k] + u[k])/2.0)

print u
print x
\end{lstlisting}  
Im Skript \ref{code:kap_3_sec_1_num_dp} entspricht $K\hat{=}k$, $nx=\hat{=}N$, $nu\hat{=}M$ und $k\hat{=}\kappa$ und dieser liefert die (im
Rahmen der Diskretisierung) optimale Steuerfolge
\begin{align*}
	u & = \left[u_0, u_1, u_2\right] = \left[-1.00, -0.30, -0.10\right]
\end{align*}
und Zustandstrajektorie
\begin{align*}
	x & = \left[x_0,x_1,x_2,x_3 \right]=\left[4.00,1.50,0.60,0.25 \right].
\end{align*}
\end{exmp}
Der Rechenaufwand des Algorithmus beträgt
\begin{itemize}
  \item $k\cdot M\cdot N$ - Berechnungen von $c(x,u)+V\left(f(x,u),\kappa+1 \right)$ für die \ac{DP},
  \item $k\cdot M^k$ - Berechnungen von $c(x,u)$ und $f(x,u)$
  \item $M^k$ - Berechnungen von $K_k(x)$ \hphantom{ $c(x,u)$ und $f(x,u)$}
  \llap{\smash{\raisebox{\dimexpr0.5\normalbaselineskip+\jot}{$\left.\begin{array}{c}\null\\[\jot]\null\end{array}\right\}\quad$}}}
      \text{\smash{\raisebox{\dimexpr0.5\normalbaselineskip+\jot}{\begin{tabular}{l}
      für vollständige Enumeration\\ bei geg. $x[0]$
      \end{tabular} }}}
\end{itemize}
Minimierungsfunktion ist parallelisierbar.

\section{Das zeitdiskrete \NoCaseChange{\acl{LQR}}-Problem}
Der Prozess wird beschrieben durch
\begin{align}
	x[\kappa + 1] & = Ax[\kappa] + B u[\kappa]
\end{align}
mit gegebenen Anfangszustand $x[0]$.\\
Das Kostenfunktional ist gegeben mit
\begin{align}
	K\left(x[0];u[0],\ldots,u[k-1] \right) & = \sum\limits_{\kappa=0}^{k-1}c\left(x[\kappa],u[\kappa] \right) + K_k(x[\kappa])
\end{align}
wobei $c(x,u):=x^TRx + u^TQu$ und $K_k(x):=x^TX_kx$ ist und $R\geq0$, $Q>0$ und $X_k\geq 0$ gilt.\\
Das \ac{LQR}-Problem wird ausgedrückt durch
\begin{align}
	\min\limits_{u[0],\ldots,u[k-1]}K\left(x[0];u[0],\ldots,u[k-1] \right).
\end{align}
\begin{satz}{Hilfssatz über quadratische Formen}\label{satz:kap_3_quad_form}
Seien $R$, $Q$ symmetrische Matrizen und $Q>0$, dann gilt
\begin{align}
	\min\limits_u\left\{\begin{bmatrix}
	x^T	& u^T
	\end{bmatrix}\begin{bmatrix}
	K	&	S^T\\ S	&	Q
	\end{bmatrix}\begin{bmatrix}
	x\\ u
	\end{bmatrix} \right\} & = x^T\left(R-S^TQ^{-1}S \right)x
\end{align}
und das Minimum wird angenommen für $u=-Q^{-1}Sx$.
\end{satz}
\begin{proof}
Es gilt 
\begin{align}
	\MoveEqLeft x^TRx + x^TS u + u^TS x + u^TQ u\\ 
	& = x^T\left(R-S^TQ^{-1}S \right)x + \left(u^T + x^TS^TQ^{-1} \right)\underbrace{Q\left(u + Q^{-1}Sx \right) }_{=:y}\\
	& \geq x^T \left(R-S^TQ^{-1}S \right)x,
\end{align}
da $y^TQy\geq 0\ \forall y$ ($Q^{-1}$ existiert, da $Q>0$ und positiv definit). Für $u=-Q^{-1}Sx$ gilt "`="'.
\end{proof}
\begin{satz}
Das \ac{LQR}-Problem wird gelöst durch $u^{\ast}[\kappa]=F_{\kappa}x[\kappa]$ mit der zeitvarianten Rückführung 
\begin{align*}
	F_{\kappa} & = -\left(Q + B^TX_{\kappa + 1}B \right)^{-1}B^TX_{\kappa + 1}A
\end{align*}
und der Riccatischen Rückwärts-Differentialgleichung
\begin{align}
	X_{\kappa} & = R + A^TX_{\kappa + 1}A-A^TX_{\kappa + 1} B\left(Q+B^TX_{\kappa + 1}B \right)^{-1}B^TX_{\kappa + 1} A 
\end{align}
mit $\kappa = k-1, k-2,\ldots,0$.
\end{satz}
\begin{proof}
	Es gilt $V(x,k)=x^TX_kx$. Mit der Gleichung der \ac{DP} und \satzref{satz:kap_3_quad_form} folgt für $\kappa = k$
	\begin{align*}
		V(x,\kappa - 1) & = \min\limits_u\left\{c(x,u)+V\left(f(x,u),\kappa \right) \right\}\\
		& = \min\limits_u\left\{x^TRx + u^TQu + V\left(Ax + Bu, \kappa \right) \right\}\\
		& = \min\limits_u\underbrace{\left\{x^TRx + u^TQu + (Ax+Bu)^TX_{\kappa}(Ax+Bu) \right\}}_{\geq 0}\\
		& = \min\limits_u\left\{\begin{bmatrix}
		x^T & u^T
		\end{bmatrix}\begin{bmatrix}
		R+A^TX_{\kappa}A	& A^TX_{\kappa} B\\
		B^TX_{\kappa} A		& Q+B^TX_{\kappa}B
		\end{bmatrix}\begin{bmatrix}
		x\\ u
		\end{bmatrix} \right\}\\
		& = x^T\underbrace{\left(R+A^TX_{\kappa}A - A^TX_{\kappa}B\left(Q+B^TX_{\kappa}B \right)^{-1}B^TX_{\kappa}A \right)}_{X_{\kappa +1}}x\\
		& = x^TX_{\kappa +1}x \geq 0,
	\end{align*}
	weil gilt
	\begin{align*}
		\underbrace{y^T\left(Q+B^TX_{\kappa}B \right)y}_{> 0\ \forall y\neq 0} & =
		\underbrace{y^TQy}_{>0}+\underbrace{\underbrace{y^TB^T}_{=z^T}X_{\kappa}\underbrace{By}_{=z} }_{\geq 0}
	\end{align*}
	und die Minimierung wird angenommen für $u = \underbrace{-\left(Q+B^TX_{\kappa}B \right)^{-1}BX_{\kappa}A}_{=F_{\kappa - 1}}x$ also
	$u^{\ast}[\kappa - 1]=F_{\kappa - 1}x[\kappa - 1]$. Weiter gilt $X_{\kappa - 1}\geq 0$ (positiv semidefinit).\\
	Analog gilt die Durchführung für $\kappa=k-1,\ldots,1$.\\
	Im Kern des Beweises ist die Nutzung der Gleichung der \ac{DP} anwendbar.
\end{proof}
\begin{remark}\hspace{1mm}
\begin{itemize}
  \item $u^{\ast}[\kappa]=F_{\kappa}x[\kappa]$ kann als zeitvariante lineare Zustandsrückführung interpretiert werden. 
  \item Eine analoge Herleitung für zeitvariante Prozess
  \begin{align*}
  	x[\kappa + 1] & = A_{\kappa}x[\kappa] + B_{\kappa}u[\kappa]
  \end{align*}
  und zeitvariante Gewichtungen $R_{\kappa}$ und $Q_{\kappa}$ ist möglich.
\end{itemize}
\end{remark}
Es ist eine Übung unter \picref{sec:uebung_zeitdiskretes_lqr_problem}{Das zeitdiskrete LQR-Problem} im Anhang zu finden.

\subsection{Übergang zu unendlichem Zeithorizont}
Das Kostenfunktional ist gegeben mit
\begin{align}
	K\left(x[0];u[0],u[1],\ldots \right) & := \sum\limits_{\kappa=0}^{\infty}c\left(x[\kappa],u[\kappa] \right),
\end{align}
wobei $c(x,u):=x^TRx + u^TQu$ ist und $R\geq 0$ und $Q>0$ gilt.\\
Das \ac{LQR}-Problem wird ausgedrückt durch
\begin{align*}
	\min\limits_{\left\{u[\kappa]|\kappa=0,1,\ldots \right\}} K\left(x[0];u[0],u[1],\ldots \right).
\end{align*} 
\begin{figure}[htb]
	\centering
	\begin{tikzpicture}[auto, >=latex'] 
	\node [input] () {};
\end{tikzpicture}

	\caption{Interpretation des Lösungszuganges beim Übergang zum unendlichen Zeithorizont}
	\label{fig:kap_3_loesungszugang}
\end{figure}
Man kann zeigen, dass falls $(A,B)$ stabilisierbar ist, die Folge der Lösungen der Riccatischen Rückwärts-Differentialgleichung $X_{\kappa}$ für $\kappa=k-1,k-2,\ldots$ gegen ein $X\geq
0$ konvergiert und somit die
\ac{DARE} 
\begin{align*}
	X & = R + A^TXA - A^TXB\left(Q + B^TXB \right)^{-1}B^TXA
\end{align*}
erfüllt ist. 

\begin{satz}
Sei $(A,B)$ stabilisierbar und $R=\bar{R}^T\bar{R}$, $(A,\bar{R})$ ermittelbar. Dann existiert eine eindeutige, positive semidefinite Lösung $X$ der \ac{DARE}. Weiter ist die
Rückkoppelung $u[\kappa]=F x[\kappa]$ mit $F:=\left(Q+B^TXB \right)^{-1}B^TXA$, stabilisierend, d.h $A+BF$ ist stabil, und führt zum minimalen Wert des Kostenfunktionals.
\end{satz}
Der Beweis dieses Satzes ist in \cite{ludyk1995theoretische} nachzulesen.
\begin{remark}\hspace{1mm}
\begin{itemize}
  \item Bei Übergang zu endlichem Zeithorizont gilt für $\kappa\rightarrow -\infty:\ X_{\kappa}\rightarrow X$ und $F_{\kappa}\rightarrow F$, also ergibt sich eine zeitinvariante
  Zustandsrückführung.
  \item Doe Bestimmung von $X$ erfolgt rekursiv mittels der Riccatischen Rückwärts-Differentialgleichung oder besser durch numerische Lösung der \ac{DARE}.
  \item "`$(A,\bar{R})$ ermiitelbar"' kann derart interpretiert werden, dass alle instabilen Eigenbewegungen im Term $x^TRx$ mit positiver Geschwindigkeit erfallst werden müssen. Die
  Bestimmung eines $\bar{R}$ kann durch \ac{SVD} $R=V\sum V^T$, $\bar{R}:=\sum^{\nicefrac12}V^T$ erfolgen. 
\end{itemize}
\end{remark}

\section{Zeitkontinuierliche Optimalsteuerng bei endlichen Zeithorizont}
\label{sec:3_3_zeitkont_opt_endl}
Der Prozess wird beschrieben durch
\begin{align}
	\dot{x} & = f(x,u,t)
\end{align}
mit $x(t)\in\mathbb{R}^n$, $u(t)\in\mathbb{R}^m$, $t\in[0,T]$ und gegebenen Anfangszustand $x(0)$.\\
Das Kostenfunktion ist gegeben mit
\begin{align}
	K(x(0),u) & = \int\limits_0^T c(x,u,t)dt + K_T(x(T)).
\end{align}
Die Aufgabe besteht darin, eine Steuerung $u^{\ast}$ zu finden, so dass $K\left(x(0,u^{\ast}) \right)$ minimal wird.
\subsection{Lösungsprinzip}
Es wird die kontinuierliche Zeit $t$ diskretisiert, so dass $t=\kappa\cdot h$, mit der Schrittweite $h$, $h>0$, gilt. Damit erfolgt die Approximation durch ein zeitdiskretes Problem. 
\begin{figure}[htb]
	\centering
	\begin{tikzpicture}[auto, >=latex'] 
	\node [input] () {};
\end{tikzpicture}

	\caption{Lösungsprinzip der zeitkontinuierlichen Optimalsteuerung bei endlichem Zeithorizont}
	\label{fig:kap_3_loesungsprinzip}
\end{figure}
Die Anwendung der Gleichung der \ac{DP} 
\begin{align}
	V(x,\kappa) & = \min\limits_{u}\left(c^D(x, u, \kappa) + V\left(f^D(x, u ,\kappa),\kappa+1 \right) \right)
\end{align}
und der Grenzübergang von $h\rightarrow +0$ liefert in kontinuierlicher Zeit ein Analogen zur Gleichung der \ac{DP}.

\subsection{Lösung}
In die Gleichung der \ac{DP} setzt man $V(x, \kappa) := \min\limits_{u(.)|_{[t,T]}} \int\limits_t^T c(x, u, \tau)d\tau + K_T\left(x(T) \right)$, $V\left(f^D(x, u, \kappa),\kappa + 1
\right) := V\left(x(t+h),t+h \right)$ und $c^D(x,u,\kappa):=\int\limits_t^{t+h}c(x,u,\tau)d\tau$. 
\begin{remark}
Differentiation nach variabler oberer Grenze $\frac{\td }{\td y}\int\limits_a^y f(t)dt=f(y)$. Also für $g:=y\mapsto\int\limits_t^y c(x,u,\tau)d\tau$ gilt $g'=y\mapsto c(x,u,y)$ und
$g'(t)=x(x,u,t)$.\\
Es wird definiert $r(x)=o(\gamma(x))$ für $x\rightarrow a\ :\Leftrightarrow\frac{r(x)}{\gamma(x)}\rightarrow 0$ für $x\rightarrow a$ was für den hier betrachtet Fall zu $r(h)=o(h)$ für
$h\rightarrow 0\ :\Leftrightarrow\frac{r(h)}{h}\rightarrow 0$ für $h\rightarrow 0$ führt.
\end{remark}
Für die in $[t,t+h]$ auflaufenden Kosten gilt
\begin{align}
	\underbrace{\int\limits_t^{t+h}c(x,u,\tau)d\tau }_{=g(t+h)} & = \underbrace{\int\limits_{t}^tc(x,u,\tau)d\tau }_{=0}+c(x,u,t)+o(h)\\
	g(t+h) & = g(t) + g'(t)\cdot h + o(h)
\end{align}
und für die minimalen Restkosten zum Zeitpunkt $t+h$ gilt
\begin{align}
	V\left(x(t+h),t+h \right) & = V(x,t)+\frac{\d V(x,t)}{\d x}f(x,u,t)\cdot h + \frac{\d V(x,t)}{\d t}\cdot h + o(h).
\end{align}
Also gilt
\begin{align}
	V(x,t) & = \min\limits_u\left(c(x,u,t)\cdot h + V(x,t) + \frac{\d V(x,t)}{\d x}f(x,u,t)\cdot h + \frac{\d V(x,t)}{\d t}\cdot h + o(h) \right).
\end{align}
Die Subtraktion von $V(x,t)+\frac{\d V(x,t)}{\d t}\cdot h$ und die Division durch $h$ liefert 
\begin{align}
	-\frac{\d V(x,t)}{\d t} & = \min\limits_u\left(c(x,u,t) + \frac{\d V(x,t)}{\d x}f(x,u,t)+\frac{o(h)}{h} \right).
\end{align}
Für $h\rightarrow +0$ ergibt sich die \ac{HJB}-Gleichung 
\begin{align*}
	-\frac{\d V(x,t)}{\d t} & = \min\limits_u\left(c(x,u,t) + \frac{\d V(x,t)}{\d x}f(x,u,t) \right).	
\end{align*}
und es gilt $V(x,T)=K_T(x)$.

Somit ist die \ac{HJB}-Gleichung für $t\in[0,T]$ unter der Randbedingung $V(x,T)=K_T(x)$ zu lösen. Die Minimalkosten sind $V(x(0),0)$ und eine optimale Zustandsrückführung ergibt sich
aus
\begin{align}
	u(x,t) & = \arg\min\limits_u\left(c(x,u,t) + \frac{\d V(x,t)}{\d x}f(x,u,t) \right).
\end{align}
\begin{remark}
Es gilt
\begin{align*}
	\left\{ \begin{array}{ll}
	y^{\ast}:=\min\limits_x f(x)	& : \text{Min. von }f\\
	x^{\ast}:=\arg\min\limits_x f(x)& : \text{Min.-stelle von }f
	\end{array}\right\} & :\Leftrightarrow f\left(x^{\ast}\right) = y^{\ast} \leq f(x) \forall x.
\end{align*}
\end{remark}

\section{Das zeitkontinuierliche \NoCaseChange{\acl{LQR}}-Problem}
Der Prozess wird beschrieben durch
\begin{align}
	\dot{x} & = A(t)x + B(t)u
\end{align}
mit dem gegebenen Anfangszustand $x(0)$. Die Systemmatrizen $A(t)$ und $B(t)$ sind stetig bezüglich der Zeit $t$.\\
Das Kostenfunktional ist gegeben mit
\begin{align}
	K\left(x(0),u \right) & := \int\limits_0^T x^TR(t)x + u^TQ(t)u dt + x(T)^TX_Tx(T),
\end{align} 
wobei $R(t)$ und $Q(t)$ stetig bezüglich der Zeit $t$ sind und $R(t)\geq 0$, $Q(t)>0\ \forall t\in[0,T]$, $X_T\geq 0$ gilt.\\
Das \ac{LQR}-Problem stellt sich wie folgt dar:
\begin{itemize}
  \item[] Finde die Rückführung $F(t)$ so, dass für $u^{\ast}(t):=F(t)x(t):\ K\left(x(0),u^{\ast} \right)$ minimal wird. 
\end{itemize}
\begin{satz}
Das \ac{LQR}-Problem wird gelöst durch die optimale Steuergröße
\begin{align}
	u^{\ast}(t) & = F(t)x(t)\quad \forall t\in[0,T],
\end{align}
mit der Rückführung $F(t)=-Q^{-1}(t)B^T(t)X(t)$ und der Riccatischen Differentialgleichung
\begin{subequations}\label{eqn:kap_3_riccatischedgl}
\begin{align}
	-\dot{X} & = R(t)+X A(t)+A^T(t)X-XB(t)Q^{-1}(t)B^T(t)X
\end{align}
mit der Endbedingung
\begin{align}
	X(T) & = X_T.
\end{align}
\end{subequations}
Dabei gilt $V\left(x(0),0 \right) = x^T(0)X(0)x(0)$.
\end{satz}
\begin{remark}
\eqnref{eqn:kap_3_riccatischedgl} ist mathematisch eine Anfangswertaufgabe (Substitution $\tau:=T-t$), somit ist die Lösung $X$ eindeutig. Mit $X$ ist auch $X^T$ eine Lösung von
\eqnref{eqn:kap_3_riccatischedgl}. Also gilt $X=X^T$.
\end{remark}
\begin{proof}
\begin{enumerate}[label=(\alph*)]
  \item Es muss gezeigt werdem, dass $V(x(t),t)=x^T(t)X(t)x(t)$ gilt.\\
  Linke Seite der \ac{HJB}-Gleichung: Es ist $-\frac{\d V}{\d t} = -x^T\dot{X}x$.
  Rechte Seite der \ac{HJB}-Gleichung: Mit $\frac{\d V}{\d x} = 2x^TX$ gilt
  \begin{multline}
  	\min\limits_{u}\left(x^TRx + u^TQu + \left(x^TX + x^TX \right)\left(Ax + Bu \right) \right)\\
  	= \min\limits_{u}\left(x^T\left(R + XA +A^TX \right)x + x^TXBu + u^TB^TXx + u^TQ u
  	\right)\\
  	= \min\limits_{u}\left(\begin{bmatrix}
  	x^T & u^T
  	\end{bmatrix}\begin{bmatrix}
  	R + XA + A^T X & XB\\ B^TX	& Q
  	\end{bmatrix}\begin{bmatrix}
  	x\\ u
  	\end{bmatrix} \right)\\
  	 = x^T\left(R + XA + A^TX - XBQ^{-1}B^TX \right)x. \label{eqn:kap_3_zeitkont_proof}
  \end{multline} 
  \item Das Minimum \eqnref{eqn:kap_3_zeitkont_proof} wird angenommen für $u = \underbrace{-Q^{-1}B^TX}_{=:F}x$, also $u^{\ast}(t)=F(t)x(t)$.
\end{enumerate}
\end{proof}

\subsection{Übergang zu unendlichem Zeithorizont}
Der Prozess wird beschrieben durch
\begin{align}
	\dot{x} & = A x + B u
\end{align}
mit gegebenen Anfangszustand $x(0)$.\\
Das Kostenfunktional wird angegeben mit
\begin{align}
	K\left(x(0), u \right) & := \int\limits_0^{\infty}x^TRx + u^TQu dt,
\end{align}
wobei $R\geq 0$ und $Q>0$ gilt.\\
Das \ac{LQR}-Problem stellt sich wie folgt dar:
\begin{itemize}
  \item[] Finde die Rückführung $F$ so, dass für $u^{\ast}(t):=Fx(t):\ K\left(x(0),\bm{u}^{\ast} \right)$ minimal wird. 
\end{itemize}
\begin{satz}
Sei $(A,B)$ stabilisierbar und $R=\bar{R}^T\bar{R}$, $(A,\bar{R})$ ermittelbar.\\
Dann exisitert eine eindeutige positive semidefinite Lösung $X$ der \ac{ARE} 
\begin{align*}
	0 & = R+XA+A^TX-XBQ^{-1}B^TX.
\end{align*} 
Weiter ist die Rückkopplung $u^{\ast}(t)=Fx(t)$ mit $F:=-Q^{-1}B^TX$,
stabilisierend und führt zum minimalen Wert des Kostenfunktionals.
\end{satz}

\section{Beispiel der räumlichen Führung (\ac{PN})}
\subsection{Modellbildung}
\begin{figure}[htb]
	\centering
	\begin{tikzpicture}[auto, >=latex'] 
	\node [input] () {};
\end{tikzpicture}

	\caption{Geometrie des Problems der räumlichen Führung}
	\label{fig:kap_3_bsp_rf_geometrie}
\end{figure}
Die Geometrie des Problems der räumlichen Führung  soll vereinfachend in der Ebene betrachtet werden (siehe \figref{fig:kap_3_bsp_rf_geometrie}). In $(r,\sigma)$-Koordinaten wird es
beschrieben durch
\begin{align}
	\begin{split}\label{eqn:kap_3_bsp_rf_rskoord}
		r\dot{\sigma} & = v_1\sin(\varphi - \sigma) - v_2\sin(\gamma - \sigma)\\
		\dot{r} & = v_1\cos(\varphi - \sigma) - v_2\cos(\gamma - \sigma)
	\end{split}
\end{align}
und in $(x,y)$-Koordinaten durch
\begin{align}
	\begin{aligned}\label{eqn:kap_3_bsp_rf_xykoord}
		\dot{x}_2 & = v_2 \cos \gamma  & \qquad \dot{x}_1 & = v_1 \cos \varphi\\
		\dot{y}_2 & = v_2 \sin \gamma  & \qquad \dot{y}_1 & = v_1 \sin \varphi.
	\end{aligned}
\end{align}
Gegeben sind der Gierwinkel $\varphi(t)$, die Geschwindigkeiten $v_1(t)$ und $v_2(t)$ sowie die Anfangswerte $x_1(0)$, $y_1(0)$, $x_2(0)$, $y_2(0)$. Es gelte $v_1(t) < v_2(t)$. Mit
\begin{align}\label{eqn:kap_3_bsp_rf_sigma}
\sigma & = \left\{ \begin{array}{rl}
	\arcsin\frac{y_1-y_2}{r}	& \text{für } x_1\geq x_2\\
	\pi-\arcsin\frac{y_1-y_2}{r} & \text{für } x_1 < x_2
\end{array} \right.
\end{align}
und
\begin{align}
	r & = \sqrt{ (x_1 - x_2)^2 + (y_1 - y_2)^2 } \label{eqn:kap_3_bsp_rf_regelgesetz}
\end{align}
werden die Anfangswerte $r(0)$ und $\sigma(0)$ für \eqnref{eqn:kap_3_bsp_rf_rskoord} berechnet (zum Wertebereich von $\sigma$ siehe \figref{fig:kap_3_bsp_rf_wertebereich}). Die Funktion
$\gamma(t)$ dient als Steuerfunktion. Im Falle der sogenannten Proportionalnavigation wird das Regelgesetz
\begin{align}
	\dot{\gamma} & = a \dot{\sigma}	\label{eqn:kap_3_bsp_rf_regelgesetz}
\end{align}
verwendet, wobei für die Proportionalverstärkung $a\geq 1$ gilt.

Das Annäherungsverhalten kann mit \eqnref{eqn:kap_3_bsp_rf_rskoord} zusammen mit \eqnref{eqn:kap_3_bsp_rf_regelgesetz} simuliert werden (ergibt drei Zustandsvariablen). Soll die
Flugbahn in die Kartenebene gezeichnet werden, so wird man für die Simulation \eqnref{eqn:kap_3_bsp_rf_xykoord} zusammen mit \eqnref{eqn:kap_3_bsp_rf_regelgesetz} (ergibt fünf
Zustandsvariablen), wobei die erforderliche Messgröße $\dot{\sigma}$ entweder aus \eqnref{eqn:kap_3_bsp_rf_rskoord} (weitere zwei Zustandsvariablen) oder ohne Vergrößerung der Anzahl
der Zustandsvariablen wie folgt berechnet wird. Mit
\begin{align}
	\zeta & := \begin{bmatrix}
	y_2 - y_1\\
	x_1 - x_2
	\end{bmatrix},\quad \zeta_0:=\frac{\zeta}{\left\| \zeta \right\|_2},\quad \alpha_1:=\zeta_0^T\begin{bmatrix}
	\dot{x}_1 \\ \dot{y}_1
	\end{bmatrix},\quad \alpha_2:=\zeta_0^T\begin{bmatrix}
	\dot{x}_2\\ \dot{y}_2
	\end{bmatrix}
\end{align}
und wegen $\alpha_1 = v_1 \sin(\varphi - \sigma)$, $\alpha_2 = v_2 \sin(\gamma - \sigma)$ (siehe \figref{fig:kap_3_bsp_rf_geschwindigkeit}), $r = \left\|\zeta \right\|_2$ und
\eqnref{eqn:kap_3_bsp_rf_rskoord} erhält man
\begin{align}
	\dot{\sigma} & = \frac{\zeta_0^T\left( \begin{bmatrix}
	\dot{x}_1 \\ \dot{y}_1
	\end{bmatrix} - \begin{bmatrix} 
	\dot{x}_2 \\ \dot{y}_2
	\end{bmatrix}\right)}{\left\| \zeta \right\|_2}.	\label{eqn:kap_3_bsp_rf_messgroesse}
\end{align}
\begin{figure}[htb]
	\centering
	\begin{tikzpicture}[auto, >=latex'] 
	\node [input] () {};
\end{tikzpicture}

	\caption{$\sigma$-Wertebereich für $x_2=y_2=0$ und $(x_1,y_1)$ auf dem Kreis $\mathcal{K}$}
	\label{fig:kap_3_bsp_rf_wertebereich}
\end{figure}
\begin{remark}
Die Berechnung der Ableitung erfolgt in \eqnref{eqn:kap_3_bsp_rf_messgroesse} ohne die Verwendung von Winkelfunktionen. Alternativ kann $\dot{\sigma}$ auch über
$r\dot{\sigma}=v_1\sin(\varphi - \sigma)-v_2\sin(\gamma - \sigma)$ bestimmt werden. Das dafür nötige $\sigma$ wird mittels \eqnref{eqn:kap_3_bsp_rf_sigma} bestimmt. Das dabei nur die
Einschränkung der Sinus-Funktion auf $\left[-\frac{\pi}{2},\frac{3}{2}\pi \right]$ betrachtet wird und damit bei einer kontinuierlichen Bewegung des Körpers 2 auf einer Kreisbahn um den Körper 1 bei
$\sigma = -\frac{\pi}{2}$ ein Sprung der Höhe $2\pi$ auftritt (\figref{fig:kap_3_bsp_rf_wertebereich}), ist unproblematisch, da dieser Sprung durch die Periodizität der Sinunsfunktion
nicht in den Funktionswerten $\sin(\varphi - \sigma)$ und $\sin(\gamma - \sigma)$ erscheint.
\end{remark}
\begin{figure}[htb]
	\centering
	\begin{tikzpicture}[auto, >=latex'] 
	\node [input] () {};
\end{tikzpicture}

	\caption{Komponenten der Geschwindigkeitsvektoren}
	\label{fig:kap_3_bsp_rf_geschwindigkeit}
\end{figure}

\subsection{Parallele Annäherung als Spezialfall der \ac{PN}}
Für die nacholgenden Betrachtungen wird angenommen, dass $\varphi$, $v_1$ und $v_2$ konstante Größen sind. Dann stellt sich für $a\rightarrow +\infty$ in
\eqnref{eqn:kap_3_bsp_rf_regelgesetz} im Regelkreis der mit dem Index $o$ bezeichnete Arbeitspunkt $\dot{\sigma}_o=0$ ein (siehe \figref{fig:kap_3_bsp_rf_methode}).\\
Aus der ersten Gleichung in \eqnref{eqn:kap_3_bsp_rf_rskoord} folgt dann
\begin{align}
	\gamma_o & = \sigma_o + \arcsin\frac{v_1 \sin(\varphi - \sigma_o)}{v_2},
\end{align}
also insbesondere $\dot{\gamma}_o=0$, und die zweite Gleichung liefert die konstante Annäherungsgeschwindigkeit $\dot{r}_o<0$. Für die Herleitung der Gleichungen für den Lenkregelkreis
reicht es aus, die erste Gleichung in \eqnref{eqn:kap_3_bsp_rf_rskoord} zu betrachten. Wir betrachten kleine Abweichungen $\Delta r$, $\Delta \sigma$, $\Delta\dot{\sigma}$, $\Delta
\gamma$ vom Arbeitspunkt. Die Linearisierung von \eqnref{eqn:kap_3_bsp_rf_rskoord} erfolgt durch Bilden des vollständigen Differentials für beide Seiten am Arbeitspunkt
\begin{align}
	r_o\cdot\Delta\dot{\sigma}+\dot{\sigma}_o\cdot\Delta r & = -v_2\cos(\gamma_o-\sigma_o)\cdot\Delta\gamma - \left(v_1\cos(\varphi - \sigma_o) - v_2\cos(\gamma_o -
	\sigma_o)\right)\cdot\Delta\sigma .
\end{align}
\begin{figure}[htb]
	\centering
	\begin{tikzpicture}[auto, >=latex'] 
	\node [input] () {};
\end{tikzpicture}

	\caption{Methode der parallelen Annäherung}
	\label{fig:kap_3_bsp_rf_methode}
\end{figure}
Bei Berücksichtigung von $\dot{\sigma}_o=0$ und Vergleich der Koeffizienten von $\Delta\sigma$ mit der zweiten Gleichung in \eqnref{eqn:kap_3_bsp_rf_rskoord} erhält man
\begin{align}
r_o\cdot\Delta\dot{\sigma}+\dot{r}_o\cdot\Delta\sigma & = -v_2\cos(\gamma_o - \sigma_o)\cdot\Delta\gamma.
\end{align}
Die Division durch $|\dot{r}_o|$ liefert mit der Restlaufzeit $T_f(t):=\frac{r_o}{|\dot{r}_o|}=\frac{r(0)}{|\dot{r}_o|} - t=t_f - t$ bis zum Endzeitpunkt $t_f$ und der
Verfolgergeschwindigkeit $\beta_2 := v_2\cos(\gamma_o - \sigma_o)\in(0,v_2]$ entlang der Sichtlinie die lineare und wegen des Koeffizienten $T_f(t)$ zeitvariante Differentialgleichung
erster Ordnung
\begin{align}
	T_f\cdot\Delta\dot{\sigma}-\Delta\sigma & = -\beta_2\cdot\Delta\gamma .
\end{align}
Durch Differentiation der Gleichung erhalten wir eine Differentialgleichung für $\Delta\dot{\sigma}_o$
\begin{align}
	T_f\cdot\Delta\ddot{\sigma} + \dot{T}_f\cdot\Delta\dot{\sigma}-\Delta\dot{\sigma} & = -\beta_2\cdot\Delta\dot{\gamma},
\end{align}
woraus wegen $\dot{T}_f=-1$
\begin{align}
	T_f\cdot\Delta\ddot{\sigma}-2\cdot\Delta\dot{\sigma} & = -\beta_2\cdot\Delta\cdot{\gamma}
\end{align}
folgt. Wir schreiben kürzer für Zustandsvariable $x:=\Delta\dot{\sigma}$ und für die Steuergröße $u:=\Delta\dot{\gamma}$ und erhalten die Lösung der Differentialgleichung im Intervall
$[t_0,t]\subset[0,t_f]$
\begin{align}
	x(t) & = \frac{\beta_2\int_{t_0}^t(\tau - t_f)u(\tau)d\tau + (t_0-t_f)^2x(t_0) }{(t-t_f)^2}.
\end{align}
Für die stückweise konstante Steuerfunktion $u(t)=const.$ für $t\in[t_k,t_{k+1})$ erhält man mit den Bezeichnungen $u[k]:=u(t_k)$ und $x[k]:=x(t_k)$ die zeitdiskrete Prozessbeschreibung
\begin{align}\label{eqn:kap_3_bsp_rf_ressys}
	x[k+1] & = \underbrace{\frac{(t_k-t_f)^2}{(t_{k+1}-t_f)^2} }_{A_k}x[k] + \underbrace{\frac{\beta_2\int_{t_k}^{t_{k+1}}(\tau - t_f)d\tau }{(t_{k+1}-t_f)^2} }_{B_k}u[k].
\end{align}
\begin{table}[htb]
\caption{Parameter des zeitvarianten Systems und \ac{LQR}-Regler aus \exmpref{exmp:kap_3_bsp_rf_c2}}
\centering
\begin{tabular}{|c|r|r|r|}
	\hline
	k 	& $A_k$		& $B_k$		& $F_k$	\\\hline
	0	& 1.1688	& -0.0422	& 5.7838\\
	1	& 1.1843	& -0.0461	& 5.7668\\
	2	& 1.2030	& -0.0507	& 5.7470\\
	3	& 1.2258	& -0.0564	& 5.7235\\
	4	& 1.2545	& -0.0636	& 5.6953\\
	5	& 1.2914	& -0.0728	& 5.6606\\
	6	& 1.3409	& -0.0852	& 5.6170\\
	7	& 1.4104	& -0.1026	& 5.5605\\
	8	& 1.5151	& -0.1288	& 5.4845\\
	9	& 1.6906	& -0.1726	& 5.3765\\
	10	& 2.0421	& -0.2605	& 5.2107\\
	11	& 3.0673	& -0.5168	& 4.9214\\
	12	& 16.1787	& -3.7942	& 4.2611\\\hline
\end{tabular}
\label{tab:kap_3_bsp_rf_parameter}
\end{table}
\begin{exmp}\label{exmp:kap_3_bsp_rf_c2}
Aus den Anfangswerten
\begin{align*}
	x_1(t_0) & = 10,\quad y_1(t_0) = 0.33,\quad x_2(t_0) = 0,\quad y_2(t_0)=0
\end{align*}
und den als zeitlich konstant angenommenen Größen
\begin{align}
	v_1(t) & = 0.2,\quad v_2(t) = 0.5,\quad \phi(t)=-\pi\qquad\forall t\in[0,t_f]
\end{align}
erhält man für die Methode der parallelen Annäherung die Anfangswerte
\begin{align}
	\gamma(0) & = 0.049,\quad \sigma(0)=0.033,\quad t_f=13.33.
\end{align}
Betrachtet man das resultierende System \eqnref{eqn:kap_3_bsp_rf_ressys} auf den Zeitintervallen $[t_k,t_{k+1}]$ ($k=0,1,\ldots,12$), so ergeben sich die in
\tabref{tab:kap_3_bsp_rf_parameter} dargestellten Werte für $A_k$ und $B_k$. Für dieses System wird ein zeitvarianter \ac{LQR}-Regler $F_k$ nach Satz ? berechnet, wobei für die
Gewichtungsmatrizen $R=0$, $Q=1$, $X_{12}=100$ gesetzt wird, d.h. neben den Steuergrößen soll allein die Abweichung des Zustandes im Endzeitpunkt vom Sollwert $0$ minimiert werden. Für
eine Anfangsabweichung $x[0]=\Delta\dot{\sigma}[0]=0.05$ vom Sollwert zeigt \figref{fig:kap_3_bsp_rf_verlauf_regelgroesse} den mit $x_{LQR}[k]$ bezeichneten Verlauf der Zustandsgröße
für den \ac{LQR}-Regler im Vergleich zu den mit $x_{PN}$ bezeichneten Verlauf für den Fall, dass die Steuergröße mit einem zeitlich konstanten Regler berechnet wird, wobei hier konkret
der \ac{LQR}-Regler zum Zeitpunkt $k=10$ eingesetzt wird, d.h. es gilt
\begin{align}
	u_{PN}[k] & = F_{10}\cdot x_{PN}[k]. \label{eqn:kap_3_bsp_rf_bsp_konst_regler}
\end{align}
In \figref{fig:kap_3_bsp_rf_verlauf_steuergroesse} sind die zugehörigen Verläufe der Steuergröße dargestellt. Man erkennt, dass der zeitvariante \ac{LQR}-Relger die Zustandsgröße im
Endzeitpunkt $k=13$ nahe an den Zielzustand $0$ bringt, während der zeitlich konstante Regler $F_{10}$ gegenüber dem \ac{LQR}-Regler anfangs eine "`zu kleine"' und am Ende des
Zeithorizontes eine "`zu große"' Verstärkung aufweist.
\end{exmp}
\begin{figure}[htb]
	\centering
	\begin{tikzpicture}[auto, >=latex'] 
	\node [input] () {};
\end{tikzpicture}

	\caption{Verlauf der Regelgröße für das \exmpref{exmp:kap_3_bsp_rf_c2}}
	\label{fig:kap_3_bsp_rf_verlauf_regelgroesse}
\end{figure}
\begin{figure}[htb]
	\centering
	\begin{tikzpicture}[auto, >=latex'] 
	\node [input] () {};
\end{tikzpicture}

	\caption{Verlauf der Steuergröße für das \exmpref{exmp:kap_3_bsp_rf_c2}}
	\label{fig:kap_3_bsp_rf_verlauf_steuergroesse}
\end{figure}
Es sind zwei Übungen unter \picref{sec:uebung_raeumliche_fuehrung}{Beispiel der räumlichen Führung} im Anhang zu finden.

\chapter{\NoCaseChange{\acl{MPR}}}
\section{Einleitung}
Die im \secref{sec:3_3_zeitkont_opt_endl} beschriebene Vorgehensweise zur Lösung eines Optimalsteuerproblems auf Basis der Dynamsichen Programmierung ist selbst für lineare
Prozesse mit quadratischem Kostenfunktional mit einem hohen Aufwand verbunden, wenn Beschränkungen im Steuer- und Zustandsraum auftreten (was die Anwendung des \ac{LQR}-Entwurfs
verhindert) und dabei der Zustands- und Steuerraum keine geringe Dimension aufweisen oder der Optimeirungshorizont nicht klein ist. Bei der numerischen Lösung stellt die
Vorwärtsrechnung dann durch die Fallunterscheidungen hohe Anforderungen an die Rechenleistung und die erforderliche Speicherung der berechneten Zustandsrückführungen
$u_{\kappa}^{\ast}$ wird bei einer ausreichend feinen Diskretisierung des Zustandsraumes im Allgemeinen einen hohen Speicherplatzbedarf verursachen. 

Eine Alternative zu dieser Vorgehensweise ist die Berechnung der optimales Steuergrößen $u^{\ast}[0]$, $u^{\ast}[1]$, $\ldots$, $u^{\ast}[k-1]$ durch eine Lösung des
Optimierungsproblems ? für den gegebenen Anfangszustand $x[0]$ durch numerische Optimierung in Echtzeit. Da hierbei zum Zeitpunkt $\kappa=0$ eine Steuerfolge $u[\kappa]$ für den
gesamten Horizont $\kappa=0,1,\ldots,k-1$ berechnet wird, tritt anders als bei den Zustandsrückführungen $u_{\kappa}^{\ast}$ keine Rückkopplung des tatsächlich erreichten
Prozesszustandes mehr auf. Praktisch wird dann, z.B. aufgrund von Modellunbestimmtheiten oder einwirkenden Störgrößen, eine Abweichung der erreichten Zustände von der bei der
Optimeierung berechneten optimalen Zustandstrajektorie auftreten. Man wird deshalb die Lösung des Optimierungsproblems ? in jedem Zeitschritt mit dem aktiellen Prozesszustand als
Anfangszustand wiederholen und somit eine Rückführung, also Regelung, erreichen. Dieser Lösungszugang ist unter der Bezeichnung \ac{MPR} bekannt und soll im Weiteren einführend für
lineare Mehrgrößenprozesse mit Beschränkungen im Steuer- und Zustandsraum dargestellt werden.

\section{Formulierung der Regelungsaufgabe}
\subsection{Grundprinzip der prädiktiven Regelung und Begriffe}
\label{subsec:grundprinzip_mpr}
Gegeben sei der lineare zeitinvariante Prozess
\begin{align}
	\begin{split}\label{eqn:kap_4_zustandsraummodell}
	x[\kappa+1] & = A x[\kappa] + B u[\kappa] \\
	y[\kappa] & = Cx[\kappa]
	\end{split}
\end{align}
in diskreter Zeit. Mit $\kappa=0,1,2,\ldots$ werden die diskreten Zeitpunkte bezeichnet und $k$ bezeichnet den \textit{aktuellen Zeitpunkt}. Die Steuergrößen $u[\kappa]$ seien für
$\kappa < k$ und die Prozessausgänge $y[\kappa]$ für $\kappa \le k$ bekannt. Die Aufgabe besteht in der Bestimmung der aktuellen Steuergröße $u[\kappa]$. Dazu bezeichne
\begin{align}
	y_f[\kappa +i|k] & := CAîx[k]\qquad (i\ge 0)
\end{align}
die freie Bewegung des Prozessausganges ab dem Zeitpunkt $k$ und 
\begin{align}
	\hat{y}[k+1|k] & := CAîx[k]+C\sum\limits_{j=0}^{i-1}A^{i-j-1}B\hat{u}[k+j|k]\qquad (i=1,2,\ldots,H_p) \label{eqn:kap_4_praediktion}
\end{align}
die \textit{Prädiktion des Prozessausganges} über dem \textit{Prädiktionshorizont} der Länge $H_p$.
\begin{exmp}[Prädiktionshorizont $H_p=2$]
\begin{align*}
	x[k+2] & = Ax[k+1]+Bu[k+1]\\
	& = A^2x[k]+ABu[k]+Bu[k+1]\\
	y[k+2] & = CA^2x[k]+\underbrace{C\left( ABu[k]+Bu[k+1] \right)}_{C\sum\limits_{j=0}^1A^{1-j}Bu[k+j]}
\end{align*}
\end{exmp}
Weiter wird der \textit{Steuerhorizont} der Länge $H_u$ definiert, es gelte $H_p\ge H_u \ge 1$ und die Steuergröße verändere sich ab dem Zeitpunkt $k+H_u$ nicht mehr\footnote{Die
Bedingung $H_p\ge H_u$ sichert, dass die Auswirkungen eines Steuereingriffs aud den Prozessausgang über einen ausreichend großen Horizont $H_o-H_u$ berücksichtigt werden. Praktisch
gilt aus diesem Grund sogar oft $H_p \gg H_u$.}, d.h. es sei
\begin{align}
	\hat{u}[k+j|k] & = \hat{u}[k+H_u-1|k]\qquad (H_u\le j\le H_p-1).\label{eqn:kap_4_steuerhorizont}
\end{align}
\begin{figure}[htb]
	\centering
	\begin{tikzpicture}[auto, >=latex'] 
	\node [input] () {};
\end{tikzpicture}

	\caption{Grundprinzip der prädiktiven Regelung}
	\label{fig:kap_4_grundprinzip_mpr}
\end{figure}
In \figref{fig:kap_4_grundprinzip_mpr} sind die bekannten Verläufe des Prozessausganges und der Steuergröße bis zum Zeitpunkt $k$ zu sehen. Wenn ab dem Zeitpunkt $k$ die Steuergrößen
$\hat{u}[k+i-1|k]$ ($i=1,2,\ldots,H_p$) auf den Prozess gegeben würden, so würde sich unter der Annahme eines Haltegliedes 0. Ordnung für die Steuergrößen gemäß
\eqnref{eqn:kap_4_praediktion} in den Abtastzeitpunkten der Prozessausgang $\hat{y}[k+i|k]$ ergeben. Durch ein numerisches Optimeirungsverfahren wird eine Steuerfolge so bestimmt, dass
vorgegebene Forderungen erfüllt werden. Beispielhaft für solche Forderungen sind in \figref{fig:kap_4_grundprinzip_mpr} konstante Schranken $\underline{u}$ und $\overline{u}$ für die
Steuergröße
\begin{align}
	\underline{u}&\le \hat{u}[k+i-1|k]\le \overline{u} \label{eqn:kap_4_ugnb_u}
\end{align}
und eine konstante obere Schranke $\overline{y}[k+i|k]$ und eine zeitabhängige untere Schranke $\underline{y}[k+i|k]$ für den Prozessausgang
\begin{align}
	\underline{y}[k+i|k]&\le \hat{y}[k+i|k]\le\overline{y}[k+i|k] \label{eqn:kap_4_ugnb_y}
\end{align}
darstellt. Darüber hinaus ist eine \textit{Solltrajektorie} $s[\kappa]$ für den Prozessausgang eingezeichnet. Aus dieser Solltrajektorie und dem Wert des Prozessausganges zum Zeitpunkt
$k$ wird die \textit{Referenztrajektorie} $r[k+i|k]$ abgeleitet, z.B. mit einer gegeben Zeitkonstante $T_c>0$ in der Form
\begin{align}
	r[k+i|k] & := s[k+i]-(s[k]-y[k])e^{-iT_c},
\end{align}
und die Zielstellung verfolgt, den Abstand (mit einem noch zu definierendem Abstandsmaß) des Prozessausganges von der Referenztrajektorie zu minimieren\footnote{Der Prozessausgang wird
nicht (direkt) der Sollwerttrajektorie, sondern der Referenztrajektorie nachgeführt. Die Referenztrajektorie gibt an, wie ein von der Sollwerttrajektorie abweichender Prozessausgang
wieder zur Sollwerttrajektorie zurückkehren soll. Die Verwendung der Referenztrajektorie anstelle der Sollwerttrajektorie zur Vorgabe des Verlaufes des Prozessausganges hat im
Wesentlichen eine praktische Motivation: Da die Regelabweichung $r[k+i|k]-\hat{y}[k+i|k]$ hinsichtlich des Betrages im Allgemeinen kleiner ist als $s[k+i]-\hat{y}[k+i|k]$ kann man
erwarten, dass bei den in der Praxis stets vorliegenden nichtlinearen Prozessen ein um die Referenztrajektorie linearisiertes Prozessmodell eine genauere Prozessbeschreibung darstellt,
als ein Modell, dass durch eine Linearisierung um die Sollwerttrajektorie gewonnen wird.}. Wenn eine optimale Steuerfolge $u^{\ast}[k+i-1|k]$ ($i=0,1,\ldots,H_u-1$) berechnet ist, so
wird das erste Glied der optimalen Steuerfolge als Steuergröße im Schritt $k$ ausgegeben. Im Schritt $k+1$ wird dann die gesamte Prozedur wiederholt. Der Prädiktionshorizont, der wie
aud der Steuerhorizont in seiner Länge unverändert bleibt, wird dabei mit dem aktuellen Zeitpunkt weiterverschoben. Man spricht von einem gleitenden Horizont. Insgesamt ergibt sich die
folgende Vorgehensweise.\\
\textbf{Prinzipieller Algorithmus der Modellprädiktiven Regelung}:
\begin{enumerate}[label=(S\arabic*)]
  \item Minimiere über alle Steuerfolgen $u[k+i|k]$ ($i=0,1,\ldots,H_u-1$) ein Maß für den Abstand zwischen $r[k+i|k]$ und $\hat{y}[k+i|k]$ ($i=1,2,\ldots,H_p$) unter den
  Nebenbedingungen \eqnref{eqn:kap_4_ugnb_u} und \eqnref{eqn:kap_4_ugnb_y}.
  \item Setze für die Steuergröße $u[k]:=u^{\ast}[k|k]$.
  \item Warte einen Zeitschritt, setzte $k:=k+1$ und gehe zu (S1).
\end{enumerate}
Folgende Merkmale einer modellprädiktiven Regelung können hervorgehoben werden.
\begin{itemize}
  \item Vorhandensein eines expliziten Prozessmodells\\
  		Das Prozessmodell dient im Rahmen der Lösung des Optimierungsproblems im Schritt (S1) zur Bestimmung des für eine vorgegebene Steuerfolge $u[k+i|k]$ ($i=0,1,\ldots,H_u-1$)
  		resultierendes Prozessausganges. Das Prozessmodell muss nicht notwendig ein parametrisches Modell sein, sondern kann z.B. auch durch eine Sprungantwort beschrieben sein.
  \item Optimierung in Echtzeit\\
  		Die optimale Steuerfolge, von der jeweils nur das erste Glied als Steuergröße an den Prozess auch tatsächlich ausgegeben wird, wird in jedem diskreten Zeitschritt durch ein
  		numerisches Optimeirungsverfahren in Echtzeit berechnet.
  \item Gleitender Optimierungshorizont\\
  		Im Gegensatz zur Vorgehensweise im Abschnitt ?, bei dem der aktuelle Zeitpunkt in jedem Schritt näher an den Endzeitpunkt heranrückt, gleiten hier der Steuer- und
  		Prädiktionshorizont mit dem aktuellen Zeitpunkt weiter. 
\end{itemize}

\subsection{Prozessbeschreibung}
Für die weiteren Darstellungen wird von den Steuergrößen $u[\kappa]$ zu den Steuergrößendifferenzen
\begin{align}
	\Delta u[\kappa] & := u[\kappa]-u[\kappa -1]
\end{align}
übergegangen, womit sich eine Regelkreisstruktur wie in \figref{fig:kap_4_steuergroessen_differenzen} ergibt\footnote{Dies hat in erster Linie prakische Gründe, denn da der
\ac{MPR}-Regler nur Steuergrößendifferenzen ("`Steuergrößenkorrekturen"') erarbeitet, kann z.B. ein stoßfreies Umschalten zwischen verschiedenen Reglern problemlos realisiert werden und auch eine Regelung um einen Arbeitspunkt
(wie z.B. bei der Regelung nichtlinearer Prozesse) ist einfacher realisierbar.}. Eine mögliche Zustandsraumdarstellung des \ac{MPR}-Prozesses kann aus \eqnref{eqn:kap_4_zustandsraummodell} mit
\begin{align}
	\xi[\kappa] & := \begin{bmatrix}
	x[\kappa]\\ u[\kappa -1]
	\end{bmatrix}
\end{align}
in der Form ("`erweitertes Modell"')
\begin{align}
\begin{split}\label{eqn:kap_4_erweitertes_modell}
	\xi[\kappa+1] & = \begin{bmatrix}
	A & B\\ 0 & I
	\end{bmatrix}\xi[\kappa]+\begin{bmatrix}
	B\\ I
	\end{bmatrix}\\
	y[\kappa] & = \begin{bmatrix}
	C & 0
	\end{bmatrix}\xi[\kappa]
\end{split}
\end{align}
gewonnen werden.
\begin{figure}[htb]
	\centering
	\begin{tikzpicture}[auto, >=latex'] 
	\node [input] () {};
\end{tikzpicture}

	\caption{Übergang von den Steuergrößen zu den Steuergrößendifferenzen}
	\label{fig:kap_4_steuergroessen_differenzen}
\end{figure}
Stellt man nun die Gleichung für die Prädiktion des Zustandsvektors auf
\begin{align}
	\hat{x}[k+i|i] & = Aîx[k]+\sum\limits_{j=0}^{i-1}A^{i-j-1}B\hat{u}[k+j|k]
\end{align}
und ersetzt die Steuergröße $\hat{u}[k+j|k]$ durch die Steuergrößendifferenz
\begin{align}
\begin{split}\label{eqn:kap_4_steuergroessendiff}
	\Delta\hat{u}[k|k] & = \hat{u}[k|k]-u[k-1]\\
	\Delta\hat{u}[k+i|k] & = \hat{u}[k+i|k]-\hat{u}[k+i-1|k]\quad\text{für }i=1,2,\ldots,H_u-1\\
	\Delta\hat{u}[k+i|k] & = 0\quad\text{für }i=H_u-1,\ldots,H_p-1
\end{split}
\end{align}
so erhält man aus \eqnref{eqn:kap_4_zustandsraummodell} die Darstellung des \ac{MPR}-Prozesses
\begin{align*}
\begin{bmatrix}
\hat{x}[k+1|k]\\
\vdots\\
\hat{x}[k+H_u|k]\\
\hat{x}[k+H_u+1|k]\\
\vdots\\
\hat{x}[k+H_p|k]
\end{bmatrix} & = \underbrace{\begin{bmatrix}
A\\ \vdots\\ A^{H_u}\\ A^{H_u+1}\\ \vdots\\ A^{H_p}
\end{bmatrix}}_{\Psi}x[k]+\underbrace{\begin{bmatrix}
B\\ \vdots \\ \sum_{i=0}^{H_u-1}A^iB\\ \sum_{i=0}^{H_u}A^iB\\ \vdots \\ \sum_{i=0}^{H_p-1}A^iB
\end{bmatrix}}_{\Upsilon}u[k-1]\\
&\quad +\underbrace{\begin{bmatrix}
B 						&	\ldots 	& 	0\\
\vdots					&	\ddots	& 	\vdots\\
\sum_{i=0}^{H_u-1}A^iB	& \ldots	& 	B\\
\sum_{i=0}^{H_u}A^iB	& \ldots	& 	AB+B\\
\vdots					&			& 	\vdots\\
\sum_{i=0}^{H_p-1}A^iB	& \ldots	& 	\sum_{i=0}^{H_p-H_u}A^iB
\end{bmatrix}}_{\Theta} \underbrace{\begin{bmatrix}
\Delta\hat{u}[k|k]\\
\vdots\\
\Delta\hat{u}[k+H_u-1|k]
\end{bmatrix}}_{\Delta\mathcal{U}[k]},\\
\begin{bmatrix}
\hat{y}[k+1|k]\\ \vdots\\ \hat{y}[k+H_p|k]
\end{bmatrix}[k]& = \begin{bmatrix}
C	&			&	\\
	& \ddots	&	\\
	&			& C
\end{bmatrix}\begin{bmatrix}
\hat{x}[k+1|k]\\ \vdots\\ \hat{x}{k+H_p|k}
\end{bmatrix},
\end{align*}
die man mit 
\begin{align*}
	\mathcal{Y}[k] & := \begin{bmatrix}
	\hat{y}[k+1|k]\\
	\vdots\\
	\hat{y}[k+H_p|k]
	\end{bmatrix},\ \Delta\mathcal{U}[k]:=\begin{bmatrix}
	\Delta \hat{u}[k|k]\\
	\vdots\\
	\Delta\hat{u}[k+H_u-1|k]
	\end{bmatrix}
\end{align*}
und entsprechender Defintion der Matrizen $\Upsilon$, $\Psi$ und $\Theta$ kürzer in der Form
\begin{align}
\mathcal{Y}[k] & = \underbrace{\Psi x[k]+\Upsilon u[k-1]}_{\text{freie Bewegung}}+\Theta\Delta\mathcal{U}[k] \label{eqn:kap_4_zustandsraummodell_steuerdiff}
\end{align}
schreiben kann.

Mit der \eqnref{eqn:kap_4_zustandsraummodell_steuerdiff} ist bei gegebenen Werten $x[k]$ und $u[k-1]$ die Prädiktion des Prozessausganges als lineare Funktion der
Steuergrößendifferenzen $\mathcal{U}[k]$ dargestellt. Im aktuellen Schritt $k$ ist die im vorhergehenden Schritt $k-1$ an den Prozess ausgegebene Steuergröße $u[k-1]$ bekannt. Der
Zustand $x[k]$ wird im Weiteren ebendalls als bekannt vorausgesetzt. Sollte $x[k]$ nicht direkt messbar sein, so wird $x[k]$ in \eqnref{eqn:kap_4_zustandsraummodell_steuerdiff} durch
einen Schätzwert $\hat{x}[k|k]$ eines Zustandsbeobachters ersetzt. Sollte im \ac{MPR}-Prozess \eqnref{eqn:kap_4_zustandsraummodell} darüber hinaus eine direkt gemessene Störgröße
$d_m[k]$ oder eine durch einen erweiterten Beobachter ermittelte Störgröße mit dem Wert $\hat{d}[k|k]$ auftreten, so kommt auf der rechten Seite von
\eqnref{eqn:kap_4_zustandsraummodell_steuerdiff} zusätzlich ein Summand
\begin{align*}
	\mathcal{D}_m\begin{bmatrix}
	d_m[k]\\ \hat{d}_m[k+1|k]\\ \vdots \\ \hat{d}_m[k+H_p-1|k]
	\end{bmatrix}\quad\text{bzw.}\quad \mathcal{D}\begin{bmatrix}
	\hat{d}[k|k]\\ \hat{d}[k+1|k]\\ \vdots\\ \hat{d}[k+H_p-1|k]
	\end{bmatrix}
\end{align*}
mit den Matrizen $\mathcal{D}$ und $\mathcal{D}_m$ hinzu. Dabei bezeichnen $\hat{d}_m[k+i|k]$ und $\hat{d}[k+i|k]$ die Prädiktion der direkt gemessen bzw. beobachteten Störgröße, die
z.B. unter der Annahme konstanter Störgrößen gemäß $\hat{d}_m[k+i|k]:=d_m[k]$ bzw. $\hat{d}[k+i|k]:=\hat{d}[k|k]$ bestimmt werden ($i=1,\ldots,H_p-1$). Die Modellgleichung
\eqnref{eqn:kap_4_zustandsraummodell_steuerdiff} bleibt in diesem Fall linear in den Steuergrößendifferenzen $\Delta\mathcal{U}[k]$, so dass auch bei auftretenden Störgrößen die im
Weiteren beschriebene Vorgehensweise anwendbar ist.

\subsection{Darstellung des Gütefunktionals und der Beschränkungen für die Steuer- und Regelgrößen}
Die BEstimmung der Steuergrößen $\Delta\hat{u}[k+i|k]$ für den \ac{MPR}-Prozess ($i=0,1,\ldots,H_u-1$) erfolgt so, dass ein vorgegebenes Gütefunktional minimiert wird und dabei
vorgegebene Schranken für die Steuergrößen und für die Regelgrößen eingehalten werden.

Im Folgenden werden drei Klassen von Gütefunktionalen betrachtet, das quadratische Gütefunktional ($\|\cdot\|_2$-Funktional)
\begin{align}
V[k] & = \sum\limits_{i=1}^{H_p}\left\|Q_y[i]\left(\hat{y}[k+i|k]-r[k+i|k]\right)\right\|_2^2+\sum\limits_{i=0}^{H_u-1}\left\|Q_{\Delta u}[i]\Delta\hat{u}[k+i|k]\right\|_2^2
\label{eqn:kap_4_quad_guetefunktional},
\end{align}
das 1-Norm-Gütefunktional ($\|\cdot\|$)
\begin{align}
V[k] & = \sum\limits_{i=1}^{H_p}\left\|Q_y[i]\left(\hat{y}[k+i|k]-r[k+i|k]\right)\right\|_1+\sum\limits_{i=0}^{H_u-1}\left\|Q_{\Delta
u}[i]\Delta\hat{u}[k+i|k]\right\|_1,\label{eqn:kap_4_eins_norm_guetefunktional}
\end{align}
und das Maximum-Norm-Gütefunktional ($\|\cdot\|_{\infty}$-Funktional)
\begin{align}
	V[k] & = \max\limits_{i=1,\ldots,H_p}\left\|Q_{y}[i]\left(\hat{y}[k+i|k]-r[k+i|k] \right)\right\|_{\infty}\label{eqn:kap_4_max_norm_guetefunktional_1}
\end{align}
bzw.
\begin{align}
	V[k] & = \max\limits_{i=0,\ldots,H_u-1}\left\|Q_{\Delta u}[i]\Delta\hat{u}[k+i|k] \right\|_{\infty}.\label{eqn:kap_4_max_norm_guetefunktional_2}
\end{align}
Hierbei sind die Gewichtsmatrizen $Q_{y}[i]$ positiv semidefinit und $Q_{\Delta u}[i]$ positiv definit, oft werden sie sogar als Diagonalmatrizen gewählt. Quadratische Gütefunktionale
sind bereits aus dem Abschnitt ? bekannt und haben wichtige Anwendungen bei statistische Aufgabenstellungen. Das 1-Norm-Gütefunktional tritt z.B. bei der Verbrauchsoptimierung auf,
während Probleme, bei denen der ungünstigste Fall (Worst-Case) betrachtet wird, oft auf Maximum-Norm-Gütefunktional führen.
\begin{exmp}\label{exmp:kap_4_bsp_siso_nb}
Falls für den Prozess \eqnref{eqn:kap_4_zustandsraummodell} bzw. das quadratische Funktional \eqnref{eqn:kap_4_quad_guetefunktional} speziell die Bedingungen 
\begin{align*}
H_u & = H_p,\quad C=I,\quad r[k+i|k]=0,\quad Q_y[i]=Q_y,\quad Q_{\Delta u}[i]=Q_{\Delta u}\ \forall i
\end{align*}
erfüllt sind, so kann \eqnref{eqn:kap_4_quad_guetefunktional} mit $\bar{Q}_x:=Q_y^TQ_y$ und $\bar{Q}_{\Delta u}:=Q^T_{\Delta u}Q_{\Delta u}$ in der Form
\begin{align}
\begin{split}\label{eqn:kap_4_bsp_1_kostenfunktionial}
V[k] & = \sum\limits_{i=1}^{H_p-1}\hat{x}[k+i|k]^T\bar{Q}_{x}\hat{x}[k+i|k]+\sum\limits_{i=0}^{H_p-1}\Delta\hat{u}[k+i|k]^T\bar{Q}_{\Delta u}\Delta\hat{u}[k+i|k]\\
& \quad +\hat{x}[k+H_p|k]^T\bar{Q}_{x}\hat{x}[k+H_p|k]
\end{split}
\end{align}
geschrieben werden. Abgesehen von dem Term $\hat{x}[k|k]^T\bar{Q}_{x}\hat{x}[k|k]$, der aber durch die Wahl der $\Delta\hat{u}[k+i|k]$ ($i=0,\ldots,H_p-1$) nicht beeinflusst wird, ist
das Gütefunktional \eqnref{eqn:kap_4_bsp_1_kostenfunktionial} gleich dem aus dem Abschnitt ? bekannten \ac{LQR}-Funktional (angewendet auf den \ac{MPR}-Prozess).
\end{exmp}
Die Beschränkungen der Regelgröße, der Steuergrößendifferenzen und der Steuergrößen werden nachfolgend durch lineare Nebenbedingungen in der Form
\begin{align}
M[k]\begin{bmatrix}
\mathcal{Z}[k]\\ \Delta\mathcal{U}[k]\\ \mathcal{U}[k]\\ 1
\end{bmatrix} & \le 0\label{eqn:kap_4_lin_nebenbedingungen}
\end{align}
dargestellt.
\begin{exmp}
Gegeben seien ein \ac{SISO}-Prozess, das Gütefunktional
\begin{align*}
	V[k] & = \max\limits_{i=0,\ldots,H_u-1}\left|\Delta\hat{u}[k+i|k] \right|
\end{align*}
und die Nebenbedingungen
\begin{align}
\begin{split}\label{ref:eqn:kap_4_bsp_1_nebenbedingungen}
	\left|\hat{y}[k+i|k]-r[k+H_p|k] \right| & \le \frac{1}{i+1}\left| y[k] - r[k+H_p|k] \right|\quad (i=1,\ldots,H_p)\\
	\hat{y}[k+H_p|k] & = r[k+H_p|k].
\end{split}
\end{align}
Durch das Gütefunktional wird das Maximum der Steuergrößenänderung minimiert. Diese Aufgabenstellung kann für ein mechanisches System dessen Steuergrößen Kräfte oder Momente sind, als
eine Minimierung der sogenannten maximalen Rucks (das ist die Ableitung der Kraft bzw. des Momentes) interpretiert werden.
\begin{figure}[htb]
	\centering
	\begin{tikzpicture}[auto, >=latex'] 
	\node [input] () {};
\end{tikzpicture}

	\caption{Veranschaulichung der Nebenbedingungen zum \exmpref{exmp:kap_4_bsp_siso_nb}}
	\label{fig:kap_4_bsp_nebenbedingungen}
\end{figure}
Die Nebenbedingungen der Aufgabe sind Forderungen an den Verlauf der Regelgröße. Wie in der \figref{fig:kap_4_bsp_nebenbedingungen} veranschaulicht, wird eine fortschreitende Annäherung
an den Wert $r[k+H_p|k]$ der Referenztrajektorie am Ende des Prädiktionshorizontes erzwungen sowie für den Zeitpunkt $k+H_p$ ein exaktes Erreichen dieses Wertes verlangt. Die
eingezeichnete Trajektorie $z^{(1)}[k+i|k]$ ist ein Beispiel für einen diesbezüglich zulässigen Verlauf. Stellt man zusätzlich die Monotonieforderung
\begin{align}
	\hat{y}[k+i+1|k] & \ge \hat{y}[k+i|k],
\end{align}
so darf die Regelgröße den Wert $r[k+H_p|k]$ nicht überschreiten. Der Verlauf $z^{(2)}[k+i|k]$ ist dann in diesem Sinne zulässig.
\end{exmp}
Es ist eine Übung unter \picref{cha:anhang_uebungen}{Übungen} zu finden.

\section{Darstellung der Steuerungsaufgabe als Optimierungsproblem und numerische Lösung}
\subsection{Aufgabe mit $\|\cdot\|_2$-Funktionial ohne Beschränukung der Steuer- und Regelgröße}
\label{subsec:aufgabe_quad_ohne_beschr}
Betrachtet werden Aufgaben, bei denen das Funktional \eqnref{eqn:kap_4_quad_guetefunktional} zu minimieren ist und als Nebenbedingung allein die Prozessgleichung
\eqnref{eqn:kap_4_zustandsraummodell_steuerdiff} auftritt.
Mit den Abkürzungen
\begin{align*}
	\mathcal{Q}_{y}[k] & :=\begin{bmatrix}
	Q_{y}[1]	& 			& 	\\
				& \ddots	&	\\
				&			& Q_{y}[H_p]
	\end{bmatrix},\ \mathcal{Q}_{\Delta\mathcal{U}}[k]:= \begin{bmatrix}
	Q_{\Delta\mathcal{U}}[0]	&			&	\\
								& \ddots	& 	\\
								&			& Q_{\Delta\mathcal{U}}[H_u-1]
	\end{bmatrix}
\end{align*}
und
\begin{align*}
	\mathcal{T}[k] & := \begin{bmatrix}
	r[k+1|k]\\
	\vdots\\
	r[k+H_p|k]
	\end{bmatrix}
\end{align*}
gilt
\begin{align}
	V(k) & = \left\| \mathcal{Q}_{y}\left(\mathcal{Z}[k]-\mathcal{T}[k]\right) \right\|_2^2 + \left\| \mathcal{Q}_{\Delta\mathcal{U}}\Delta\mathcal{U}[k] \right\|_2^2
\end{align}
Mit $\mathcal{E}[k]:=\mathcal{T}[k]-\Psi x[k]-\Upsilon u[k-1]$ wird die Differenz zwischen der Referenztrajektorie $\mathcal{T}[k]$ und der freien Bewegung $\Psi x[k]+\Upsilon u[k-1]$
des \ac{MPR}-Prozesses (nicht des Prozesses!) bezeichnet. Somit gilt
\begin{align}
\begin{split}\label{eqn:kap_4_quad_kostenfunktional_ohne_beschr}
	V[k] & = \left\| \mathcal{Q}_{y}\left(\Theta\Delta\mathcal{U}[k]-\mathcal{E}[k]\right) \right\|_2^2 + \left\| \mathcal{Q}_{\Delta\mathcal{U}}\Delta\mathcal{U}[k] \right\|_2^2\\
	& = \left\| \begin{bmatrix} \mathcal{Q}_y\left(\Theta\Delta\mathcal{U}[k] - \mathcal{E}[k]\right)\\ \mathcal{Q}_{\Delta\mathcal{U}\Delta\mathcal{U}[k]} \end{bmatrix}
	\right\|_2^2
\end{split}
\end{align}
und man erhält die Lösung $\Delta\mathcal{U}^{\ast}[k]$ der unbeschränkten Optimierungsaufgabe
\begin{align}
	\min\limits_{\Delta\mathcal{U}[k]}V[k]
\end{align}
als Quadratmittellösung eines überbestimmten linearen Gleichungssystems
\begin{align}
	\begin{bmatrix}
	\mathcal{Q}_y\Theta\\ \mathcal{Q}_{\Delta\mathcal{U}}
	\end{bmatrix}\Delta\mathcal{U}[k] & \cong \begin{bmatrix}
	\mathcal{Q}_y\mathcal{E}[k]\\ 0
	\end{bmatrix}. \label{eqn:kap_4_quadratmittelproblem}
\end{align}
Da $\mathcal{Q}_{\Delta\mathcal{U}}$ positiv definit ist, ist die Lösung eindeutig.
\begin{remark}
Die Matrix $A\in\mathbb{R}^{m\times n}$ ($m\ge n$) habe vollen Spaltenrang und es sei $b\in\mathbb{R}^m$. Dann ist im generischen Dall die Lösungsmenge des Gleichungssystems $A x= b$
leer. Die Aufgabe der Bestimmung der Quadratmittellösung, i.Z.
\begin{align}
	A x & \cong b\quad \text{bzw.}\quad \left\| A x -b \right\|_2\rightarrow \min,
\end{align}
hat dann die eindeutige Lösung $x^{\ast}=A^+b$. Die Matrix $A^+:=\left(A^TA\right)^{-1}A^T$ heisst \textit{Pseudoinverse} zu $A$ und die Matrix $P:=A A^+$ heißt Projektor auf das Bild
$\im A:=\left\{ A x : x\in\mathbb{R}^n \right\}$ von $A$. Die Berechnung von $x^{\ast}$ kann z.B. mit der Singulärwertzerlegung oder der QR-Zerlegung von $A$ erfolgen. In \textsc{Matlab}
verwendet man zur Bestimmung der Quadratmittellösung $x^{\ast}$ den Backslash-Operator: $x^{\ast}=A\backslash b$.
\end{remark} 
\begin{figure}[htb]
	\centering
	\begin{tikzpicture}[auto, >=latex'] 
	\node [input] () {};
\end{tikzpicture}

	\caption{Interpretation der Quadratmittellösung $A x\cong b$}
	\label{fig:kap_4_interp_quadratmittelloesung}
\end{figure}
\begin{remark}[Struktur des \ac{MPR}-Reglers]
Seien $A\in\mathbb{R}^{m\times n}, B=\begin{bmatrix} b_1 & \ldots & b_p \end{bmatrix}\in\mathbb{R}^{m\times p},\alpha_i\in\mathbb{R} (i=1,\ldots,p)$. Weiter sei $x$ Lösung von
\begin{align*}
	A x & \cong B\begin{bmatrix}
	\alpha_1\\ \vdots\\ \alpha_p
	\end{bmatrix}.
\end{align*}
Dann gilt
\begin{align}
	x & = A\backslash(B\begin{bmatrix}
	\alpha_1 \\ \vdots\\ \alpha_p
	\end{bmatrix})= \begin{bmatrix}
	A\backslash b_1 & \ldots & A\backslash b_p
	\end{bmatrix}\begin{bmatrix}
	\alpha_1 \\ \vdots\\ \alpha_p
	\end{bmatrix}.\label{eqn:kap_4_berechnung_x}
\end{align}
Aus \eqnref{eqn:kap_4_quadratmittelproblem} folgt damit
\begin{align*}
	\Delta\mathcal{U}^{\ast}[k] & = \begin{bmatrix}
	\Delta u^{\ast}[k|k]\\ \vdots\\ \Delta u^{\ast}[k+H_u-1|k]
	\end{bmatrix}=\bar{K}\mathcal{E}[k]\quad\text{mit }\bar{K}:=\begin{bmatrix}
	\mathcal{Q}_y\Theta\\ \mathcal{Q}_{\Delta\mathcal{U}}
	\end{bmatrix}\backslash\begin{bmatrix}
	\mathcal{Q}_y\\ 0
	\end{bmatrix}.
\end{align*}
Sei $K:=\bar{K}(1:n_{\mathcal{U}},:)$, wobei $n_{\mathcal{U}}$ die Länge des Vektors $u[k]$ ist. Gemäß dem Schritt (S2) des Algorithmus der modellprädiktiven Regelung aus
\secref{subsec:grundprinzip_mpr} ist die im Schritt $k$ ausgegebene Steuergröße gleich
\begin{align*}
	\Delta u[k] & = \Delta u^{\ast}[k|k] = K\mathcal{E}[k]=K\left(\mathcal{T}[k]-\Psi x[k]-\Upsilon u[k-1]\right).
\end{align*}
Die Multiplikation der Signale $\mathcal{T}[k]$, $x[k]$ und $u[k-1]$ mit zeitunabhängigen Matrizen zur Ermittlung der optimalen Steuergröße $u^{\ast}[k]$ stellt ein lineares und
zeitinvariantes Regelgesetz dar. Die Struktur des realisierten Reglers ist in der \figref{fig:kap_4_interp_quadratmittelloesung} dargestellt. Man stellt fest, dass durch die geschickte
Nutzung der Gleichung \eqnref{eqn:kap_4_berechnung_x} zur Berechnung der optimalen Steuergröße im jeweils aktuellen Schritt keine Online-Optimierung mehr erforderlich ist.
\end{remark}

\subsection{Aufgabe mit $\|\cdot\|_2$-Funktionial und Beschränkung}
\label{subsec:aufgabe_quad_mit_beschr}
Die in \secref{subsec:aufgabe_quad_ohne_beschr} betrachtete Aufgabe wird nun auf den Fall beschränkter Steuer- und Regelgrößen erweitert. Die Aufgabe besteht nun in der
Minimierung von 
\begin{align}
V[k] & = \left\| \begin{bmatrix}
	\mathcal{Q}_y \left(\Theta\Delta\mathcal{U}[k]-\mathcal{E}[k] \right)\\
	\mathcal{U}_{\Delta\mathcal{Q}}\Delta\mathcal{U}[k]
\end{bmatrix} \right\|_2^2
\end{align}
\begin{figure}[htb]
	\centering
	\begin{tikzpicture}[auto, >=latex'] 
	\node [input] () {};
\end{tikzpicture}

	\caption{Struktur den \ac{MPR}-Reglers zur Minimierung von \eqnref{eqn:kap_4_quad_guetefunktional} bei unbeschränkten Steuer- und Regelgrößen}
	\label{fig:kap_4_grundprinzip_mpr}
\end{figure}
unter der Nebenbedingung \eqnref{eqn:kap_4_lin_nebenbedingungen}, d.h.
\begin{align}
	M[k]\begin{bmatrix}
	\mathcal{Z}\\ \Delta\mathcal{U}[k]\\ \mathcal{U}[k]\\ 1
	\end{bmatrix}\le 0. \label{eqn:kap_4_nebenbedinungen_beschr}
\end{align}
Mit $\mathcal{Z}[k]=\Psi x[k]+\Upsilon u[k-1]+\Theta\Delta\mathcal{U}[k]$ und
\begin{align*}
\mathcal{U}[k] & = \begin{bmatrix}
I & \ldots & 0\\
\vdots & \ddots & \cdots\\
I & \ldots & I
\end{bmatrix}\Delta\mathcal{U}[k]+\begin{bmatrix}
u[k-1]\\ \vdots\\ u[k-1]
\end{bmatrix}
\end{align*}
kann man \eqnref{eqn:kap_4_nebenbedinungen_beschr} in die Form
\begin{align}
	\Xi\Delta\mathcal{U}[k] & \le \xi\label{eqn:kap_4_lin_nebenbedinung_form}
\end{align}
mit einer Matrix $\Xi=\Xi[k]$ und einem Vektor $\xi = \xi[k]$ überführen. Weiter gilt füt die Zielfunktion
\begin{align*}
	V[k] & = \left( \Theta\Delta\mathcal{U}[k]-\mathcal{E}[k] \right)^T\mathcal{Q}_y^T\mathcal{Q}_y\left( \Theta\Delta\mathcal{U}[k]-\mathcal{E}[k]
	\right)+\Delta\mathcal{U}[k]^T\mathcal{Q}^T_{\Delta\mathcal{U}}\mathcal{Q}_{\Delta\mathcal{U}}\Delta\mathcal{U}[k]\\
	& = \Delta\mathcal{U}[k]^T H \Delta\mathcal{U}[k]-\Delta\mathcal{U}[k]^TG+F
\end{align*}
mit $H:=\Theta^T\mathcal{Q}_y^T\mathcal{Q}_y\Theta+\mathcal{Q}^T_{\Delta\mathcal{U}}\mathcal{Q}_{\Delta\mathcal{U}}$, $G:=2\Theta^T\mathcal{Q}_y^T\mathcal{Q}_y\mathcal{E}[k]$ und
$F:=\mathcal{E}[k]^T\mathcal{Q}_y^T\mathcal{Q}_y\mathcal{E}[k]$. Also ist $V[k]$ eine quadratische Zielfunktion und mit der linearen Nebenbedingung
\eqnref{eqn:kap_4_lin_nebenbedinung_form} ist
\begin{align}
	\min\limits_{\Delta\mathcal{U}[k]}V[k] & \text{ bei } \Xi\Delta\mathcal{U}[k]\le \xi\label{eqn:kap_4_quad_opt_aufgabe_lin_nebenbedingungen}
\end{align}
eine Aufgabe der Quadratischen Optimierung. Die Lösung kann im Allgemeinen nur durch numeirsche Verfahren bestimmt werden. Der zulässige Bereich der Aufgabe ist aufgrund der
ausschließlich linearen Nebenbedingungen konvex und es gibt weiter
\begin{align*}
\frac{\d^2 V[k]}{\d\Delta\mathcal{U}[k]^2} & = 2 H \ge 0.
\end{align*}
Damit ist \eqnref{eqn:kap_4_quad_opt_aufgabe_lin_nebenbedingungen} ein konvexes Optimeirungsproblem, d.h. falls eine Lösung dieser Aufgabe exisitert, ist jede lokale Minimalstelle auch
stets eine globale Minimalstelle. Darüber hinaus ist aufgrund von $\mathcal{Q}_{\Delta\mathcal{U}}>0$ sogar $H>0$ erfüllt und daher ist die Zielfunktion $V[k]$ streng konvex, d.h. im
Falle der Existenz ist die Lösung von \eqnref{eqn:kap_4_quad_opt_aufgabe_lin_nebenbedingungen} dann sogar eindeutig. Unter diesen Umständen kann die Lösung $\Delta\mathcal{U}^{\ast}[k]$
der Aufgabe \eqnref{eqn:kap_4_quad_opt_aufgabe_lin_nebenbedingungen} numerisch effizient mit Algorithmen der Quadratischen Optimierung bestimmt werden\footnote{In \textsc{Matlab} ist zur Lösung
der Aufgabe der Quadratischen Programmierung die Routine \lstinline[columns=fixed]{quadprog} implementiert, die speziell für den durch Ungleichungsbeschränkungen gekennzeichneten
Problemtyp \eqnref{eqn:kap_4_quad_opt_aufgabe_lin_nebenbedingungen} einer Aktiven-Mengen-Strategie nach \cite{Gill1981} verwendet. Der Aufruf erfolgt mit
$\Delta\mathcal{U}[k]=\text{\lstinline[columns=fixed]{quadprog}}(2H,G,\Xi,\xi)$}.

\subsection{Aufgabe mit $\|\cdot\|_1$- und $\|\cdot\|_{\infty}$-Funktionial}
Bei Aufgaben mit dem Funktional \eqnref{eqn:kap_4_eins_norm_guetefunktional} oder \eqnref{eqn:kap_4_max_norm_guetefunktional_1} bzw. \eqnref{eqn:kap_4_max_norm_guetefunktional_2} muss
man nicht zwischen dem Fall unbeschränkter Steuer- und Regelgrößen und dem Fall beschränkter Steuer- und Regelgrößen unterscheiden, da sich in jedem Fall ein Optimierungsproblem mit
Beschränkungen ergibt. Formt man die Zielfunktion $V[k]$ analog dem \secref{subsec:aufgabe_quad_mit_beschr} um, so erhält man für das Gütefunktional
\eqnref{eqn:kap_4_eins_norm_guetefunktional}
\begin{align}
V[k] & = \left\| \begin{bmatrix}
\mathcal{Q}_y\left( \Theta\Delta\mathcal{U}[k]-\mathcal{E}[k] \right)\\ \mathcal{Q}_{\Delta\mathcal{U}}\Delta\mathcal{U}[k]
\end{bmatrix} \right\|_1 = \left\| \begin{bmatrix}
\mathcal{Q}_y\Theta\\ \mathcal{Q}_{\Delta\mathcal{U}}
\end{bmatrix}\Delta\mathcal{U}[k]-\begin{bmatrix}
\mathcal{Q}_y\mathcal{E}[k]\\ 0
\end{bmatrix} \right\|_1
\end{align}
und für die Gütefunktionale \eqnref{eqn:kap_4_max_norm_guetefunktional_1} und \eqnref{eqn:kap_4_max_norm_guetefunktional_2} erhält man
\begin{align}
	V[k] & = \left\| \mathcal{Q}_y\Theta\Delta\mathcal{U}[k]-\mathcal{Q}_y\mathcal{E}[k] \right\|_{\infty}
\end{align}
bzw.
\begin{align}
	V[k] & = \left\| \mathcal{Q}_{\Delta\mathcal{U}}\Delta\mathcal{U}[k] \right\|_{\infty}
\end{align}
Die Minimierung von $V[k]$ unter der Nebenbedingung \eqnref{eqn:kap_4_lin_nebenbedinung_form} kann dann als Aufgabe der Linearen Optimierung dargestellt werden\footnote{Die numerische
Lösung von Aufgaben der Linearen Optimierung kann wie für Aufgaben der Quadratischen Optimierung durch eine Aktive-Mengen-Strategie erfolgen. In \textsc{Matlab} erfolgt die Lösung
von $\min\limits_{x\in\mathbb{R}^n}c^Tx$ bei $Ax\le b$ durch $x=\text{\lstinline[columns=fixed]{linprog}}(c,A,b)$}, denn mit Matrizen $F$ und $C$ sowie Vektoren $f$ und $d$ passender
Dimension kann die Aufgabe
\begin{align*}
	\min\limits_{x\in\mathbb{R}^n}\left\| F x-f \right\|_1
\end{align*}
durch
\begin{align*}
	\min\limits_{\substack{\mu\in\mathbb{R}^1\\\eta\in\mathbb{R}^m\\x\in\mathbb{R}^n }}\mu & \text{ bei } \mu\ge 0,\ \unity^T\eta\le \mu,\ -\eta\le Fx-f\le\eta,\ Cx\le d
\end{align*}
und die Aufgabe 
\begin{align*}
	\min\limits_{x\in\mathbb{R}^n}\left\| F x-f \right\|_{\infty}
\end{align*}
durch
\begin{align*}
	\min\limits_{\substack{\delta\in\mathbb{R}^1\\x\in\mathbb{R}^n }}\delta & \text{ bei } \delta\ge 0,\ -\unity^T\delta\le Fx-f\le\unity\delta,\ Cx\le d
\end{align*}
gelöst werden (Beweis Übungsaufgabe!). Hierbei ist $\unity:=\begin{bmatrix}1 & \ldots & 1 \end{bmatrix}^T$.

\section{Stabilität prädiktiver Regelungen}
\subsection{Problembeschreibung}
\label{subsec:kap_4_problembeschreibung}
Der in \secref{subsec:grundprinzip_mpr} beschriebene prinzipielle Algorithmus der modellprädiktiven Regelung stellt eine rückführung dar, womit potenziell die Gefahr der Instabilität
des entstehenden Regelkreises besteht. Zur Erläuterung wird als eine einfache Form des quadratischen Gütefunktionals das Funktional
\begin{align}
	V[k] & = \sum\limits_{i=1}^{H_p}\left\| \hat{x}[k+i|k] \right\|_2^2
\end{align}
betrachtet, wobei für die Länge des Steuer- und des Prädiktionshorizonts $H_u=H_p$ gelten soll. Beschränkungen hinsichtlich der Steuer- und Zustandsgröße sollen nicht vorliegen. Im
Zeitpunkt $k$ wird die optimale Steuerfolge $(\Delta u^{\ast}[k+i|k])_{i=0,\ldots,H_p-1}$ berechnet, die dann die optimale Zustandstrajektorie $(x^{\ast}[k+i|k])_{i=1,\ldots,H_p}$
liefert. Gemäß dem Algorithmus der modellprädiktiven Regelung wird dann $\Delta u[k]:=\Delta u^{\ast}[k|k]$ gesetzt und mit dieser Steuergröße ergibt sich (unter der Annahme das keine
Störgrößen oder Modellunbestimmtheiten auftreten) der Folgezustand $x[k+1]=x^{\ast}[k+1|k]$. Würde man dann im Zeitpunkt $k+1$ das Gütefunktional
\begin{align}
	\tilde{V}[k] & = \sum\limits_{i=1}^{H_p-1}\left\| \hat{x}[k+1+i|k+1] \right\|_2^2
\end{align}
minimieren, so würde nach dem \textsc{Bellman}schen Optimalitätsprinzip für die optimale Zustandstrajektorie $\tilde{x}^{\ast}[k+1+i|k+1]=x^{\ast}[k+1+i|k]$ ($i=1,\ldots,H_p-1$) gelten.
Jedoch löst man im Zeitpunkt $k+1$ die Aufgabe mit dem Gütefunktional
\begin{align}
	V[k+1] & = \sum\limits_{i=1}^{H_p}\left\| \hat{x}[k+1+i|k+1] \right\|_2^2
\end{align}
und bestimmt damit den im Allgemeinen von $\tilde{x}^{\ast}[k+1+i|k+1]$ abweichenden Verlauf $x^{\ast}[k+1+i|k+1]$. Weiter gilt dann $x[k+2]=x^{\ast}[k+2|k+1]$. Setzt man diese
Überlegung fort, so erhält man eine Trajektorie $x[k+i]$ ($i=1,\ldots,H_p$) des Zustandes im \ac{MPR}-Regelkreis, die im Normalfall von der zum Zeitpunkt $k$ bestimmten optimalen
Prädiktion $x^{\ast}[k+i|k]$ abweichen wird (siehe \figref{fig:kap_4_abweichung_opt_tatsaechlich_verlauf}).

Hätte der Prädiktionshorizont eine unendliche Länge, so würde der tatsächliche Verlauf der Zustandstrajektorie $x[k+i]$ mit dem Verlauf der optimalen Prädiktion $x^{\ast}[k+i|k]$
übereinstimmen (falls keine Störgrößen wirken und keine Modellunbestimmtheiten auftreten). Bei einer endlichen Länge werden $x[k+i]$ und $x^{\ast}[k+i|k]$ jedoch im Allgemeinen
voneinander verschieden sein, so dass sich dann nicht nur die Güte des Regelkreises (gegenüber der aufgrund der Vorhersage $x^{\ast}[k+i|k]$ über dem Prädiktionshorizont erwarteten
Güte) verschlechtern wird, sondern sogar ein instabiler Regelkreis resultieren kann.
\begin{figure}[htb]
	\centering
	\begin{tikzpicture}[auto, >=latex'] 
	\node [input] () {};
\end{tikzpicture}

	\caption{Zur Abweichung zwischen der optimalen Trajektorie $x^{\ast}[k+i|k]$ und dem tatsächlichen Verlauf $x[k+i]$ ($i=1,\ldots,H_p$)}
	\label{fig:kap_4_abweichung_opt_tatsaechlich_verlauf}
\end{figure}
\begin{exmp}
Gegeben seien ein stabiler Prozess
\begin{align*}
	x[k+1] & = \begin{bmatrix}
	0 & 0\\ 1 & 0
	\end{bmatrix}x[k]+\begin{bmatrix}
	1\\ 0
	\end{bmatrix}u[k]
\end{align*}
und das quadratische Gütefunktional
\begin{align*}
	V[k] & = \hat{x}[k+1|k]^T\begin{bmatrix}
	1 & 2\\ 2& 6
	\end{bmatrix}\hat{x}[k+1|k].
\end{align*}
Beschränkungen seien nicht vorhanden. Mit den Prozessgleichungen $\hat{x}_1[k+1|k]=\hat{u}[k|k]$ und $\hat{x}_2[k+1|k]=x_1[k|k]=x_1[k]$ erhält man
\begin{align*}
	V[k] & = \hat{x}_1[k+1|k]^2+6\hat{x}_2[k+1|k]^2+4\hat{x}_1[k+1|k]\hat{x}_2[k+1|k]\\
		 & = \hat{u}[k|k]^2+6x_1[k]^2+4\hat{u}[k|k]x_1[k]
\end{align*}
und weiter
\begin{align*}
\frac{\d V[k]}{\d\hat{u}[k|k]} & = 2\hat{u}[k|k]+4 x_1[k]=0\quad \Leftrightarrow\quad \underbrace{\hat{u}[k|k]}_{=:u^{\ast}[k]}=-2x_1[k].
\end{align*}
Der mit diesem (optimalen) Regelgesetz geschlossene Regelkreis wird durch die Differenzengleichung $x[k+1]=\begin{bmatrix}-2 & 0\\ 1 & 0 \end{bmatrix}x[k]$ beschrieben und ist
offensichtlich instabil.
\end{exmp}

\subsection{Stabilität des geschlossenen Kreises bei endlichem Optimierungshorizont durch eine Endbedingung für den Prozesszustand}
\label{subsec:kap_4_stab_geschl_rk}
Der folgende Satz liefert eine erste Aussage über die Stabilität einer modellprädiktiven Regelung.
\begin{satz}\label{satz:kap_4_stabilitaet}
Gegeben seien der Prozess $x[\kappa +1]=f(x[\kappa],u[\kappa])$ mit stetigem $f$ und einem stationären Punkt für $(x[\kappa],u[\kappa])=(0,0)$. Das Gütefunktional sei
\begin{align*}
	V[k] & = \sum\limits_{i=1}^{H_p}l\left( \hat{x}[k+i|k],\hat{u}[k+i-1|k] \right)
\end{align*}
mit stetigem $l,l(x,u)\ge 0$ und $l(x,u)=0$ genau dann, wenn $(x,u)=(0,0)$. Als Endbedingung sei $\hat{x}[k+H_p|k]=0$ gegeben und die Beschränkungen seien $\hat{u}[k+i|k]\in U$ und
$\hat{x}[k+i|k]\in X$ für gegebene Mengen $U$ und $X$. Für jedes $k$ existiert eine optimale Stellfolge $\left(u^{\ast}[k+1|k]\right)_{i=0,1,\ldots,H_p-1}$ und es sei $u^{\ast}[k|k]$
die im Schritt $k$ auf den Prozess aufgeschaltete Steuergröße.

Dann ist $(x,u)=(0,0)$ eine asymptotische stabile Gleichgewichtslage des geschlossenen Kreises.
\end{satz}
\begin{proof}
O.B.d.A. sei $f(x[k],0)\neq 0$ für $x[k]\neq 0$. Die Funktion
\begin{align*}
	\left( x[k],\hat{u}[k|k],\ldots,\hat{u}[k+H_p-1|k] \right) \mapsto V[k]
\end{align*}
ist stetig und damit ist der Minimalwert
\begin{align*}
	\mathcal{V}(x[k]) & := \min\limits_{\hat{u}[k|k],\ldots,\hat{u}[k+H_p-1|k]}V[k]
\end{align*}
stetig in $x[k]$. Wir zeigen jetzt, das $\mathcal{V}$ eine strikte Ljapunov-Funktion für $x[k+1]=f(x[k],u^{\ast}[k|k])$ ist.
\begin{enumerate}
  \item "`$\mathcal{V}$ ist positiv definit"':
  \begin{itemize}
    \item Aus $l\ge 0$ folgt die Ungleichung $\mathcal{V}(x)\ge 0$.
    \item Sei $x[k]=0$. Setze $u^{\ast}[k|k]:=0$, dann folgt $x^{\ast}[k+1|k]=0$, setze $u^{\ast}[k+1|k]:=0$, dann ist $x^{\ast}[k+2|k]=0$ usw. Somit gilt $\mathcal{V}(x[k])=0$.\\
    Umgekehrt sei $\mathcal{V}(x[k])=0$. Wegen $f(x[k],0)\neq 0$, $l\ge 0$ und $l=0 \Leftrightarrow (x,u)=(0,0)$ ist dann $x[k]=0$.\\
    Also gilt $\mathcal{V}(x)\Leftrightarrow x=0$.
  \end{itemize}
  \item "`$\mathcal{V}(x[k+1])<\mathcal{V}(x[k])$ für $x[k]\neq 0$"':
  \begin{align*}
  	\mathcal{V}(x[k+1]) & = \min\limits_{\hat{u}[k+1|k],\ldots,\hat{u}[k+H_p|k]}\sum\limits_{i=1}^{H_p}l\left( \hat{x}[k+i+1|k],\hat{u}[k+i|k] \right)\\
  	& = \min\limits_{\ldots}\left\{ \sum\limits_{i=1}^{H_p}l\left( \hat{x}[k+i|k],\hat{u}[k+i-1|k] \right) - l\left( \hat{x}[k+1|k],\hat{u}[k|k] \right) + \right. \\
  	& \qquad \qquad \qquad \left.\vphantom{\sum\limits_{i=1}^{H_p}} l\left( \hat{x}[k+H_p+1|k],\hat{u}[k+H_p|k] \right) \right\}\\
  	& \le \min\limits_{\hat{u}[k+1|k],\ldots,\hat{u}[k+H_p-1|k]}\left\{ \sum\limits_{i=1}^{H_p}l\left( \hat{x}[k+i|k],\hat{u}[k+i-1|k] \right) - \right.\\
  	& \qquad \qquad \qquad\qquad \qquad\qquad\quad \left.\vphantom{\sum\limits_{i=1}^{H_p}} l\left( \hat{x}[k+1|k],\hat{u}[k|k] \right) \right\}+\\
  	& \qquad \qquad \min\limits_{\hat{u}[k+H_p|k]}\sum\limits_{i=1}^{H_p}l\left( \hat{x}[k+H_p+1|k],\hat{u}[k+H_p|k] \right)\\
  	& = \underbrace{ \min\limits_{\hat{u}[k|k],\hat{u}[k+1|k],\ldots,\hat{u}[k+H_p-1|k]}V[k] }_{=\mathcal{V}(x[k])}-\underbrace{ l\left( \hat{x}[k+1|k],u^{\ast}[k|k] \right) }_{>0\text{
  	wegen }f(x[k],0)\neq 0}+\\
  	& \qquad\qquad \underbrace{ \min\limits_{\hat{u}[k+H_p|k]}\sum\limits_{i=1}^{H_p}l\left( \hat{x}[k+H_p+1|k],\hat{u}[k+H_p|k] \right) }_{=0}\\
  	& < \mathcal{V}(x[k]).
  \end{align*}
\end{enumerate}
\end{proof}
\begin{remark}
Für den \satzref{satz:kap_4_stabilitaet} gibt es allgemeinere Formulierungen. So kann dieser Satz z.B. auf die Endbedingung $\hat{x}[k+H_p|k]=r[k+H_p|k]\neq 0$, d.h. auf den Dall
beliebiger Referenztrajektorien, und auf die Bewertung der Steuergrößendifferenz $\Delta u$ (anstelle bzw. zusätzlich zur Steuergröße $u$) erweitert werden.
\end{remark}
Die Voraussetzung der Existenz einer optimalen Steuerfolge mit der in $H_p$ Schritten der Zustand $x=0$ erreicht wird, ist eine sehr starke Voraussetzung, die z.B. bei einem kurzen
Horizont $H_p$ und bei Steuergrößenbeschränkungen häufig nicht erfüllt ist. Wie im nächsten Abschnitt gezeigt wird, kann diese Voraussetzung insbesondere für lineare Prozesse
abgeschwächt werden, indem der Endzustand lediglich geeignet gewichtet wird.

\subsection{Stabilität des geschlossenen Kreises bei endlichem Steuerhorizont durch eine Gewichtung des Endzustandes}
Wie betrachten zunächst die Aufgabe mit quadratischem Gütefunktional bei unendlichem Optimierungshorizont (d.h. einem unendlichen Steuer- und Prädiktionshorizont)
\begin{align}
	V[k] & = \sum\limits_{i=1}^{\infty}\left\| Q_x\hat{x}[k+i|u] \right\|_2^2 + \sum\limits_{i=0}^{\infty}\left\|Q_{\Delta u}[k+i|k] \right\|_2^2\rightarrow \min .
\end{align}
Nach den Überlegungen in \secref{subsec:kap_4_problembeschreibung} gilt aufgrund des unendlichen Horizonts
\begin{align}
	x[k+i] & = x^{\ast}[k+i|k]\qquad (i=1,2,\ldots),\label{eqn:kap_4_x_normal_geschaetzt}
\end{align}
d.h. es kann aus dem Verlauf der optimalen Prädiktion auf das Verhalten (speziell auf die Stabilität) des \ac{MPR}-Regelkreises geschlossen werden, denn falls $V[k]$ einen endlichen
Wert aufweist, so impliziert dies $x[k+i]\rightarrow 0$ für $i\rightarrow\infty$.
\begin{figure}[htb]
	\centering
	\begin{tikzpicture}[auto, >=latex'] 
	\node [input] () {};
\end{tikzpicture}

	\caption{Optimale Trajektorie $x^{\ast}[k+i|k]$ und dem tatsächlicher Verlauf $x[k+i]$ bei unendlichem Horizont}
	\label{fig:kap_4_abweichung_opt_tatsaechlich_verlauf_unendl}
\end{figure}
Betrachtet man hingegen einen endlichen Steuerhorizont und einen unendlichen Prädiktionshorizont
\begin{align}
	V[k] & = \sum\limits_{i=1}^{\infty}\left\| Q_y\hat{y}[k+i|k] \right\|_2^2 + \sum\limits_{i=0}^{H_u-1}\left\| Q_{\Delta u}[i]\Delta \hat{u}[k+i|k]\right\|_2^2\rightarrow\min .
\end{align}
so gilt aufgrund von $H_u<\infty$ die Gleichung \eqnref{eqn:kap_4_x_normal_geschaetzt} im Allgemeinen nicht mehr. Im Weiteren wird vorausgesetzt, dass der Prozess stabil sei, das Paar
$(A,Q_yC_y)$ beobachtbar ist und für die Steuergröße am Ende des Steuerhorizontes $\hat{u}[k+H_u|k]=0$ gilt, was dann wegen \eqnref{eqn:kap_4_steuerhorizont} auch
$\hat{u}[k+H_u+i|k]=0\forall i\ge 0$ impliziert. Somit gelten
\begin{align*}
	\hat{y}[k+H_u+i|k] & =  C_yA^i\hat{x}[k+H_u|k]\qquad (i=1,2,\ldots)
\end{align*}
und
\begin{align*}
	\sum\limits_{i=H_u}^{\infty}\left\| Q_y\hat{y}[k+i|k]\right\|_2^2 & = \hat{x}[k+H_u|k]^T\underbrace{ \left(\sum\limits_{i=0}^{\infty}\left(A^T\right)^iC_y^TQ_y^TQ_yC_yA^i\right)
	}_{=:\bar{Q}}\hat{x}[k+H_u|k].
\end{align*}
Da die Matrix $A$ stabil ist, existiert die Matrix $\bar{Q}$ und kann als eindeutige Lösung der Ljapunov-Gleichung
\begin{align}
	A^T\bar{Q}A & = \bar{Q}-C_y^tQ_y^TQ_yC_y \label{eqn:kap_4_ljapunov_glg}
\end{align}
bestimmt werden. Da $(A,Q_yC_y)$ beobachtbar ist, ist $\bar{Q}$ positiv definit. Damit gilt
\begin{align}
\begin{split}\label{eqn:kap_4_guetefunktional_bedingung}
	V[k] & = \hat{x}[k+H_u|k]^T\bar{Q}\hat{x}[k+H_u|k]\\
	& \quad + \sum\limits_{i=1}^{H_u-1}\left\| Q_y\hat{y}[k+i|k]\right\|_2^2+\sum\limits_{i=0}^{H_u-1}\left\| Q_{\Delta u}[i]\Delta \hat{u}[k+i|k]\right\|_2^2 < +\infty  
\end{split}
\end{align}
also ist insbesondere auch
\begin{align}
	\mathcal{V}(x[k]) & := \min\limits_{\Delta\hat{u}[k|k],\ldots,\Delta\hat{u}[k+H_u-1|k]}V[k] < +\infty.
\end{align}
Man findet 
\begin{align*}
	\mathcal{V}(x[k+1]) & \le \mathcal{V}(x[k])-\left\| Q_y z^{\ast}[k+1|k]\right\|_2^2-\left\| Q_{\Delta u}\Delta u^{\ast}[k|k]\right\|_2^2.
\end{align*}
also gilt $\mathcal{V}(x[k+1])\le \mathcal{V}(x[k])$. Weiter wird in ? gezeigt, dass es im Fall $x[k]\neq 0$ ein $i_0\in\left\{ 1,\ldots,n_x \right\}$ gibt, wobei $n_x$ die Länge des
Vektors $x[k]$ bezeichnet, so dass sogar $\mathcal{V}(x[k+i_0])<\mathcal{V}(x[k])$ erfüllt ist. Wegen $\mathcal{V}\ge 0$ und $\mathcal{V}$ stetig, ist $\mathcal{V}$ eine
Ljapunov-Funktion zur Gleichgewichtslage $x=0$. Und bei geeigneter Indizierung ist $\mathcal{V}$ sogar eine strikte Ljapunov-Funktion. Weiter folgt aus $\left\| x[k]
\right\|_2\rightarrow+\infty$ dann $\mathcal{V}(x[k])\rightarrow +\infty$. Also ist der geschlossene Regelkreis asymptotisch stabil!
\begin{remark}
Wegen \eqnref{eqn:kap_4_guetefunktional_bedingung} ergibt sich eine Aufgabe mit endlichem Prädiktionshorizont bzw. Steuerhorizont und einer Gewichtung des Endzustandes durch die
positiv definite Matrix $\bar{Q}$. Diese Gewichtungsmatrix kann gemäß \eqnref{eqn:kap_4_ljapunov_glg} durch das \textsc{Matlab}-Kommando
$\bar{Q}=\text{\lstinline[columns=fixed]{dlyap}}\left(A^T,C_y^TQ_y^TQ_yC_y\right)$ bestimmt werden. Die Gewichtung des Endzustandes ist praktisch eine wesentlich schwächere
Voraussetzung als das Einführen einer Endbedingung der Form $\hat{x}[k+H_u|k]=0$ wie in \secref{subsec:kap_4_stab_geschl_rk}.\\
Die beschriebene Vorgehensweise kann auf instabile Prozesse, auf von Null verschiedene Referenztrajektorien, auf beschränkte Steuer- und Regelgrößen sowie auf $\|.\|_1$- und
$\|\cdot\|_{\infty}$-Gütefunktionale erweitert werden.
\end{remark}



% Anhang (Bibliographie darf im deutschen nicht in den Anhang!)
\bibliography{bib/quellen}
\cleardoublepage

% Anhang
\backmatter
\appendix
\chapter{Übungen}
\label{cha:anhang_uebungen}

\section*{Übungen Kapitel 1}
\addcontentsline{toc}{section}{Übungen Kapitel 1} 
\label{sec:uebung_kapitel_1}

\subsection*{Definitheit und Extremstellen}
\addcontentsline{toc}{subsection}{Übung - Definitheit und Extremstellen}
\label{sec:uebung_kapitel_1_def_extrem} 
Bestimmen Sie von beiden Funktionen die Definitheit und die Extremstellen!
\begin{alignat*}{2}
f_1(x_1,x_2) & = \frac12\left(ax_1^2+bx_2^2\right) &\Rightarrow H f_1&=\begin{bmatrix} a & 0\\ 0 & b \end{bmatrix}\\
f_2(x_1,x_2) & = \frac12x_1^2+\frac16x_2^3 & \Rightarrow H f_2&=\begin{bmatrix} 1 & 0\\ 0 & x_2 \end{bmatrix}
\end{alignat*}

\subsubsection{Lösung}
 
\subsection*{Kuhn-Tucker-Bedingung}
\addcontentsline{toc}{subsection}{Kuhn-Tucker-Bedingung}
\label{sec:uebung_kapitel_1_ktb} 
Finden Sie alle Lösungen von $f(x)=-x_1-x_2\rightarrow\min\limits_{x\in\mathbb{R}^2}$ bei $g_1(x)=x_1+x_2^2\le 0$! Skizzieren Sie ihr Ergebnis.

\subsubsection{Lösung} 

\subsection*{Beispiele zur Quadratischen Optimierung}
\addcontentsline{toc}{subsection}{Beispiele zur Quadratischen Optimierung}
\label{sec:uebung_bsp_quad_opt}
Seien $Q=\begin{bmatrix}
1 & 1\\ 1 & 2
\end{bmatrix}$, $q=\begin{bmatrix}
2\\1
\end{bmatrix}$, $C = \begin{bmatrix}
1 & 1
\end{bmatrix}$ und $d = 1$. Lösen Sie die folgenden Aufgaben
\begin{enumerate}[label=\alph*)]
  \item $\min\limits_{x\in\mathbb{R}^n}\frac12 x^TQx + q^T x$
  \item $\min\limits_{x\in\mathbb{R}^n}\frac12 x^TQx +q^T x$ bei $Cx=d$
\end{enumerate}

\subsubsection{Lösung}

\subsection*{Projektion auf Untervektorraum}
\addcontentsline{toc}{subsection}{Projektion auf Untervektorraum}
\label{sec:uebung_proj_untervekraum} 
Seien $A=\begin{bmatrix}
							2\\1
							\end{bmatrix}$, $b=\begin{bmatrix}
							1\\2
							\end{bmatrix}$. Berechnen Sie 							
\begin{enumerate}[label=\alph*)]
	\item $Pb$. 
	\item $(I-P)b$.
	\item $(I-P)P$.
	\item $Pb+(I-P)b$.
	\item $(Pb)^T(I-P)b$.
	\item Was ist das kleinste $\alpha\in\mathbb{R}^1$, so dass $\|Pb\|_2\le\alpha\|b\|_2$ erfüllt ist? 
\end{enumerate}  

\subsubsection{Lösung} 

\subsection*{Projektion eines Vektors}
\addcontentsline{toc}{subsection}{Projektion eines Vektors}
\label{sec:uebung_proj_vektor} 
\begin{enumerate}[label=\alph*)]
  \item Für $C=\begin{bmatrix}1 & 1 \end{bmatrix}$, $d=1$ projiziere $v_1=\begin{bmatrix}1 \\ 1 \end{bmatrix}$, $v_2=\begin{bmatrix}1 \\ 0 \end{bmatrix}$ auf $M$ (Skizze)!
  \item Für $C=\begin{bmatrix}1 & 1\\ 0 & 1 \end{bmatrix}$, $d=\begin{bmatrix} 1\\1\end{bmatrix}$ projiziere $v_1=\begin{bmatrix}1 \\ 1 \end{bmatrix}$, $v_2=\begin{bmatrix}1 \\ 0
  \end{bmatrix}$ auf $M$ (Skizze)!
\end{enumerate}

\subsubsection{Lösung} 

\subsection*{Finden eines zulässigen Startwertes für den Aktiven Mengen Algorithmus}
\addcontentsline{toc}{subsection}{Finden eines zulässigen Startwertes für den Aktiven Mengen Algorithmus}
\label{sec:uebung_finden_startwert} 
Erweitern Sie den Algorithmus um die Auffindung eines zulässigen Startpunktes!
 
\subsubsection{Lösung}
 
\subsection*{Erweiterung des Aktiven Mengen Algorithmuss}
\addcontentsline{toc}{subsection}{Erweiterung des Aktiven Mengen Algorithmus}
\label{sec:uebung_erweiterung_algo} 
Erweitern Sie den Algorithmus um die Aufgabe der linearen Optimierung und die Verwendung von GNB als aktive UNB!

\subsubsection{Lösung} 

\subsection*{Aktiver Mengen Algorithmus}
\addcontentsline{toc}{subsection}{Aktiver Mengen Algorithmus}
\label{sec:uebung_ama} 
Wiederholen Sie das \exmpref{exmp:kap_1_ama} mit dem Aktiven Mengen Algorithmus mit dem Startwert $x^0:=\begin{bmatrix}
1\\ -1
\end{bmatrix}$.
\subsubsection{Lösung} 


\section*{Übungen Kapitel 2}
\addcontentsline{toc}{section}{Übungen Kapitel 2} 
\label{sec:uebung_kapitel_2}

\subsection*{Vorgehensweise zur analytischen Bestimmung der Optimallösung}
\addcontentsline{toc}{subsection}{Vorgehensweise zur analytischen Bestimmung der Optimallösung}
\label{sec:uebung_anal_best_opt_lsg}
\begin{enumerate}
  \item Bestimmen Sie die optimale Steuerfunktion zu folgenden Problemen
  \begin{enumerate}
    \item $J = \int\limits_0^1 u(t)^2dt\rightarrow\min$ bei $\dot{x}=u-c$ mit gegebenen KOnstanten $c>0$ sowie $x(0)=0$ und $x(1)=1$.
    \item $J = \int\limits_0^{t_b} u(t)^2dt\rightarrow\min$ bei $\dot{x}=u-c$ mit gegebenen KOnstanten $c>0$ sowie $x(t_b)=0$ und $x(1)=1$, $t_b$ ist
    frei.
  \end{enumerate}
  Diskutieren Sie für die beiden Probleme jeweils die Abhängigkeit der Lösung, d.h. von $u$ und $J$, vom Parameter $c$. Vergleichen Sie die Ergebnisse
  der Lösung der beiden Probleme miteinander. 
  \item Bestimmen Sie eine optimale Steuerfunktion für die Aufgabenstellung
  \begin{align*}
  	J & = \int\limits_0^1 u_1(t)^2+u_2(t)^2 dt\rightarrow\min
  \end{align*} 
  bei
  \begin{align*}
  	\dot{x}_1 & = u_1,\ \dot{x}_2=u_2,\ \dot{x}_3=x_1,\ \dot{x}_4=x_2\\
  	x_3(1)-x_4(1) & = 0
  \end{align*}
  für die Anfangszustände
  \begin{enumerate}
    \item $x_1(0) = x_2(0)=x_3(0)=x_4(0)=0$,
    \item $x_1(0) = 2,\ x_2(0)=0,\ x_3(0)=1,\ x_4(0)=0$.
  \end{enumerate}
\end{enumerate}


\section*{Übungen Kapitel 3}
\addcontentsline{toc}{section}{Übungen Kapitel 3} 
\label{sec:uebung_kapitel_3}

\subsection*{Beispiel der räumlichen Führung}
\addcontentsline{toc}{subsection}{Beispiel der räumlichen Führung} 
\label{sec:uebung_raeumliche_fuehrung}
\begin{enumerate}
  \item Überlegen Sie, warum in \eqnref{eqn:kap_3_bsp_rf_bsp_konst_regler} eine konstante Rückführung mit $F_{12}$ anstelle von $F_{10}$ zu besseren Ergebnissen führen könnte.
  \item Praktisch steht die Gierwinkelrate $\Delta\dot{\gamma}$ nicht direkt als Steuergröße zur Verfügung. Vielmehr kann diese nur zeitlich verzögert beeinflusst werden, was über die
  Differentialgleichung $T\cdot\Delta\ddot{\gamma}+\Delta\dot{\gamma}=u$ mit einer Zeitkonstante $T$ (deutlich unter einer Sekunde) modelliert werden kann. Stellen Sie die lineare
  zeitvariante Zustandsdifferentialgleichung für das resultierende System 2. Ordung auf und machen Sie sich anhand eines Beispiels durch Simulation klar, dass ein zeitlich konstanter
  Regler im Vergleich zum zeitvarianten \ac{LQR}-Regler zu deutlichen Stabilitätsproblemen führt.
\end{enumerate}
\section*{Übungen Kapitel 4}
\addcontentsline{toc}{section}{Übungen Kapitel 4} 
\label{sec:uebung_kapitel_4}

\subsection*{Aufgabe 1}
\addcontentsline{toc}{subsection}{Aufgabe 1} 


\IfDefined{printindex}{\printindex}
\IfDefined{printnomenclature}{\printnomenclature}
\cleardoublepage
\leerseite{}
\cleardoublepage


\end{document}

